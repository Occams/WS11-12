%%%
% Excercise Template
%%%

%\listfiles 
\documentclass[12pt]{scrartcl} 
\usepackage[english,ngerman]{babel}
    
\usepackage[latin1]{inputenc}
\usepackage[T1]{fontenc}

\usepackage{graphicx,enumitem,wasysym,multicol,libertine,xcolor,microtype,lipsum,fixltx2e,
	fancyhdr,vmargin,calc,lastpage,hyperref,listings,lastpage,lipsum}

% ------------------------------------------
% -------- xcolor - (Tango)
% ------------------------------------------

\definecolor{LightButter}{rgb}{0.98,0.91,0.31}
\definecolor{LightOrange}{rgb}{0.98,0.68,0.24}
\definecolor{LightChocolate}{rgb}{0.91,0.72,0.43}
\definecolor{LightChameleon}{rgb}{0.54,0.88,0.20}
\definecolor{LightSkyBlue}{rgb}{0.45,0.62,0.81}
\definecolor{LightPlum}{rgb}{0.68,0.50,0.66}
\definecolor{LightScarletRed}{rgb}{0.93,0.16,0.16}
\definecolor{Butter}{rgb}{0.93,0.86,0.25}
\definecolor{Orange}{rgb}{0.96,0.47,0.00}
\definecolor{Chocolate}{rgb}{0.75,0.49,0.07}
\definecolor{Chameleon}{rgb}{0.45,0.82,0.09}
\definecolor{SkyBlue}{rgb}{0.20,0.39,0.64}
\definecolor{Plum}{rgb}{0.46,0.31,0.48}
\definecolor{ScarletRed}{rgb}{0.80,0.00,0.00}
\definecolor{DarkButter}{rgb}{0.77,0.62,0.00}
\definecolor{DarkOrange}{rgb}{0.80,0.36,0.00}
\definecolor{DarkChocolate}{rgb}{0.56,0.35,0.01}
\definecolor{DarkChameleon}{rgb}{0.30,0.60,0.02}
\definecolor{DarkSkyBlue}{rgb}{0.12,0.29,0.53}
\definecolor{DarkPlum}{rgb}{0.36,0.21,0.40}
\definecolor{DarkScarletRed}{rgb}{0.64,0.00,0.00}
\definecolor{Aluminium1}{rgb}{0.93,0.93,0.92}
\definecolor{Aluminium2}{rgb}{0.82,0.84,0.81}
\definecolor{Aluminium3}{rgb}{0.73,0.74,0.71}
\definecolor{Aluminium4}{rgb}{0.53,0.54,0.52}
\definecolor{Aluminium5}{rgb}{0.33,0.34,0.32}
\definecolor{Aluminium6}{rgb}{0.18,0.20,0.21}
\definecolor{Brown}{cmyk}{0,0.81,1,0.60}
\definecolor{OliveGreen}{cmyk}{0.64,0,0.95,0.40}
\definecolor{CadetBlue}{cmyk}{0.62,0.57,0.23,0}

% ------------------------------------------
% -------- vmargin
% ------------------------------------------

%\setmarginsrb{hleftmargini}{htopmargini}{hrightmargini}{hbottommargini}%{hheadheighti}{hheadsepi}{hfootheighti}{hfootskipi}
\setpapersize{A4}
\setmarginsrb{3cm}{1cm}{3cm}{1cm}{8mm}{4mm}{3mm}{15mm}

% ------------------------------------------
% -------- listings
% ------------------------------------------
 
\lstdefinestyle{inline} {
		basicstyle=\normalsize
		}

\lstset{
		breakautoindent=true,
		breakindent=2em,
		breaklines=true,
		tabsize=4,
		frame=blrt,
		frameround=tttt,
		captionpos=b,
		basicstyle=\scriptsize\ttfamily,
		keywordstyle={\color{SkyBlue}},
		%commentstyle={\color{OliveGreen}},
		stringstyle={\color{OliveGreen}},
		showspaces=false,
		%numbers=right,
		%numberstyle=\scriptsize,
		%stepnumber=1, 
		%numbersep=5pt,
		%showtabs=false
		prebreak = \raisebox{0ex}[0ex][0ex]{\ensuremath{\hookleftarrow}},
		aboveskip={1.5\baselineskip},
		columns=fixed,
		upquote=true,
		extendedchars=true
		}


% ------------------------------------------
% -------- hyperref
% ------------------------------------------

\hypersetup{
	%breaklinks=true,
	pdfborder={0 0 0},
	bookmarks=true,         % show bookmarks bar?
	unicode=false,          % non-Latin characters in Acrobat�s bookmarks
	pdftoolbar=true,        % show Acrobat�s toolbar?
	pdfmenubar=true,        % show Acrobat�s menu?
	pdffitwindow=true,     % window fit to page when opened
	pdfstartview={FitH},    % fits the width of the page to the window
	pdftitle={Bachelor Thesis - Optimizing SPARQL Queries on Behalf of the MPEG Query Format},    % title
	pdfauthor={Daniel Watzinger},     % author
    pdfsubject={Bachelor Thesis},   % subject of the document
    pdfcreator={Daniel Watzinger},   % creator of the document
    pdfproducer={Daniel Watzinger}, % producer of the document
    pdfkeywords={Bachelor Thesis, SPARQL, MPEG Query Format, MPQF, Passau}, % list of keywords
    pdfnewwindow=true,      % links in new window
    colorlinks=true,       % false: boxed links; true: colored links
    linkcolor=black,          % color of internal links
    citecolor=black,        % color of links to bibliography
    filecolor=magenta,      % color of file links
    urlcolor=DarkSkyBlue           % color of external links
}

% ------------------------------------------
% -------- fancyhdr
% ------------------------------------------
\fancyheadoffset{\marginparsep+\marginparwidth}
\fancyhf{}
\fancyhead[L]{\bfseries{\nouppercase{\leftmark}}}
\fancyfoot[C]{\bfseries{\thepage\ of \pageref{LastPage}}}
\renewcommand{\headrulewidth}{0.5pt}
\renewcommand{\footrulewidth}{0pt}

% ------------------------------------------
% -------- pdf specific
% ------------------------------------------

\pdfcompresslevel=3
%\pdfimageresolution=300
%\pdfinfo{
%/CreationDate (D:2010 09 01 00 00 00) % year(4) month(2) day(2) hour(2) minute(2) second(2)
%/ModDate      (D:2010 09 01 00 00 00) % modification date
%}

% ------------------------------------------
% -------- misc
% ------------------------------------------
\setcounter{secnumdepth}{3}
\setcounter{tocdepth}{3}
\clubpenalty = 10000
\widowpenalty = 10000
\displaywidowpenalty = 10000
\setlength\fboxsep{6pt}
\setlength\fboxrule{1pt}
\renewcommand*\oldstylenums[1]{{\fontfamily{fxlj}\selectfont #1}}

% ------------------------------------------
% -------- hyphenation rules
% ------------------------------------------
\hyphenation{}

\begin{document}
\pagestyle{fancy}

% -- Title

\title{\LARGE Transaktionssysteme -- Blatt 01\\ \large Team Amazonen -- Gruppe 1}
\author{\large Bastian Huber (51746) \and \large Sebastian Rainer (51746) \and \large Daniel Watzinger (51746)}
\date{\large\today}
\maketitle

\section{Acid}
	\begin{enumerate}[label={\alph*)}]
		\item	\begin{description}
						\item[Atomicity] Transaktionen sind atomar, d.h. sie werden entweder komplett oder garnicht ausgef�hrt.
						\item[Consistency] Eine Transaktion �berf�hrt einen konsistenten Systemzustand in einen neuen konsistenten Zustand.
						\item[Isolation] Parallel laufende Transaktionen beeinflussen sich nicht gegenseitig. Transaktionen arbeiten immer auf konsistenten Daten.
						\item[Durability] Wird eine Transaktion erfolgreich ausgef�hrt sind ihre Effekte dauerhaft, unabh�ngig von Fehlern im System oder Kommunikationsnetz.
					\end{description}
		\item \begin{description}
					\item[Atomicity] \textit{Umbuchung auf Konten:} Transaktion hebt Geld von einem Konto ab. Es tritt ein Fehler auf und die Transaktionsausf�hrung bricht ab. Das System befindet sich in einem inkonsistenten Zustand.
					\item[Consistency] Siehe \textbf{Atomicity}.
					\item[Isolation] \textit{Lost-Update Problem:} Zwei nebenl�ufige Transaktionen lesen den aktuellen Kontostand. Beide Transaktionen manipulieren selbigen lokal. Beim Zur�ckschreiben der �nderungen verliert man die Information �ber eines der Updates.
					\item[Durability] Eine Transaktion wird erfolgreich ausgef�hrt. Ein Fehler in der Persistenzschicht f�hrt dazu, dass die �nderungen der Transaktion verworfen werde. Das System befindet sich nun in einem inkonsistenten Zustand.
				\end{description}
	\end{enumerate}

\section{MySQL}
	\begin{enumerate}[label={\alph*)}]
		\item 
			\begin{multicols}{2} 
      \begin{itemize}[label={}]
					\item $\Square$ \textbf{Atomicity}
					\item $\Square$ \textbf{Consistency}
					\item $\CheckedBox$ \textbf{Isolation} (\texttt{LOCK TABLES})
					\item $\Square$ \textbf{Durability}
      \end{itemize} 
			\end{multicols}
		\item 
			\begin{multicols}{2} 
      \begin{itemize}[label={}]
					\item $\CheckedBox$ \textbf{Atomicity}
					\item $\CheckedBox$ \textbf{Consistency}
					\item $\CheckedBox$ \textbf{Isolation} (\texttt{LOCK TABLES})
					\item $\Square$ \textbf{Durability}
      \end{itemize} 
			\end{multicols}
		\item Performance
		\item Bei kleinen, privaten Projekten ohne kritische Daten und Transaktionsanforderungen. Prototyping. Gro�e Datenmengen die fehlertolerant sind.
		\item Business-to-Business-Services. Zum Beispiel Paypal, Banktransaktionen, etc.
	\end{enumerate}

\section{Autonome Systeme}
	\begin{enumerate}[label={\alph*)}]
		\item Jedes autonome System muss die \textbf{ACID}-Eigenschaften implementieren. Zus�tzlich muss eine �bergeordnete Instanz die \textbf{ACID}-Eigenschaft
			des Gesamtsystems garantieren. (Beispiel \textbf{Durability}: Das Gesamtsystem darf die Transaktion nur erfolgreich beenden, falls jedes Subsystem den (Speicher) Erfolg best�tigt hat.)
		 \item Siehe oben. �bergeordnete Instanz die f�r die \textbf{ACID}-Eigenschaft des Gesamtsystems sorgt. (Problem: \textbf{Single Point of Failure})
	\end{enumerate}

\end{document}
