%%%
% Excercise Template
%%%

%\listfiles 
\documentclass[12pt]{scrartcl} 
\usepackage[ngerman]{babel}
    
\usepackage[latin1]{inputenc}
\usepackage[T1]{fontenc}

\usepackage{amsmath}
\usepackage{graphicx,enumitem,wasysym,multicol,libertine,xcolor,microtype,lipsum,fixltx2e,
	fancyhdr,vmargin,calc,lastpage,hyperref,listings,lastpage,lipsum,preview,tikz,rotating,multirow,tabularx}
\usetikzlibrary{snakes,arrows,shapes}


% ------------------------------------------
% -------- xcolor - (Tango)
% ------------------------------------------

\definecolor{LightButter}{rgb}{0.98,0.91,0.31}
\definecolor{LightOrange}{rgb}{0.98,0.68,0.24}
\definecolor{LightChocolate}{rgb}{0.91,0.72,0.43}
\definecolor{LightChameleon}{rgb}{0.54,0.88,0.20}
\definecolor{LightSkyBlue}{rgb}{0.45,0.62,0.81}
\definecolor{LightPlum}{rgb}{0.68,0.50,0.66}
\definecolor{LightScarletRed}{rgb}{0.93,0.16,0.16}
\definecolor{Butter}{rgb}{0.93,0.86,0.25}
\definecolor{Orange}{rgb}{0.96,0.47,0.00}
\definecolor{Chocolate}{rgb}{0.75,0.49,0.07}
\definecolor{Chameleon}{rgb}{0.45,0.82,0.09}
\definecolor{SkyBlue}{rgb}{0.20,0.39,0.64}
\definecolor{Plum}{rgb}{0.46,0.31,0.48}
\definecolor{ScarletRed}{rgb}{0.80,0.00,0.00}
\definecolor{DarkButter}{rgb}{0.77,0.62,0.00}
\definecolor{DarkOrange}{rgb}{0.80,0.36,0.00}
\definecolor{DarkChocolate}{rgb}{0.56,0.35,0.01}
\definecolor{DarkChameleon}{rgb}{0.30,0.60,0.02}
\definecolor{DarkSkyBlue}{rgb}{0.12,0.29,0.53}
\definecolor{DarkPlum}{rgb}{0.36,0.21,0.40}
\definecolor{DarkScarletRed}{rgb}{0.64,0.00,0.00}
\definecolor{Aluminium1}{rgb}{0.93,0.93,0.92}
\definecolor{Aluminium2}{rgb}{0.82,0.84,0.81}
\definecolor{Aluminium3}{rgb}{0.73,0.74,0.71}
\definecolor{Aluminium4}{rgb}{0.53,0.54,0.52}
\definecolor{Aluminium5}{rgb}{0.33,0.34,0.32}
\definecolor{Aluminium6}{rgb}{0.18,0.20,0.21}
\definecolor{Brown}{cmyk}{0,0.81,1,0.60}
\definecolor{OliveGreen}{cmyk}{0.64,0,0.95,0.40}
\definecolor{CadetBlue}{cmyk}{0.62,0.57,0.23,0}

% ------------------------------------------
% -------- vmargin
% ------------------------------------------

%\setmarginsrb{hleftmargini}{htopmargini}{hrightmargini}{hbottommargini}%{hheadheighti}{hheadsepi}{hfootheighti}{hfootskipi}
\setpapersize{A4}
\setmarginsrb{3cm}{1cm}{3cm}{1cm}{8mm}{4mm}{3mm}{15mm}

% ------------------------------------------
% -------- listings
% ------------------------------------------
 
\lstdefinestyle{inline} {
		basicstyle=\normalsize
		}

\lstset{
		breakautoindent=true,
		breakindent=2em,
		breaklines=true,
		tabsize=4,
		frame=blrt,
		frameround=tttt,
		captionpos=b,
		basicstyle=\scriptsize\ttfamily,
		keywordstyle={\color{SkyBlue}},
		%commentstyle={\color{OliveGreen}},
		stringstyle={\color{OliveGreen}},
		showspaces=false,
		%numbers=right,
		%numberstyle=\scriptsize,
		%stepnumber=1, 
		%numbersep=5pt,
		%showtabs=false
		prebreak = \raisebox{0ex}[0ex][0ex]{\ensuremath{\hookleftarrow}},
		aboveskip={1.5\baselineskip},
		columns=fixed,
		upquote=true,
		extendedchars=true
		}


% ------------------------------------------
% -------- hyperref
% ------------------------------------------

\hypersetup{
	%breaklinks=true,
	pdfborder={0 0 0},
	bookmarks=true,         % show bookmarks bar?
	unicode=false,          % non-Latin characters in Acrobat�s bookmarks
	pdftoolbar=true,        % show Acrobat�s toolbar?
	pdfmenubar=true,        % show Acrobat�s menu?
	pdffitwindow=true,     % window fit to page when opened
	pdfstartview={FitH},    % fits the width of the page to the window
    pdfnewwindow=true,      % links in new window
    colorlinks=true,       % false: boxed links; true: colored links
    linkcolor=black,          % color of internal links
    citecolor=black,        % color of links to bibliography
    filecolor=magenta,      % color of file links
    urlcolor=DarkSkyBlue           % color of external links
}

% ------------------------------------------
% -------- fancyhdr
% ------------------------------------------
\fancyheadoffset{\marginparsep+\marginparwidth}
\fancyhf{}
\fancyhead[L]{\bfseries{\nouppercase{\leftmark}}}
\fancyfoot[C]{\bfseries{\thepage\ of \pageref{LastPage}}}
\renewcommand{\headrulewidth}{0.5pt}
\renewcommand{\footrulewidth}{0pt}

% ------------------------------------------
% -------- pdf specific
% ------------------------------------------

\pdfcompresslevel=3
%\pdfimageresolution=300
%\pdfinfo{
%/CreationDate (D:2010 09 01 00 00 00) % year(4) month(2) day(2) hour(2) minute(2) second(2)
%/ModDate      (D:2010 09 01 00 00 00) % modification date
%}

% ------------------------------------------
% -------- misc
% ------------------------------------------
\setcounter{secnumdepth}{3}
\setcounter{tocdepth}{3}
\clubpenalty = 10000
\widowpenalty = 10000
\displaywidowpenalty = 10000
\setlength\fboxsep{6pt}
\setlength\fboxrule{1pt}
\renewcommand*\oldstylenums[1]{{\fontfamily{fxlj}\selectfont #1}}

% ------------------------------------------
% -------- hyphenation rules
% ------------------------------------------
\hyphenation{}

\begin{document}
\pagestyle{fancy}

% -- Title

\title{\LARGE Transaktionssysteme -- Blatt 05\\ \large Team Amazonen -- Gruppe 1}
\author{\large Bastian Huber (51432) \and \large Sebastian Rainer (50882) \and \large Daniel Watzinger (51746)}
\date{\large\today}
\maketitle

\section{BFO}
\begin{enumerate}[label={\alph*)}]
	\item \begin{itemize}
					\item \textbf{Initial:} $r-TS(x) = w-TS(x) = r-TS(y) = w-TS(y) = 0$
					\item \begin{tabularx}{\textwidth}{ | X | X |}
										\hline
										\textbf{Output} & \textbf{Status�nderungen} \\
										\hline \hline
										$w_1(x)$ & $w-TS(x) = 1$ \\
										\hline
										$r_2(y)$ & $r-TS(y) = 2$ \\
										\hline
										$r_1(x)$ & $r-TS(x) = 1$ \\
										\hline
										$r_2(x)$ & $r-TS(x) = 2$ \\
										\hline
										$w_2(y)$ & $w-TS(y) = 2$ \\
										\hline
									\end{tabularx}
					\item $$S_{BFO} = w_1(x) r_2(y) r_1(x) c_1 r_2(x) w_2(y) c_2$$
				\end{itemize}
	\item \begin{itemize}
					\item \textbf{Initial:} $r-TS(x) = w-TS(x) = 0$
					\item \begin{tabularx}{\textwidth}{ | X | X |}
										\hline
										\textbf{Output} & \textbf{Status�nderungen} \\
										\hline \hline
										$r_1(x)$ & $r-TS(x) = 1$ \\
										\hline
										$r_2(x)$ & $r-TS(x) = 2$ \\
										\hline
										$w_3(x)$ & $w-TS(x) = 3$ \\
										\hline
										$w_4(x)$ & $w-TS(x) = 4$ \\
										\hline
										$w_1(x)$ & $TS(T_1) < w-TS(x) \newline TS(T_1) < r-TS(x) \newline \Rightarrow abort(T_1)$ \\
										\hline
										$w_2(x)$ & $TS(T_2) < w-TS(x) \newline \Rightarrow abort(T_2)$ \\
										\hline
									\end{tabularx}
					\item $$S_{BFO} = r_1(x) r_2(x) w_3(x) w_4(x) c_1 w_1(x) a_1 w_2(x) a_2 c_3$$
				\end{itemize}
\end{enumerate}
	
\section{SGT}
\begin{enumerate}[label={\alph*)}]
	\item \begin{itemize}
					\item \textbf{Initial:} $emptyGraph()$
					\item \begin{tabularx}{\textwidth}{ | X | X |}
										\hline
										\textbf{Output} & \textbf{Status�nderungen} \\
										\hline \hline
										$w_1(x)$ &  \\
										\hline
										$r_2(y)$ &  \\
										\hline
										$r_1(x)$ & \\
										\hline
										$r_2(x)$ & \\
										\hline
										$w_2(y)$ & \\
										\hline
									\end{tabularx}
					\item $$S_{SGT} = w_1(x) r_2(y) r_1(x) c_1 r_2(x) w_2(y) c_2$$
				\end{itemize}
	\item \begin{itemize}
					\item \textbf{Initial:} $r-TS(x) = w-TS(x) = 0$
					\item \begin{tabularx}{\textwidth}{ | X | X |}
										\hline
										\textbf{Output} & \textbf{Status�nderungen} \\
										\hline \hline
										$r_1(x)$ & $r-TS(x) = 1$ \\
										\hline
										$r_2(x)$ & $r-TS(x) = 2$ \\
										\hline
										$w_3(x)$ & $w-TS(x) = 3$ \\
										\hline
										$w_4(x)$ & $w-TS(x) = 4$ \\
										\hline
										$w_1(x)$ & $TS(T_1) < w-TS(x) \newline TS(T_1) < r-TS(x) \newline \Rightarrow abort(T_1)$ \\
										\hline
										$w_2(x)$ & $TS(T_2) < w-TS(x) \newline \Rightarrow abort(T_2)$ \\
										\hline
									\end{tabularx}
					\item $$S_{BFO} = r_1(x) r_2(x) w_3(x) w_4(x) c_1 w_1(x) a_1 w_2(x) a_2 c_3$$
				\end{itemize}
\end{enumerate}
	
\section{SS2PL}
Gegeben sei folgender Schedule S:
	$$
		S :=r_1(y) r_2(y) r_3(y) r_4(z) w_2(y) r_3(x) w_1(x) r_4(x) w_2(x) w_3(z) c_1 c_2 c_3 c_4
	$$
	\begin{align*}
			S_{out}^{SS2PL} = r_1l(y) r_1(y) r_2l(y) r_2(y) r_3l(y) r_3(y) r_4l(z) r_4(z) r_3l(x) r_3(x) r_4l(x) r_4(x) \\
								c_4 r_4u(z) r_4u(x) w_3l(z) w_3(z) c_3 r_3u(y) r_3u(x) w_3u(z) w_1l(x) w_1(x) c_1 r_1u(y) \\
								w_1u(x) w_2l(y) w_2(y) w_2l(x) w_2(x) c_3 r_2u(y) w_2u(y) w_2u(x)
	\end{align*}
	
\section{S2PL}
Gegeben sei folgender Schedule S:
	$$
		S :=r_1(x) r_3(z) w_1(y) r_2(y) w_1(x) r_3(x) w_3(z) w_2(z) w_3(x) c_1 c_2 c_3
	$$
	\begin{align*}
			S_{out}^{S2PL} = r_1l(x) r_1(x) r_3l(z) r_3(z) w_1l(y) w_1(y) w_1l(x) w_1(x) r_1u(x) c_1 w_1u(y) w_1u(x) r_2l(y) \\
											r_2(y) r_3l(x) r_3(x) w_3l(z) w_3(z) w_3l(x) r_3u(x) w_3(x) c_3 w_3u(z) w_3u(x) w_2l(z) r_2u(y) \\
											w_2(z) c_2 w_2u(z)
	\end{align*}

\section{2PL, C2PL, Deadlocks}
Gegeben sei der Schedule $S := w_1(x) r_2(y) w_2(x) w_1(y) c_1 c_2$
	\begin{enumerate}[label={\alph*)}]
		\item $ S_{out}^{2PL} = w_1l(x) w_1(x) r_2l(y) r_2(y)$ \lightning Deadlock
		
		\item $ S_{out}^{C2PL} = w_1l(x) w_1l(y) w_1(x) w_1u(x) w_1(y) w_1u(y) r_2l(y) w_2l(x) r_2(y) r_2u(y) w_2(x) w_2u(x) c_1 c_2 $
		
		\item Ein \textit{konservativer 2PL-Scheduler} ist deadlockfrei, l�sst aber weniger Parallelit�t bei der Auf�hrung von Operationen zu was zu einem Performanzverlust f�hren kann.
		
		\item Sei $S$ ein Schedule und $o_i$ die erste Datenoperation von $S$. Angenommen ein sp�terer Schritt $p_i(x)$ f�hrt zu einem Deadlock.
			Die Transaktion $T_j$ besitzt also einen nicht kompatiblen Lock $o_jl(x)$ auf ein Datenobjekt $x$.
			Dann kann aber $o_i$ nicht die erste Datenoperation des Schedules sein, da $T_i$ keinen Lock $o_il(x)$ besitzt (\textit{C2PL-Scheduler}).
			Widerspruch zur Annahme \lightning
			
			Variante 2: Sei $S$ ein Schedule und sei $trans(S)=\left\{T_1,\dots,T_n\right\}$. Sei 
			$$\phi := \{1,\dots, n\} \rightarrow \{1,\dots, n\}$$
			eine Permutation.
			Angenommen es besteht eine zyklische Deadlockbeziehung $\rightarrow_D$ zwischen den Transaktionen
			$$T_{\phi(1)} \rightarrow_D T_{\phi(2)} \rightarrow_D \dots \rightarrow_D T_{\phi(n)} \rightarrow_D T_{\phi(1)}$$
			Sei 
			$$var(T_i) = \left\{x | x \text{ wird von } op(x) \in op(T_i) \text{ gelesen/geschrieben}\right\}$$
			F�r obige Situation gilt also 
			$$X = var(T_\phi(1)) \cap var(T_\phi(n)) \neq \emptyset$$
			O.b.d.A halte $T_{\phi(1)}$ einen Lock $o_{\phi(1)}l(x \in X)$ und $T_{\phi(n)}$ einen Lock
			$o_{\phi(n)}l(y \in X)$ mit $x \neq y$. F�r einen \textit{C2PL-Scheduler} folgt also f�r die erste Datenoperation der Transaktionen
			$$first(T_{\phi(n)}) <_S first(T_{\phi(1)})$$ 
			und 
			$$first(T_{\phi(1)}) <_S first(T_{\phi(n)})$$ 
			Widerspruch zur Annahme \lightning
			
			Variante 3:
			Angenommen wir haben einen Schedule erzeugt von einem konservativen 2PL-Scheduler mit Transaktionen $T_1$ und $T_2$ und einem Deadlock. Dann gibt es zwei Datenobjekte $x$ und $y$ wobei $T_1$ den Lock auf $x$ hat und $T_2$
			auf $y$. O.b.d.A wartet $T_1$ auf $y$ und $T_2$ auf $x$. Sei die erste Datenoperation von $T_1$ vor allen Datenoperationen vor $T_2$.
			Dann f�hrt $T_1$ zuerst alle Locks aus die er braucht. Das hei�t $w_1l(x) und w_1l(y)$. Somit kann $T_2$ seinen Lock auf $y$ gar nicht erhalten bevor $T_1$ fertig ist. Widerspruch zur Annahme. Somit ist jeder C2PL Scheduler deadlockfrei.
	\end{enumerate}
	
\end{document}
