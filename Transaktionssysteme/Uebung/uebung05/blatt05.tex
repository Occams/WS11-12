%%%
% Excercise Template
%%%

%\listfiles 
\documentclass[12pt]{scrartcl} 
\usepackage[ngerman]{babel}
    
\usepackage[latin1]{inputenc}
\usepackage[T1]{fontenc}

\usepackage{amsmath}
\usepackage{graphicx,enumitem,wasysym,multicol,libertine,xcolor,microtype,lipsum,fixltx2e,
	fancyhdr,vmargin,calc,lastpage,hyperref,listings,lastpage,lipsum,preview,tikz,rotating,multirow,tabularx}
\usetikzlibrary{snakes,arrows,shapes}


% ------------------------------------------
% -------- xcolor - (Tango)
% ------------------------------------------

\definecolor{LightButter}{rgb}{0.98,0.91,0.31}
\definecolor{LightOrange}{rgb}{0.98,0.68,0.24}
\definecolor{LightChocolate}{rgb}{0.91,0.72,0.43}
\definecolor{LightChameleon}{rgb}{0.54,0.88,0.20}
\definecolor{LightSkyBlue}{rgb}{0.45,0.62,0.81}
\definecolor{LightPlum}{rgb}{0.68,0.50,0.66}
\definecolor{LightScarletRed}{rgb}{0.93,0.16,0.16}
\definecolor{Butter}{rgb}{0.93,0.86,0.25}
\definecolor{Orange}{rgb}{0.96,0.47,0.00}
\definecolor{Chocolate}{rgb}{0.75,0.49,0.07}
\definecolor{Chameleon}{rgb}{0.45,0.82,0.09}
\definecolor{SkyBlue}{rgb}{0.20,0.39,0.64}
\definecolor{Plum}{rgb}{0.46,0.31,0.48}
\definecolor{ScarletRed}{rgb}{0.80,0.00,0.00}
\definecolor{DarkButter}{rgb}{0.77,0.62,0.00}
\definecolor{DarkOrange}{rgb}{0.80,0.36,0.00}
\definecolor{DarkChocolate}{rgb}{0.56,0.35,0.01}
\definecolor{DarkChameleon}{rgb}{0.30,0.60,0.02}
\definecolor{DarkSkyBlue}{rgb}{0.12,0.29,0.53}
\definecolor{DarkPlum}{rgb}{0.36,0.21,0.40}
\definecolor{DarkScarletRed}{rgb}{0.64,0.00,0.00}
\definecolor{Aluminium1}{rgb}{0.93,0.93,0.92}
\definecolor{Aluminium2}{rgb}{0.82,0.84,0.81}
\definecolor{Aluminium3}{rgb}{0.73,0.74,0.71}
\definecolor{Aluminium4}{rgb}{0.53,0.54,0.52}
\definecolor{Aluminium5}{rgb}{0.33,0.34,0.32}
\definecolor{Aluminium6}{rgb}{0.18,0.20,0.21}
\definecolor{Brown}{cmyk}{0,0.81,1,0.60}
\definecolor{OliveGreen}{cmyk}{0.64,0,0.95,0.40}
\definecolor{CadetBlue}{cmyk}{0.62,0.57,0.23,0}

% ------------------------------------------
% -------- vmargin
% ------------------------------------------

%\setmarginsrb{hleftmargini}{htopmargini}{hrightmargini}{hbottommargini}%{hheadheighti}{hheadsepi}{hfootheighti}{hfootskipi}
\setpapersize{A4}
\setmarginsrb{3cm}{1cm}{3cm}{1cm}{8mm}{4mm}{3mm}{15mm}

% ------------------------------------------
% -------- listings
% ------------------------------------------
 
\lstdefinestyle{inline} {
		basicstyle=\normalsize
		}

\lstset{
		breakautoindent=true,
		breakindent=2em,
		breaklines=true,
		tabsize=4,
		frame=blrt,
		frameround=tttt,
		captionpos=b,
		basicstyle=\scriptsize\ttfamily,
		keywordstyle={\color{SkyBlue}},
		%commentstyle={\color{OliveGreen}},
		stringstyle={\color{OliveGreen}},
		showspaces=false,
		%numbers=right,
		%numberstyle=\scriptsize,
		%stepnumber=1, 
		%numbersep=5pt,
		%showtabs=false
		prebreak = \raisebox{0ex}[0ex][0ex]{\ensuremath{\hookleftarrow}},
		aboveskip={1.5\baselineskip},
		columns=fixed,
		upquote=true,
		extendedchars=true
		}


% ------------------------------------------
% -------- hyperref
% ------------------------------------------

\hypersetup{
	%breaklinks=true,
	pdfborder={0 0 0},
	bookmarks=true,         % show bookmarks bar?
	unicode=false,          % non-Latin characters in Acrobat�s bookmarks
	pdftoolbar=true,        % show Acrobat�s toolbar?
	pdfmenubar=true,        % show Acrobat�s menu?
	pdffitwindow=true,     % window fit to page when opened
	pdfstartview={FitH},    % fits the width of the page to the window
    pdfnewwindow=true,      % links in new window
    colorlinks=true,       % false: boxed links; true: colored links
    linkcolor=black,          % color of internal links
    citecolor=black,        % color of links to bibliography
    filecolor=magenta,      % color of file links
    urlcolor=DarkSkyBlue           % color of external links
}

% ------------------------------------------
% -------- fancyhdr
% ------------------------------------------
\fancyheadoffset{\marginparsep+\marginparwidth}
\fancyhf{}
\fancyhead[L]{\bfseries{\nouppercase{\leftmark}}}
\fancyfoot[C]{\bfseries{\thepage\ of \pageref{LastPage}}}
\renewcommand{\headrulewidth}{0.5pt}
\renewcommand{\footrulewidth}{0pt}

% ------------------------------------------
% -------- pdf specific
% ------------------------------------------

\pdfcompresslevel=3
%\pdfimageresolution=300
%\pdfinfo{
%/CreationDate (D:2010 09 01 00 00 00) % year(4) month(2) day(2) hour(2) minute(2) second(2)
%/ModDate      (D:2010 09 01 00 00 00) % modification date
%}

% ------------------------------------------
% -------- misc
% ------------------------------------------
\setcounter{secnumdepth}{3}
\setcounter{tocdepth}{3}
\clubpenalty = 10000
\widowpenalty = 10000
\displaywidowpenalty = 10000
\setlength\fboxsep{6pt}
\setlength\fboxrule{1pt}
\renewcommand*\oldstylenums[1]{{\fontfamily{fxlj}\selectfont #1}}

% ------------------------------------------
% -------- hyphenation rules
% ------------------------------------------
\hyphenation{}

\begin{document}
\pagestyle{fancy}

% -- Title

\title{\LARGE Transaktionssysteme -- Blatt 05\\ \large Team Amazonen -- Gruppe 1}
\author{\large Bastian Huber (51432) \and \large Sebastian Rainer (50882) \and \large Daniel Watzinger (51746)}
\date{\large\today}
\maketitle

\section{BFO}
\begin{enumerate}[label={\alph*)}]
	\item \begin{itemize}
					\item \textbf{Initial:} $r-TS(x) = w-TS(x) = r-TS(y) = w-TS(y) = 0$
					\item \begin{tabularx}{\textwidth}{ | X | X |}
										\hline
										\textbf{Output} & \textbf{Status�nderungen} \\
										\hline \hline
										$w_1(x)$ & $w-TS(x) = 1$ \\
										\hline
										$r_2(y)$ & $r-TS(y) = 2$ \\
										\hline
										$r_1(x)$ & $r-TS(x) = 1$ \\
										\hline
										$r_2(x)$ & $r-TS(x) = 2$ \\
										\hline
										$w_2(y)$ & $w-TS(y) = 2$ \\
										\hline
									\end{tabularx}
					\item $$S_{BFO} = w_1(x) r_2(y) r_1(x) c_1 r_2(x) w_2(y) c_2$$
				\end{itemize}
	\item \begin{itemize}
					\item \textbf{Initial:} $r-TS(x) = w-TS(x) = 0$
					\item \begin{tabularx}{\textwidth}{ | X | X |}
										\hline
										\textbf{Output} & \textbf{Status�nderungen} \\
										\hline \hline
										$r_1(x)$ & $r-TS(x) = 1$ \\
										\hline
										$r_2(x)$ & $r-TS(x) = 2$ \\
										\hline
										$w_3(x)$ & $w-TS(x) = 3$ \\
										\hline
										$w_4(x)$ & $w-TS(x) = 4$ \\
										\hline
										$w_1(x)$ & $TS(T_1) < w-TS(x) \newline TS(T_1) < r-TS(x) \newline \Rightarrow abort(T_1)$ \\
										\hline
										$w_2(x)$ & $TS(T_2) < w-TS(x) \newline \Rightarrow abort(T_2)$ \\
										\hline
									\end{tabularx}
					\item $$S_{BFO} = r_1(x) r_2(x) w_3(x) w_4(x) c_1 w_1(x) a_1 w_2(x) a_2 c_3$$
				\end{itemize}
\end{enumerate}
	
\section{SGT}
\begin{enumerate}[label={\alph*)}]
	\item \begin{itemize}
					\item \textbf{Initial:} $emptyGraph()$
					\item \textbf{Output}: $w_1(x)$
					\item \textbf{Status�nderungen}: 
\begin{tikzpicture}[>=latex,line join=bevel,]
  \pgfsetlinewidth{1bp}
%%
\pgfsetcolor{black}
  % Node: T_1
\begin{scope}
  \definecolor{strokecol}{rgb}{0.0,0.0,0.0};
  \pgfsetstrokecolor{strokecol}
  \draw (27bp,27bp) ellipse (27bp and 27bp);
  \draw (27bp,27bp) node {$T_1$};
\end{scope}
%
\end{tikzpicture}


					
					\item \textbf{Output}: $r_2(y)$
					\item \textbf{Status�nderungen}: 
\begin{tikzpicture}[>=latex,line join=bevel,]
  \pgfsetlinewidth{1bp}
%%
\pgfsetcolor{black}
  % Node: T_2
\begin{scope}
  \definecolor{strokecol}{rgb}{0.0,0.0,0.0};
  \pgfsetstrokecolor{strokecol}
  \draw (27bp,88bp) ellipse (27bp and 27bp);
  \draw (27bp,88bp) node {$T_2$};
\end{scope}
  % Node: T_1
\begin{scope}
  \definecolor{strokecol}{rgb}{0.0,0.0,0.0};
  \pgfsetstrokecolor{strokecol}
  \draw (27bp,27bp) ellipse (27bp and 27bp);
  \draw (27bp,27bp) node {$T_1$};
\end{scope}
%
\end{tikzpicture}


					
					\item \textbf{Output}: $r_1(x) c_1$
					\item \textbf{Status�nderungen}: 
\begin{tikzpicture}[>=latex,line join=bevel,]
  \pgfsetlinewidth{1bp}
%%
\pgfsetcolor{black}
  % Node: T_2
\begin{scope}
  \definecolor{strokecol}{rgb}{0.0,0.0,0.0};
  \pgfsetstrokecolor{strokecol}
  \draw (27bp,88bp) ellipse (27bp and 27bp);
  \draw (27bp,88bp) node {$T_2$};
\end{scope}
  % Node: T_1
\begin{scope}
  \definecolor{strokecol}{rgb}{0.0,0.0,0.0};
  \pgfsetstrokecolor{strokecol}
  \draw (27bp,27bp) ellipse (27bp and 27bp);
  \draw (27bp,27bp) node {$T_1$};
\end{scope}
%
\end{tikzpicture}


								$\xrightarrow{Garbage Collection}$ 
\begin{tikzpicture}[>=latex,line join=bevel,]
  \pgfsetlinewidth{1bp}
%%
\pgfsetcolor{black}
  % Node: T_2
\begin{scope}
  \definecolor{strokecol}{rgb}{0.0,0.0,0.0};
  \pgfsetstrokecolor{strokecol}
  \draw (27bp,27bp) ellipse (27bp and 27bp);
  \draw (27bp,27bp) node {$T_2$};
\end{scope}
%
\end{tikzpicture}


					
					\item \textbf{Output}: $r_2(x)$
					\item \textbf{Status�nderungen}: 
\begin{tikzpicture}[>=latex,line join=bevel,]
  \pgfsetlinewidth{1bp}
%%
\pgfsetcolor{black}
  % Node: T_2
\begin{scope}
  \definecolor{strokecol}{rgb}{0.0,0.0,0.0};
  \pgfsetstrokecolor{strokecol}
  \draw (27bp,27bp) ellipse (27bp and 27bp);
  \draw (27bp,27bp) node {$T_2$};
\end{scope}
%
\end{tikzpicture}


					
					\item \textbf{Output}: $w_2(y) c_2$
					\item \textbf{Status�nderungen}: 
\begin{tikzpicture}[>=latex,line join=bevel,]
  \pgfsetlinewidth{1bp}
%%
\pgfsetcolor{black}
  % Node: T_2
\begin{scope}
  \definecolor{strokecol}{rgb}{0.0,0.0,0.0};
  \pgfsetstrokecolor{strokecol}
  \draw (27bp,27bp) ellipse (27bp and 27bp);
  \draw (27bp,27bp) node {$T_2$};
\end{scope}
%
\end{tikzpicture}


								$\xrightarrow{Garbage Collection} emptyGraph()$
								
					\item $$S_{SGT} = w_1(x) r_2(y) r_1(x) c_1 r_2(x) w_2(y) c_2$$
				\end{itemize}
	\item \begin{itemize}
					\item \textbf{Initial:} $emptyGraph()$
					\item \textbf{Output}: $r_1(x)$
					\item \textbf{Status�nderungen}: 
\begin{tikzpicture}[>=latex,line join=bevel,]
  \pgfsetlinewidth{1bp}
%%
\pgfsetcolor{black}
  % Node: T_1
\begin{scope}
  \definecolor{strokecol}{rgb}{0.0,0.0,0.0};
  \pgfsetstrokecolor{strokecol}
  \draw (27bp,27bp) ellipse (27bp and 27bp);
  \draw (27bp,27bp) node {$T_1$};
\end{scope}
%
\end{tikzpicture}


					
					\item \textbf{Output}: $r_2(x)$
					\item \textbf{Status�nderungen}: 
\begin{tikzpicture}[>=latex,line join=bevel,]
  \pgfsetlinewidth{1bp}
%%
\pgfsetcolor{black}
  % Node: T_2
\begin{scope}
  \definecolor{strokecol}{rgb}{0.0,0.0,0.0};
  \pgfsetstrokecolor{strokecol}
  \draw (27bp,88bp) ellipse (27bp and 27bp);
  \draw (27bp,88bp) node {$T_2$};
\end{scope}
  % Node: T_1
\begin{scope}
  \definecolor{strokecol}{rgb}{0.0,0.0,0.0};
  \pgfsetstrokecolor{strokecol}
  \draw (27bp,27bp) ellipse (27bp and 27bp);
  \draw (27bp,27bp) node {$T_1$};
\end{scope}
%
\end{tikzpicture}


					
					\item \textbf{Output}: $w_3(x)$
					\item \textbf{Status�nderungen}: 
\begin{tikzpicture}[>=latex,line join=bevel,]
  \pgfsetlinewidth{1bp}
%%
\pgfsetcolor{black}
  % Edge: T_2 -> T_3
  \draw [->] (52.743bp,78.531bp) .. controls (59.73bp,75.856bp) and (67.459bp,72.898bp)  .. (84.409bp,66.411bp);
  % Edge: T_1 -> T_3
  \draw [->] (52.743bp,36.164bp) .. controls (59.73bp,38.752bp) and (67.459bp,41.614bp)  .. (84.409bp,47.892bp);
  % Node: T_2
\begin{scope}
  \definecolor{strokecol}{rgb}{0.0,0.0,0.0};
  \pgfsetstrokecolor{strokecol}
  \draw (27bp,88bp) ellipse (27bp and 27bp);
  \draw (27bp,88bp) node {$T_2$};
\end{scope}
  % Node: T_3
\begin{scope}
  \definecolor{strokecol}{rgb}{0.0,0.0,0.0};
  \pgfsetstrokecolor{strokecol}
  \draw (110bp,57bp) ellipse (27bp and 27bp);
  \draw (110bp,57bp) node {$T_3$};
\end{scope}
  % Node: T_1
\begin{scope}
  \definecolor{strokecol}{rgb}{0.0,0.0,0.0};
  \pgfsetstrokecolor{strokecol}
  \draw (27bp,27bp) ellipse (27bp and 27bp);
  \draw (27bp,27bp) node {$T_1$};
\end{scope}
%
\end{tikzpicture}


					
					\item \textbf{Output}: $w_4(x)$
					\item \textbf{Status�nderungen}: 
\begin{tikzpicture}[>=latex,line join=bevel,]
  \pgfsetlinewidth{1bp}
%%
\pgfsetcolor{black}
  % Edge: T_2 -> T_4
  \draw [->] (136.99bp,37.099bp) .. controls (164.38bp,35.094bp) and (207.73bp,31.922bp)  .. (248.86bp,28.912bp);
  % Edge: T_3 -> T_4
  \draw [->] (217.86bp,53.804bp) .. controls (225.44bp,50.248bp) and (233.97bp,46.25bp)  .. (251.31bp,38.116bp);
  % Edge: T_1 -> T_2
  \draw [->] (54.075bp,39bp) .. controls (60bp,39bp) and (66.4bp,39bp)  .. (82.974bp,39bp);
  % Edge: T_1 -> T_3
  \draw [->] (49.687bp,54.1bp) .. controls (59.407bp,59.981bp) and (71.301bp,66.017bp)  .. (83bp,69bp) .. controls (106.97bp,75.112bp) and (134.86bp,73.945bp)  .. (166.46bp,69.965bp);
  % Edge: T_2 -> T_3
  \draw [->] (136.18bp,47.084bp) .. controls (142.95bp,49.257bp) and (150.39bp,51.644bp)  .. (167.11bp,57.01bp);
  % Edge: T_1 -> T_4
  \draw [->] (49.687bp,23.9bp) .. controls (59.407bp,18.019bp) and (71.301bp,11.983bp)  .. (83bp,9bp) .. controls (136.75bp,-4.7068bp) and (201.36bp,6.8701bp)  .. (250.03bp,19.33bp);
  % Node: T_2
\begin{scope}
  \definecolor{strokecol}{rgb}{0.0,0.0,0.0};
  \pgfsetstrokecolor{strokecol}
  \draw (110bp,39bp) ellipse (27bp and 27bp);
  \draw (110bp,39bp) node {$T_2$};
\end{scope}
  % Node: T_3
\begin{scope}
  \definecolor{strokecol}{rgb}{0.0,0.0,0.0};
  \pgfsetstrokecolor{strokecol}
  \draw (193bp,65bp) ellipse (27bp and 27bp);
  \draw (193bp,65bp) node {$T_3$};
\end{scope}
  % Node: T_1
\begin{scope}
  \definecolor{strokecol}{rgb}{0.0,0.0,0.0};
  \pgfsetstrokecolor{strokecol}
  \draw (27bp,39bp) ellipse (27bp and 27bp);
  \draw (27bp,39bp) node {$T_1$};
\end{scope}
  % Node: T_4
\begin{scope}
  \definecolor{strokecol}{rgb}{0.0,0.0,0.0};
  \pgfsetstrokecolor{strokecol}
  \draw (276bp,27bp) ellipse (27bp and 27bp);
  \draw (276bp,27bp) node {$T_4$};
\end{scope}
%
\end{tikzpicture}


					
					\item \textbf{Output}: $w_1(x) a_1$
					\item \textbf{Status�nderungen}: 
\begin{tikzpicture}[>=latex,line join=bevel,]
  \pgfsetlinewidth{1bp}
%%
\pgfsetcolor{black}
  % Edge: T_4 -> T_1
  \draw [->] (249.03bp,25.414bp) .. controls (211.65bp,21.62bp) and (140.82bp,15.257bp)  .. (83bp,30bp) .. controls (75.392bp,31.94bp) and (67.701bp,35.171bp)  .. (51.282bp,43.67bp);
  % Edge: T_2 -> T_4
  \draw [->] (136.62bp,51.469bp) .. controls (164.22bp,46.588bp) and (208.34bp,38.787bp)  .. (249.28bp,31.547bp);
  % Edge: T_3 -> T_4
  \draw [->] (214.86bp,73.776bp) .. controls (224.4bp,66.354bp) and (235.86bp,57.439bp)  .. (254.21bp,43.169bp);
  % Edge: T_1 -> T_2
  \draw [->] (54.075bp,53.882bp) .. controls (60bp,53.769bp) and (66.4bp,53.729bp)  .. (82.974bp,53.883bp);
  % Edge: T_2 -> T_1
  \draw [->] (82.974bp,58.823bp) .. controls (77.051bp,58.974bp) and (70.653bp,59.028bp)  .. (54.075bp,58.824bp);
  % Edge: T_1 -> T_3
  \draw [->] (51.069bp,68.897bp) .. controls (60.542bp,74.279bp) and (71.882bp,79.947bp)  .. (83bp,83bp) .. controls (106.7bp,89.507bp) and (134.38bp,90.445bp)  .. (165.95bp,89.97bp);
  % Edge: T_2 -> T_3
  \draw [->] (135.3bp,66.201bp) .. controls (142.59bp,69.26bp) and (150.72bp,72.672bp)  .. (167.71bp,79.806bp);
  % Edge: T_3 -> T_1
  \draw [->] (166.39bp,95.493bp) .. controls (143.98bp,97.603bp) and (110.76bp,97.622bp)  .. (83bp,90bp) .. controls (74.008bp,87.531bp) and (64.87bp,83.35bp)  .. (47.927bp,73.4bp);
  % Edge: T_1 -> T_4
  \draw [->] (48.13bp,39.139bp) .. controls (58.121bp,32.681bp) and (70.669bp,26.144bp)  .. (83bp,23bp) .. controls (136.14bp,9.4486bp) and (200.28bp,13.729bp)  .. (249.65bp,20.911bp);
  % Node: T_2
\begin{scope}
  \definecolor{strokecol}{rgb}{0.0,0.0,0.0};
  \pgfsetstrokecolor{strokecol}
  \draw (110bp,56bp) ellipse (27bp and 27bp);
  \draw (110bp,56bp) node {$T_2$};
\end{scope}
  % Node: T_3
\begin{scope}
  \definecolor{strokecol}{rgb}{0.0,0.0,0.0};
  \pgfsetstrokecolor{strokecol}
  \draw (193bp,90bp) ellipse (27bp and 27bp);
  \draw (193bp,90bp) node {$T_3$};
\end{scope}
  % Node: T_1
\begin{scope}
  \definecolor{strokecol}{rgb}{0.0,0.0,0.0};
  \pgfsetstrokecolor{strokecol}
  \draw (27bp,56bp) ellipse (27bp and 27bp);
  \draw (27bp,56bp) node {$T_1$};
\end{scope}
  % Node: T_4
\begin{scope}
  \definecolor{strokecol}{rgb}{0.0,0.0,0.0};
  \pgfsetstrokecolor{strokecol}
  \draw (276bp,27bp) ellipse (27bp and 27bp);
  \draw (276bp,27bp) node {$T_4$};
\end{scope}
%
\end{tikzpicture}

 Zyklus \lightning 
\begin{tikzpicture}[>=latex,line join=bevel,]
  \pgfsetlinewidth{1bp}
%%
\pgfsetcolor{black}
  % Edge: T_2 -> T_3
  \draw [->] (53.629bp,33.012bp) .. controls (59.895bp,34.481bp) and (66.712bp,36.081bp)  .. (83.538bp,40.027bp);
  % Edge: T_2 -> T_4
  \draw [->] (53.308bp,20.851bp) .. controls (62.587bp,18.886bp) and (73.213bp,16.969bp)  .. (83bp,16bp) .. controls (106.88bp,13.635bp) and (113.12bp,13.635bp)  .. (137bp,16bp) .. controls (143.42bp,16.636bp) and (150.21bp,17.68bp)  .. (166.69bp,20.851bp);
  % Edge: T_3 -> T_4
  \draw [->] (136.63bp,39.988bp) .. controls (142.89bp,38.519bp) and (149.71bp,36.919bp)  .. (166.54bp,32.973bp);
  % Node: T_2
\begin{scope}
  \definecolor{strokecol}{rgb}{0.0,0.0,0.0};
  \pgfsetstrokecolor{strokecol}
  \draw (27bp,27bp) ellipse (27bp and 27bp);
  \draw (27bp,27bp) node {$T_2$};
\end{scope}
  % Node: T_3
\begin{scope}
  \definecolor{strokecol}{rgb}{0.0,0.0,0.0};
  \pgfsetstrokecolor{strokecol}
  \draw (110bp,46bp) ellipse (27bp and 27bp);
  \draw (110bp,46bp) node {$T_3$};
\end{scope}
  % Node: T_4
\begin{scope}
  \definecolor{strokecol}{rgb}{0.0,0.0,0.0};
  \pgfsetstrokecolor{strokecol}
  \draw (193bp,27bp) ellipse (27bp and 27bp);
  \draw (193bp,27bp) node {$T_4$};
\end{scope}
%
\end{tikzpicture}


								
					\item \textbf{Output}: $w_2(x) a_2$
					\item \textbf{Status�nderungen}: 
\begin{tikzpicture}[>=latex,line join=bevel,]
  \pgfsetlinewidth{1bp}
%%
\pgfsetcolor{black}
  % Edge: T_2 -> T_3
  \draw [->] (53.629bp,51.535bp) .. controls (60.168bp,53.101bp) and (67.309bp,54.912bp)  .. (84.092bp,59.448bp);
  % Edge: T_2 -> T_4
  \draw [->] (53.142bp,39.741bp) .. controls (62.497bp,37.984bp) and (73.216bp,36.293bp)  .. (83bp,35bp) .. controls (107.2bp,31.803bp) and (134.52bp,29.047bp)  .. (165.96bp,26.777bp);
  % Edge: T_3 -> T_2
  \draw [->] (83.005bp,64.083bp) .. controls (76.488bp,62.56bp) and (69.394bp,60.77bp)  .. (52.743bp,56.203bp);
  % Edge: T_3 -> T_4
  \draw [->] (134.43bp,56.141bp) .. controls (142.24bp,52.189bp) and (151.08bp,47.712bp)  .. (168.62bp,38.832bp);
  % Edge: T_4 -> T_2
  \draw [->] (166.4bp,32.314bp) .. controls (144.28bp,35.086bp) and (111.47bp,38.237bp)  .. (83bp,42bp) .. controls (76.978bp,42.796bp) and (70.602bp,43.743bp)  .. (54.249bp,46.142bp);
  % Node: T_2
\begin{scope}
  \definecolor{strokecol}{rgb}{0.0,0.0,0.0};
  \pgfsetstrokecolor{strokecol}
  \draw (27bp,47bp) ellipse (27bp and 27bp);
  \draw (27bp,47bp) node {$T_2$};
\end{scope}
  % Node: T_3
\begin{scope}
  \definecolor{strokecol}{rgb}{0.0,0.0,0.0};
  \pgfsetstrokecolor{strokecol}
  \draw (110bp,68bp) ellipse (27bp and 27bp);
  \draw (110bp,68bp) node {$T_3$};
\end{scope}
  % Node: T_4
\begin{scope}
  \definecolor{strokecol}{rgb}{0.0,0.0,0.0};
  \pgfsetstrokecolor{strokecol}
  \draw (193bp,27bp) ellipse (27bp and 27bp);
  \draw (193bp,27bp) node {$T_4$};
\end{scope}
%
\end{tikzpicture}

 Zyklus \lightning 
\begin{tikzpicture}[>=latex,line join=bevel,]
  \pgfsetlinewidth{1bp}
%%
\pgfsetcolor{black}
  % Edge: T_3 -> T_4
  \draw [->] (54.075bp,27bp) .. controls (60bp,27bp) and (66.4bp,27bp)  .. (82.974bp,27bp);
  % Node: T_3
\begin{scope}
  \definecolor{strokecol}{rgb}{0.0,0.0,0.0};
  \pgfsetstrokecolor{strokecol}
  \draw (27bp,27bp) ellipse (27bp and 27bp);
  \draw (27bp,27bp) node {$T_3$};
\end{scope}
  % Node: T_4
\begin{scope}
  \definecolor{strokecol}{rgb}{0.0,0.0,0.0};
  \pgfsetstrokecolor{strokecol}
  \draw (110bp,27bp) ellipse (27bp and 27bp);
  \draw (110bp,27bp) node {$T_4$};
\end{scope}
%
\end{tikzpicture}


								
					\item \textbf{Output}: $c_3$
					\item \textbf{Status�nderungen}: 
\begin{tikzpicture}[>=latex,line join=bevel,]
  \pgfsetlinewidth{1bp}
%%
\pgfsetcolor{black}
  % Edge: T_3 -> T_4
  \draw [->] (54.075bp,27bp) .. controls (60bp,27bp) and (66.4bp,27bp)  .. (82.974bp,27bp);
  % Node: T_3
\begin{scope}
  \definecolor{strokecol}{rgb}{0.0,0.0,0.0};
  \pgfsetstrokecolor{strokecol}
  \draw (27bp,27bp) ellipse (27bp and 27bp);
  \draw (27bp,27bp) node {$T_3$};
\end{scope}
  % Node: T_4
\begin{scope}
  \definecolor{strokecol}{rgb}{0.0,0.0,0.0};
  \pgfsetstrokecolor{strokecol}
  \draw (110bp,27bp) ellipse (27bp and 27bp);
  \draw (110bp,27bp) node {$T_4$};
\end{scope}
%
\end{tikzpicture}


								$\xrightarrow{Garbage Collection}$ 
\begin{tikzpicture}[>=latex,line join=bevel,]
  \pgfsetlinewidth{1bp}
%%
\pgfsetcolor{black}
  % Node: T_4
\begin{scope}
  \definecolor{strokecol}{rgb}{0.0,0.0,0.0};
  \pgfsetstrokecolor{strokecol}
  \draw (27bp,27bp) ellipse (27bp and 27bp);
  \draw (27bp,27bp) node {$T_4$};
\end{scope}
%
\end{tikzpicture}


								
					\item \textbf{Output}: $c_4$
					\item \textbf{Status�nderungen}: 
\begin{tikzpicture}[>=latex,line join=bevel,]
  \pgfsetlinewidth{1bp}
%%
\pgfsetcolor{black}
  % Node: T_4
\begin{scope}
  \definecolor{strokecol}{rgb}{0.0,0.0,0.0};
  \pgfsetstrokecolor{strokecol}
  \draw (27bp,27bp) ellipse (27bp and 27bp);
  \draw (27bp,27bp) node {$T_4$};
\end{scope}
%
\end{tikzpicture}


								$\xrightarrow{Garbage Collection} emptyGraph()$
								
					\item $$S_{SGT} = r_1(x) r_2(x) w_3(x) w_4(x) w_1(x) a_1 w_2(x) a_2 c_3 c_4$$
				\end{itemize}
\end{enumerate}
	
\section{Optimistische Protokolle: BOCC und FOCC}
\begin{enumerate}[label={\alph*)}]
	\item \begin{itemize}
					\item \textbf{Output:} $r_1(x) r_2(x) r_1(y) r_3(x) c_1$
					\item \textbf{Validate($T_1$):} $accept()$ da bisher keine Transaktion erfolgreich abgeschlossen ist.
					\item \textbf{Write:} $w_1(x) w_1(y)$
					\item \textbf{Output:} $r_2(y) r_3(z) a_3$
					\item \textbf{Validate($T_3$):} $$RS(T_3) = \{x,z\} \cap \{x,y\} = WS(T_1) = \{x\} \neq \emptyset$$
					\item \textbf{Output:} $r_2(z) a_2$
					\item \textbf{Validate($T_2$):} $$RS(T_2) = \{x,y,z\} \cap \{x,y\} = WS(T_1) = \{x,y\} \neq \emptyset$$
					\item $$S_{BOCC} = r_1(x) r_2(x) r_1(y) r_3(x) c_1 w_1(x) w_1(y) r_2(y) r_3(z) a_3 r_2(z) a_2$$
				\end{itemize}
				
	\item \begin{itemize}
					\item \textbf{Output:} $r_1(x) r_2(x) r_1(y) r_3(x) a_1$
					\item \textbf{Validate($T_1$):} $$WS(T_1) = \{x,y\} \cap \{x\} = RS^{n^{1}}(T_2) = \{x\} \neq \emptyset$$
																					$$WS(T_1) = \{x,y\} \cap \{x\} = RS^{n^{1}}(T_3) = \{x\} \neq \emptyset$$
					
					\item \textbf{Output:} $r_2(y) r_3(z) c_3$
					\item \textbf{Validate($T_3$):} $$WS(T_3) = \{z\} \cap \{x,y\} = RS^{n^{2}}(T_2) = \emptyset$$
					\item \textbf{Write:} $w_3(z)$
					\item \textbf{Output:} $r_2(z) c_2$
					\item \textbf{Validate($T_2$):} $accept()$ da keine andere Transaktion zurzeit in der Lesephase ist. (Au�erdem ist $T_2$ eine rein lesende Transaktion.)
					\item $$S_{FOCC} = r_1(x) r_2(x) r_1(y) r_3(x) c_1 w_1(x) w_1(y) r_2(y) r_3(z) c_3 w_3(z) r_2(z) c_2$$
				\end{itemize}
\end{enumerate}

\section{MGL (Multiple Granularity Locking)}
	Angenommen $T_1$ und $T_2$ halten nicht kompatible Sperren $l_1$ und $l_2$ implizit oder explizit auf ein Datenobjekt $A$ in $H$.
	\begin{description}
		\item[Fall 1] Sei $A$ die Wurzel von $H$. Dann verhindert die MGL-Vertr�glichkeitsmatrix die obige Situation \lightning
		\item[Fall 2] O.b.d.A. gilt $l_1 <_S l_2$ in dem betrachteten Schedule $S$. Dann h�lt $T_1$ unter anderem auch Intentionssperren auf die Vorg�nger
			des Datenobjekts $A$. Also verhindert wiederum die MGL-Vertr�glichkeitsmatrix eine implizite oder explizite Sperre $l_2 $auf $A$ \lightning
	\end{description}
	
\section{Phantom Problem}
\begin{enumerate}[label={\alph*)}]
		\item \begin{tabularx}{\textwidth}{ | X | X | X |}
										\hline
										\textbf{Transaktion 1} & \textbf{Transaktion 2} & \textbf{Output} \\
										\hline \hline
										SELECT Count(*) FROM personen WHERE geburtsjahr > 1750 & & 2\\
										\hline
										& SELECT Count(*) FROM personen WHERE geburtsjahr > 1750& 2\\
										\hline
										INSERT INTO personen VALUES (121212,'Albert Einstein',1879) & & blockiert\\
										\hline 
										& INSERT INTO personen VALUES (424242,'Friedrich Schiller',1759) & blockiert\\
										\hline 
										& & INSERT von Transaktion 1 wird aufgef�hrt \\
										\hline 
										& & Fehlermeldung bei Transaktion 2 \\
										\hline
										SELECT Count(*) FROM personen WHERE geburtsjahr > 1750 & & 3 \\
										\hline
										& SELECT Count(*) FROM personen WHERE geburtsjahr > 1750 & blockiert \\
										\hline
										commit & & Ausgabe $3$ bei Transaktion 2 \\
										\hline
										& commit & \\
										\hline
								\end{tabularx}
		\item Leck mich doch, Arschloch!
		
	\item \begin{tabularx}{\textwidth}{ | X | X | X |}
										\hline
										\textbf{Transaktion 1} & \textbf{Transaktion 2} & \textbf{Output} \\
										\hline \hline
										SELECT Count(*) FROM personen WHERE geburtsjahr > 1750 & & 2\\
										\hline
										& SELECT Count(*) FROM personen WHERE geburtsjahr > 1750& 2\\
										\hline
										INSERT INTO personen VALUES (121212,'Albert Einstein',1879) & & erfolgreich\\
										\hline 
										& INSERT INTO personen VALUES (424242,'Friedrich Schiller',1759) & erfolgreich\\
										\hline 
										& & INSERT von Transaktion 1 wird aufgef�hrt \\
										\hline 
										& & Fehlermeldung bei Transaktion \\
										\hline
										SELECT Count(*) FROM personen WHERE geburtsjahr > 1750 & & 3 \\
										\hline
										& SELECT Count(*) FROM personen WHERE geburtsjahr > 1750 & 3 \\
										\hline
										commit & & \\
										\hline
										& commit & \\
										\hline
								\end{tabularx}
	\item Leck mich doch, Arschloch!
\end{enumerate}
\end{document}
