%%%
% Excercise Template
%%%

%\listfiles 
\documentclass[12pt]{scrartcl} 
\usepackage[english,ngerman]{babel}
    
\usepackage[latin1]{inputenc}
\usepackage[T1]{fontenc}

\usepackage{graphicx,enumitem,wasysym,multicol,libertine,xcolor,microtype,lipsum,fixltx2e,
	fancyhdr,vmargin,calc,lastpage,hyperref,listings,lastpage,lipsum,preview,tikz}
\usetikzlibrary{snakes,arrows,shapes}


% ------------------------------------------
% -------- xcolor - (Tango)
% ------------------------------------------

\definecolor{LightButter}{rgb}{0.98,0.91,0.31}
\definecolor{LightOrange}{rgb}{0.98,0.68,0.24}
\definecolor{LightChocolate}{rgb}{0.91,0.72,0.43}
\definecolor{LightChameleon}{rgb}{0.54,0.88,0.20}
\definecolor{LightSkyBlue}{rgb}{0.45,0.62,0.81}
\definecolor{LightPlum}{rgb}{0.68,0.50,0.66}
\definecolor{LightScarletRed}{rgb}{0.93,0.16,0.16}
\definecolor{Butter}{rgb}{0.93,0.86,0.25}
\definecolor{Orange}{rgb}{0.96,0.47,0.00}
\definecolor{Chocolate}{rgb}{0.75,0.49,0.07}
\definecolor{Chameleon}{rgb}{0.45,0.82,0.09}
\definecolor{SkyBlue}{rgb}{0.20,0.39,0.64}
\definecolor{Plum}{rgb}{0.46,0.31,0.48}
\definecolor{ScarletRed}{rgb}{0.80,0.00,0.00}
\definecolor{DarkButter}{rgb}{0.77,0.62,0.00}
\definecolor{DarkOrange}{rgb}{0.80,0.36,0.00}
\definecolor{DarkChocolate}{rgb}{0.56,0.35,0.01}
\definecolor{DarkChameleon}{rgb}{0.30,0.60,0.02}
\definecolor{DarkSkyBlue}{rgb}{0.12,0.29,0.53}
\definecolor{DarkPlum}{rgb}{0.36,0.21,0.40}
\definecolor{DarkScarletRed}{rgb}{0.64,0.00,0.00}
\definecolor{Aluminium1}{rgb}{0.93,0.93,0.92}
\definecolor{Aluminium2}{rgb}{0.82,0.84,0.81}
\definecolor{Aluminium3}{rgb}{0.73,0.74,0.71}
\definecolor{Aluminium4}{rgb}{0.53,0.54,0.52}
\definecolor{Aluminium5}{rgb}{0.33,0.34,0.32}
\definecolor{Aluminium6}{rgb}{0.18,0.20,0.21}
\definecolor{Brown}{cmyk}{0,0.81,1,0.60}
\definecolor{OliveGreen}{cmyk}{0.64,0,0.95,0.40}
\definecolor{CadetBlue}{cmyk}{0.62,0.57,0.23,0}

% ------------------------------------------
% -------- vmargin
% ------------------------------------------

%\setmarginsrb{hleftmargini}{htopmargini}{hrightmargini}{hbottommargini}%{hheadheighti}{hheadsepi}{hfootheighti}{hfootskipi}
\setpapersize{A4}
\setmarginsrb{3cm}{1cm}{3cm}{1cm}{8mm}{4mm}{3mm}{15mm}

% ------------------------------------------
% -------- listings
% ------------------------------------------
 
\lstdefinestyle{inline} {
		basicstyle=\normalsize
		}

\lstset{
		breakautoindent=true,
		breakindent=2em,
		breaklines=true,
		tabsize=4,
		frame=blrt,
		frameround=tttt,
		captionpos=b,
		basicstyle=\scriptsize\ttfamily,
		keywordstyle={\color{SkyBlue}},
		%commentstyle={\color{OliveGreen}},
		stringstyle={\color{OliveGreen}},
		showspaces=false,
		%numbers=right,
		%numberstyle=\scriptsize,
		%stepnumber=1, 
		%numbersep=5pt,
		%showtabs=false
		prebreak = \raisebox{0ex}[0ex][0ex]{\ensuremath{\hookleftarrow}},
		aboveskip={1.5\baselineskip},
		columns=fixed,
		upquote=true,
		extendedchars=true
		}


% ------------------------------------------
% -------- hyperref
% ------------------------------------------

\hypersetup{
	%breaklinks=true,
	pdfborder={0 0 0},
	bookmarks=true,         % show bookmarks bar?
	unicode=false,          % non-Latin characters in Acrobat�s bookmarks
	pdftoolbar=true,        % show Acrobat�s toolbar?
	pdfmenubar=true,        % show Acrobat�s menu?
	pdffitwindow=true,     % window fit to page when opened
	pdfstartview={FitH},    % fits the width of the page to the window
    pdfnewwindow=true,      % links in new window
    colorlinks=true,       % false: boxed links; true: colored links
    linkcolor=black,          % color of internal links
    citecolor=black,        % color of links to bibliography
    filecolor=magenta,      % color of file links
    urlcolor=DarkSkyBlue           % color of external links
}

% ------------------------------------------
% -------- fancyhdr
% ------------------------------------------
\fancyheadoffset{\marginparsep+\marginparwidth}
\fancyhf{}
\fancyhead[L]{\bfseries{\nouppercase{\leftmark}}}
\fancyfoot[C]{\bfseries{\thepage\ of \pageref{LastPage}}}
\renewcommand{\headrulewidth}{0.5pt}
\renewcommand{\footrulewidth}{0pt}

% ------------------------------------------
% -------- pdf specific
% ------------------------------------------

\pdfcompresslevel=3
%\pdfimageresolution=300
%\pdfinfo{
%/CreationDate (D:2010 09 01 00 00 00) % year(4) month(2) day(2) hour(2) minute(2) second(2)
%/ModDate      (D:2010 09 01 00 00 00) % modification date
%}

% ------------------------------------------
% -------- misc
% ------------------------------------------
\setcounter{secnumdepth}{3}
\setcounter{tocdepth}{3}
\clubpenalty = 10000
\widowpenalty = 10000
\displaywidowpenalty = 10000
\setlength\fboxsep{6pt}
\setlength\fboxrule{1pt}
\renewcommand*\oldstylenums[1]{{\fontfamily{fxlj}\selectfont #1}}

% ------------------------------------------
% -------- hyphenation rules
% ------------------------------------------
\hyphenation{}

\begin{document}
\pagestyle{fancy}

% -- Title

\title{\LARGE Transaktionssysteme -- Blatt 03\\ \large Team Amazonen -- Gruppe 1}
\author{\large Bastian Huber (51432) \and \large Sebastian Rainer (50882) \and \large Daniel Watzinger (51746)}
\date{\large\today}
\maketitle

\section{FSR, VSR, CSR}
Gegeben sei der Schedule $S := r_1(z)r_3(x)r_2(z)w_1(z)w_1(y)c_1w_2(y)w_2(z)c_2w_3(y)c_3$
	\begin{enumerate}[label={\alph*)}]
		\item
			Schrittgraph
		\item 
		
		Liest - Von - Relation: 

			$
			RF(S) = \{ (T_0 , z , T_1) , (T_0 , x , T_3) , (T_0 , z , T_2) , (T_0 , x , T_\infty) , (T_3 , y , T_\infty) , (T_2 , z , T_\infty)\} 
			$

			N�tzlich - Relation: 

			$
			w_0(z) \rightarrow r_1(z) \\
			w_0(z) \rightarrow r_2(z) \\
			w_0(x) \rightarrow r_3(x) \\
			w_0(x) \rightarrow r_\infty(x) \\
			w_3(y) \rightarrow r_\infty(y) \\
			w_2(z) \rightarrow r_\infty(z) \\
			r_1(z) \rightarrow w_1(z) \\
			r_1(z) \rightarrow w_1(y) \\
			r_2(z) \rightarrow w_2(y) \\
			r_2(z) \rightarrow w_2(z) \\ 
			r_3(x) \rightarrow w_3(y) \\
			$

			Lebendige Schritte: 

			$
			w_3(y) , w_2(z) , w_0(z) , w_0(x) , r_3(x) , r_2(z) , r_\infty(x) , r_\infty(y) , r_\infty(z)
			$

			Lebendige Liest - Von - Relation: 

			$
			LRF(S) = \{ (T_0 , x , T_3) , (T_0 , z , T_2) , (T_0 , x , T_\infty) , (T_3 , y , T_\infty) , (T_2 , z , T_\infty)\} \\ 
			$
			
			\item
				$
			conf(S) = \{ (r_1(z) , w_2(z)) , (r_2(z) , w_1(z)) , (w_1(z) , w_2(z)) , (w_1(y) , w_2(y)) ,\\ (w_1(y) , w_3(y)) , (w_2(y) , w_3(y))\} \\ 
			$
			\item
				Konfliktgraphzeichnung
			\item
				Es wird �berpr�ft ob $S \in FSR$. $S \in FSR \Leftrightarrow \exists S' : S'\ ist\ seriell\ und\ S' \approx_f S$.
				Laut Theorem 3.50 gilt $S \approx_f S' \Leftrightarrow op(S) = op(S') \and LRF(S) = LRF(S')$.
				Au�erdem gilt $LRF(S) = LRF(S') \Leftrightarrow D_r(S) = D_r(S')$.
				
				Bei allen zu testenden seriellen Historien $S'$ gilt $op(S) = op(S')$.
				
				\begin{itemize}
					\item Teste $S_1' = T_1 T_2 T_3$: $r_1(x) w_1(x) w_1(y) r_2(z) w_2(y) w_2(x) r_3(x) w_3(y)$. $(T_0, x, T_3) \in LRF(S) $ ist in $RF(S_1')$ nicht enthalten. Deshalb ist diese Reihenfolge nicht korrekt. Um $(T_0, x, T_3)$ zu gew�hrleisten muss $T_3$ vor $T_2$ und $T_1$ sein, denn beide schreiben auf $x$. Somit m�ssen nur noch zwei Folgen getestet werden:
					\item Teste $S_2' = T_3 T_1 T_2$: $r_3(x) w_3(y) r_1(x) w_1(x) w_1(y) r_2(z) w_2(y) w_2(x)$. noch nicht fertig!
				\end{itemize}
			
	\end{enumerate}
	
\section{FSR,VSR,CSR,COCSR}
Gegeben sei der Schedule $S := r_1(x)r_3(z)w_1(y)r_2(y)w_1(x)r_3(x)w_3(z)w_2(z)w_3(x)$
	\begin{enumerate}[label={\alph*)}]

		\item 
			$H_{S} (r_{1}(x)) = H_{S} (w_{0}(x)) = f_{0}^{x}()$\\
			$H_{S} (r_{3}(z)) = H_{S} (w_{0}(z)) = f_{0}^{z}()$\\
			$H_{S} (w_{1}(y)) = f_{1}^{y}(H_{S} (r_{1}(x))) = f_{1}^{y}(f_{0}^{x}())$\\
			$H_{S} (r_{2}(y)) = H_{S} (w_{1}(y)) = f_{1}^{y}(f_{0}^{x}())$\\
			$H_{S} (w_{1}(x)) = f_{1}^{x}(H_{S} (r_{1}(x))) = f_{1}^{x}(f_{0}^{x}())$\\
			$H_{S} (r_{3}(x)) = H_{S} (w_{1}(x)) = f_{1}^{x}(f_{0}^{x}())$\\
			$H_{S} (w_{3}(z)) = f_{3}^{z}(H_{S} (r_{3}(z)) , H_{S} (r_{3}(x)) ) = f_{3}^{z}(f_{0}^{z}(),f_{1}^{x}(f_0^{x}()))$\\
			$H_{S} (w_{2}(z)) = f_{2}^{z}(H_{S} (r_{2}(y))) = f_{2}^{y}(f_{1}^{y}(f_0^{x}()))$\\
			$H_{S} (w_{3}(x)) = f_{3}^{x}(H_{S} (r_{3}(x)),H_{S} (r_{3}(z))) = f_{3}^{x}(f_{1}^{x}(f_0^{x}()),f_{0}^{z}())$\\[0.5cm]
			$H_S[S](x) = H_S(w_3 (x)) = f_{3}^{x}(f_{1}^{x}(f_0^{x}()),f_{0}^{z}())$\\
			
			\item
			
				Liest - Von - Relation: 

			$
			RF(S) = \{ (T_0 , x , T_1) , (T_0 , z , T_3) , (T_1 , y , T_2) , (T_1 , x , T_3) , (T_3 , x , T_\infty) , (T_1 , y , T_\infty) , (T_2 , z , T_\infty)\} 
			$

			N�tzlich - Relation: 

			$
			w_0(z) \rightarrow r_3(z) \\
			w_0(x) \rightarrow r_1(x) \\
			w_1(y) \rightarrow r_2(y) \\
			w_1(x) \rightarrow r_3(x) \\
			w_3(x) \rightarrow r_\infty(x) \\
			w_1(y) \rightarrow r_\infty(y) \\
			w_2(z) \rightarrow r_\infty(z) \\
			r_1(x) \rightarrow w_1(y) \\
			r_1(x) \rightarrow w_1(x) \\
			r_2(y) \rightarrow w_2(z) \\
			r_3(z) \rightarrow w_3(z) \\ 
			r_3(x) \rightarrow w_3(z) \\
			r_3(x) \rightarrow w_3(x) \\
			r_3(z) \rightarrow w_3(x) \\
			$

			Lebendige Schritte: 

			$
			w_3(x) , w_1(y) , w_2(z) , w_1(x) , w_0(z) , w_0(x) , r_3(x) , r_2(y) , r_1(x) , r_3(z) , r_\infty(x) , r_\infty(y) , r_\infty(z)
			$

			Lebendige Liest - Von - Relation: 

			$
			LRF(S) = \{ (T_0 , x , T_1) , (T_0 , z , T_3) , (T_1 , y , T_2) , (T_1 , x , T_3) , (T_3 , x , T_\infty) , (T_1 , y , T_\infty) , (T_2 , z , T_\infty)\} \\ 
			$
			
			\item
			$
			conf(S) = \{ (r_1(x) , w_3(x)) , (r_3(z) , w_2(z)) , (w_1(y) , r_2(y)) , (w_1(x) , r_3(x)) , \\ (w_1(x) , w_3(x)) , (w_3(z) , w_2(z))\} \\ 
			$
			
			\item
				Der Konfliktgraph ist azyklisch, somit ist $S \in CSR$. Da $CSR \subset VSR \subset FSR$ gilt, ist $S \in VSR$ und $S \in FSR$.
				
				\item
Die Commitoperationen m�ssen anhand der im Konfliktgraphen beschriebenen Reihenfolge ablaufen. Damit $S \in COCSR$ gilt, muss gelten: $c_1 <_S c_3 <_S c_2$
	\end{enumerate}
	
\section{CSR,OCSR,COCSR}
Gegeben sei der Schedule $S := r_1(x)w_1(z)r_2(z)w_1(y)c_1r_3(y)w_2(z)c_2w_3(x)w_3(y)c_3$
	\begin{enumerate}[label={\alph*)}]
		\item 
			$
			conf(S) = \{ (r_1(x) , w_3(x)) , (w_1(z) , r_2(z)) , (w_1(z) , w_2(z)) , (w_1(y) , r_3(y)) , (w_1(y) , w_3(y))\} \\ 
			$
			
			\item
			
			Der Konfliktgraph ist azyklisch, somit gilt $S \in CSR$
			
			
			\item 
$S \in COCSR$ , da die Ordnung der Commitoperationen der Konfliktordnung entspricht, da gilt: $c_1 <_S c_2$ und $c_1 <_S c_3$. Da zudem $COCSR \subset OCSR$ gilt, ist $S \in OCSR$.
	\end{enumerate}
\end{document}
