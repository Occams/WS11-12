%%%
% Excercise Template
%%%

%\listfiles 
\documentclass[12pt]{scrartcl} 
\usepackage[english,ngerman]{babel}
    
\usepackage[latin1]{inputenc}
\usepackage[T1]{fontenc}

\usepackage{graphicx,enumitem,wasysym,multicol,libertine,xcolor,microtype,lipsum,fixltx2e,
	fancyhdr,vmargin,calc,lastpage,hyperref,listings,lastpage,lipsum,preview,tikz}
\usetikzlibrary{snakes,arrows,shapes}


% ------------------------------------------
% -------- xcolor - (Tango)
% ------------------------------------------

\definecolor{LightButter}{rgb}{0.98,0.91,0.31}
\definecolor{LightOrange}{rgb}{0.98,0.68,0.24}
\definecolor{LightChocolate}{rgb}{0.91,0.72,0.43}
\definecolor{LightChameleon}{rgb}{0.54,0.88,0.20}
\definecolor{LightSkyBlue}{rgb}{0.45,0.62,0.81}
\definecolor{LightPlum}{rgb}{0.68,0.50,0.66}
\definecolor{LightScarletRed}{rgb}{0.93,0.16,0.16}
\definecolor{Butter}{rgb}{0.93,0.86,0.25}
\definecolor{Orange}{rgb}{0.96,0.47,0.00}
\definecolor{Chocolate}{rgb}{0.75,0.49,0.07}
\definecolor{Chameleon}{rgb}{0.45,0.82,0.09}
\definecolor{SkyBlue}{rgb}{0.20,0.39,0.64}
\definecolor{Plum}{rgb}{0.46,0.31,0.48}
\definecolor{ScarletRed}{rgb}{0.80,0.00,0.00}
\definecolor{DarkButter}{rgb}{0.77,0.62,0.00}
\definecolor{DarkOrange}{rgb}{0.80,0.36,0.00}
\definecolor{DarkChocolate}{rgb}{0.56,0.35,0.01}
\definecolor{DarkChameleon}{rgb}{0.30,0.60,0.02}
\definecolor{DarkSkyBlue}{rgb}{0.12,0.29,0.53}
\definecolor{DarkPlum}{rgb}{0.36,0.21,0.40}
\definecolor{DarkScarletRed}{rgb}{0.64,0.00,0.00}
\definecolor{Aluminium1}{rgb}{0.93,0.93,0.92}
\definecolor{Aluminium2}{rgb}{0.82,0.84,0.81}
\definecolor{Aluminium3}{rgb}{0.73,0.74,0.71}
\definecolor{Aluminium4}{rgb}{0.53,0.54,0.52}
\definecolor{Aluminium5}{rgb}{0.33,0.34,0.32}
\definecolor{Aluminium6}{rgb}{0.18,0.20,0.21}
\definecolor{Brown}{cmyk}{0,0.81,1,0.60}
\definecolor{OliveGreen}{cmyk}{0.64,0,0.95,0.40}
\definecolor{CadetBlue}{cmyk}{0.62,0.57,0.23,0}

% ------------------------------------------
% -------- vmargin
% ------------------------------------------

%\setmarginsrb{hleftmargini}{htopmargini}{hrightmargini}{hbottommargini}%{hheadheighti}{hheadsepi}{hfootheighti}{hfootskipi}
\setpapersize{A4}
\setmarginsrb{3cm}{1cm}{3cm}{1cm}{8mm}{4mm}{3mm}{15mm}

% ------------------------------------------
% -------- listings
% ------------------------------------------
 
\lstdefinestyle{inline} {
		basicstyle=\normalsize
		}

\lstset{
		breakautoindent=true,
		breakindent=2em,
		breaklines=true,
		tabsize=4,
		frame=blrt,
		frameround=tttt,
		captionpos=b,
		basicstyle=\scriptsize\ttfamily,
		keywordstyle={\color{SkyBlue}},
		%commentstyle={\color{OliveGreen}},
		stringstyle={\color{OliveGreen}},
		showspaces=false,
		%numbers=right,
		%numberstyle=\scriptsize,
		%stepnumber=1, 
		%numbersep=5pt,
		%showtabs=false
		prebreak = \raisebox{0ex}[0ex][0ex]{\ensuremath{\hookleftarrow}},
		aboveskip={1.5\baselineskip},
		columns=fixed,
		upquote=true,
		extendedchars=true
		}


% ------------------------------------------
% -------- hyperref
% ------------------------------------------

\hypersetup{
	%breaklinks=true,
	pdfborder={0 0 0},
	bookmarks=true,         % show bookmarks bar?
	unicode=false,          % non-Latin characters in Acrobat�s bookmarks
	pdftoolbar=true,        % show Acrobat�s toolbar?
	pdfmenubar=true,        % show Acrobat�s menu?
	pdffitwindow=true,     % window fit to page when opened
	pdfstartview={FitH},    % fits the width of the page to the window
    pdfnewwindow=true,      % links in new window
    colorlinks=true,       % false: boxed links; true: colored links
    linkcolor=black,          % color of internal links
    citecolor=black,        % color of links to bibliography
    filecolor=magenta,      % color of file links
    urlcolor=DarkSkyBlue           % color of external links
}

% ------------------------------------------
% -------- fancyhdr
% ------------------------------------------
\fancyheadoffset{\marginparsep+\marginparwidth}
\fancyhf{}
\fancyhead[L]{\bfseries{\nouppercase{\leftmark}}}
\fancyfoot[C]{\bfseries{\thepage\ of \pageref{LastPage}}}
\renewcommand{\headrulewidth}{0.5pt}
\renewcommand{\footrulewidth}{0pt}

% ------------------------------------------
% -------- pdf specific
% ------------------------------------------

\pdfcompresslevel=3
%\pdfimageresolution=300
%\pdfinfo{
%/CreationDate (D:2010 09 01 00 00 00) % year(4) month(2) day(2) hour(2) minute(2) second(2)
%/ModDate      (D:2010 09 01 00 00 00) % modification date
%}

% ------------------------------------------
% -------- misc
% ------------------------------------------
\setcounter{secnumdepth}{3}
\setcounter{tocdepth}{3}
\clubpenalty = 10000
\widowpenalty = 10000
\displaywidowpenalty = 10000
\setlength\fboxsep{6pt}
\setlength\fboxrule{1pt}
\renewcommand*\oldstylenums[1]{{\fontfamily{fxlj}\selectfont #1}}

% ------------------------------------------
% -------- hyphenation rules
% ------------------------------------------
\hyphenation{}

\begin{document}
\pagestyle{fancy}

% -- Title

\title{\LARGE Transaktionssysteme -- Blatt 03\\ \large Team Amazonen -- Gruppe 1}
\author{\large Bastian Huber (51432) \and \large Sebastian Rainer (50882) \and \large Daniel Watzinger (51746)}
\date{\large\today}
\maketitle

\section{FSR, VSR, CSR}
Gegeben sei der Schedule $S := r_1(z)r_3(x)r_2(z)w_1(z)w_1(y)c_1w_2(y)w_2(z)c_2w_3(y)c_3$
	\begin{enumerate}[label={\alph*)}]
		\item
			
\begin{tikzpicture}[>=latex,line join=bevel,]
  \pgfsetlinewidth{1bp}
%%
\begin{scope}
  \pgfsetstrokecolor{black}
  \definecolor{strokecol}{rgb}{1.0,0.0,0.0};
  \pgfsetstrokecolor{strokecol}
  \draw [solid] (75bp,262bp) -- (75bp,393bp) -- (228bp,393bp) -- (228bp,262bp) -- cycle;
\end{scope}
\begin{scope}
  \pgfsetstrokecolor{black}
  \definecolor{strokecol}{rgb}{1.0,0.0,0.0};
  \pgfsetstrokecolor{strokecol}
  \draw [solid] (158bp,184bp) -- (158bp,254bp) -- (228bp,254bp) -- (228bp,184bp) -- cycle;
\end{scope}
  \pgfsetcolor{black}
  % Edge: w_0(z) -> r_1(z)
  \draw [->] (51.865bp,269.49bp) .. controls (59.445bp,273.14bp) and (67.966bp,277.24bp)  .. (85.306bp,285.59bp);
  % Edge: r_1(z) -> w_1(y)
  \draw [->] (137.07bp,297bp) .. controls (143bp,297bp) and (149.4bp,297bp)  .. (165.97bp,297bp);
  % Edge: r_1(z) -> w_1(z)
  \draw [->] (132.28bp,313.03bp) .. controls (141.58bp,320.03bp) and (152.65bp,328.36bp)  .. (170.86bp,342.08bp);
  % Edge: r_2(z) -> w_2(y)
  \draw [->] (137.07bp,219bp) .. controls (143bp,219bp) and (149.4bp,219bp)  .. (165.97bp,219bp);
  % Edge: w_3(y) -> r_\\infty(y)
  \draw [->] (220.07bp,88bp) .. controls (226bp,88bp) and (232.4bp,88bp)  .. (248.97bp,88bp);
  % Edge: w_2(z) -> r_\\infty(z)
  \draw [->] (220.07bp,149bp) .. controls (226bp,149bp) and (232.4bp,149bp)  .. (248.97bp,149bp);
  % Edge: r_3(x) -> w_3(y)
  \draw [->] (137.07bp,88bp) .. controls (143bp,88bp) and (149.4bp,88bp)  .. (165.97bp,88bp);
  % Edge: w_0(x) -> r_\\infty(x)
  \draw [->] (52.743bp,48.531bp) .. controls (59.73bp,45.856bp) and (67.459bp,42.898bp)  .. (84.409bp,36.411bp);
  % Edge: r_2(z) -> w_2(z)
  \draw [->] (131.02bp,201.7bp) .. controls (141.05bp,193.03bp) and (153.35bp,182.4bp)  .. (171.93bp,166.35bp);
  % Edge: w_0(x) -> r_3(x)
  \draw [->] (52.743bp,67.164bp) .. controls (59.73bp,69.752bp) and (67.459bp,72.614bp)  .. (84.409bp,78.892bp);
  % Edge: w_0(z) -> r_2(z)
  \draw [->] (51.865bp,246.51bp) .. controls (59.445bp,242.86bp) and (67.966bp,238.76bp)  .. (85.306bp,230.41bp);
  % Node: w_1(z)
\begin{scope}
  \definecolor{strokecol}{rgb}{0.0,0.0,0.0};
  \pgfsetstrokecolor{strokecol}
  \draw (193bp,358bp) ellipse (27bp and 27bp);
  \draw (193bp,358bp) node {$w_1(z)$};
\end{scope}
  % Node: w_0(y)
\begin{scope}
  \definecolor{strokecol}{rgb}{0.0,0.0,0.0};
  \pgfsetstrokecolor{strokecol}
  \draw (27bp,178bp) ellipse (27bp and 27bp);
  \draw (27bp,178bp) node {$w_0(y)$};
\end{scope}
  % Node: w_3(y)
\begin{scope}
  \definecolor{strokecol}{rgb}{0.0,0.0,0.0};
  \pgfsetstrokecolor{strokecol}
  \draw (193bp,88bp) ellipse (27bp and 27bp);
  \draw (193bp,88bp) node {$w_3(y)$};
\end{scope}
  % Node: w_0(x)
\begin{scope}
  \definecolor{strokecol}{rgb}{0.0,0.0,0.0};
  \pgfsetstrokecolor{strokecol}
  \draw (27bp,58bp) ellipse (27bp and 27bp);
  \draw (27bp,58bp) node {$w_0(x)$};
\end{scope}
  % Node: r_2(z)
\begin{scope}
  \definecolor{strokecol}{rgb}{0.0,0.0,0.0};
  \pgfsetstrokecolor{strokecol}
  \draw (110bp,219bp) ellipse (27bp and 27bp);
  \draw (110bp,219bp) node {$r_2(z)$};
\end{scope}
  % Node: r_1(z)
\begin{scope}
  \definecolor{strokecol}{rgb}{0.0,0.0,0.0};
  \pgfsetstrokecolor{strokecol}
  \draw (110bp,297bp) ellipse (27bp and 27bp);
  \draw (110bp,297bp) node {$r_1(z)$};
\end{scope}
  % Node: w_2(y)
\begin{scope}
  \definecolor{strokecol}{rgb}{0.0,0.0,0.0};
  \pgfsetstrokecolor{strokecol}
  \draw (193bp,219bp) ellipse (27bp and 27bp);
  \draw (193bp,219bp) node {$w_2(y)$};
\end{scope}
  % Node: w_0(z)
\begin{scope}
  \definecolor{strokecol}{rgb}{0.0,0.0,0.0};
  \pgfsetstrokecolor{strokecol}
  \draw (27bp,258bp) ellipse (27bp and 27bp);
  \draw (27bp,258bp) node {$w_0(z)$};
\end{scope}
  % Node: r_\\infty(z)
\begin{scope}
  \definecolor{strokecol}{rgb}{0.0,0.0,0.0};
  \pgfsetstrokecolor{strokecol}
  \draw (276bp,149bp) ellipse (27bp and 27bp);
  \draw (276bp,149bp) node {$r_\infty(z)$};
\end{scope}
  % Node: r_\\infty(x)
\begin{scope}
  \definecolor{strokecol}{rgb}{0.0,0.0,0.0};
  \pgfsetstrokecolor{strokecol}
  \draw (110bp,27bp) ellipse (27bp and 27bp);
  \draw (110bp,27bp) node {$r_\infty(x)$};
\end{scope}
  % Node: r_3(x)
\begin{scope}
  \definecolor{strokecol}{rgb}{0.0,0.0,0.0};
  \pgfsetstrokecolor{strokecol}
  \draw (110bp,88bp) ellipse (27bp and 27bp);
  \draw (110bp,88bp) node {$r_3(x)$};
\end{scope}
  % Node: r_\\infty(y)
\begin{scope}
  \definecolor{strokecol}{rgb}{0.0,0.0,0.0};
  \pgfsetstrokecolor{strokecol}
  \draw (276bp,88bp) ellipse (27bp and 27bp);
  \draw (276bp,88bp) node {$r_\infty(y)$};
\end{scope}
  % Node: w_1(y)
\begin{scope}
  \definecolor{strokecol}{rgb}{0.0,0.0,0.0};
  \pgfsetstrokecolor{strokecol}
  \draw (193bp,297bp) ellipse (27bp and 27bp);
  \draw (193bp,297bp) node {$w_1(y)$};
\end{scope}
  % Node: w_2(z)
\begin{scope}
  \definecolor{strokecol}{rgb}{0.0,0.0,0.0};
  \pgfsetstrokecolor{strokecol}
  \draw (193bp,149bp) ellipse (27bp and 27bp);
  \draw (193bp,149bp) node {$w_2(z)$};
\end{scope}
%
\end{tikzpicture}


		\item 
		
		Liest - Von - Relation: 

			$
			RF(S) = \{ (T_0 , z , T_1) , (T_0 , x , T_3) , (T_0 , z , T_2) , (T_0 , x , T_\infty) , (T_3 , y , T_\infty) , (T_2 , z , T_\infty)\} 
			$

			N�tzlich - Relation: 

			$
			w_0(z) \rightarrow r_1(z) \\
			w_0(z) \rightarrow r_2(z) \\
			w_0(x) \rightarrow r_3(x) \\
			w_0(x) \rightarrow r_\infty(x) \\
			w_3(y) \rightarrow r_\infty(y) \\
			w_2(z) \rightarrow r_\infty(z) \\
			r_1(z) \rightarrow w_1(z) \\
			r_1(z) \rightarrow w_1(y) \\
			r_2(z) \rightarrow w_2(y) \\
			r_2(z) \rightarrow w_2(z) \\ 
			r_3(x) \rightarrow w_3(y) \\
			$

			Lebendige Schritte: 

			$
			w_3(y) , w_2(z) , w_0(z) , w_0(x) , r_3(x) , r_2(z) , r_\infty(x) , r_\infty(y) , r_\infty(z)
			$

			Lebendige Liest - Von - Relation: 

			$
			LRF(S) = \{ (T_0 , x , T_3) , (T_0 , z , T_2) , (T_0 , x , T_\infty) , (T_3 , y , T_\infty) , (T_2 , z , T_\infty)\} \\ 
			$
			
			\item
				$
			conf(S) = \{ (r_1(z) , w_2(z)) , (r_2(z) , w_1(z)) , (w_1(z) , w_2(z)) , (w_1(y) , w_2(y)) ,\\ (w_1(y) , w_3(y)) , (w_2(y) , w_3(y))\} \\ 
			$
			\item
				
\begin{tikzpicture}[>=latex,line join=bevel,]
  \pgfsetlinewidth{1bp}
%%
\pgfsetcolor{black}
  % Edge: w_2(y) -> w_3(y)
  \draw [->] (136.63bp,157.99bp) .. controls (142.89bp,156.52bp) and (149.71bp,154.92bp)  .. (166.54bp,150.97bp);
  % Edge: w_1(y) -> w_2(y)
  \draw [->] (53.629bp,151.01bp) .. controls (59.895bp,152.48bp) and (66.712bp,154.08bp)  .. (83.538bp,158.03bp);
  % Edge: r_2(z) -> w_1(z)
  \draw [->] (54.075bp,27bp) .. controls (60bp,27bp) and (66.4bp,27bp)  .. (82.974bp,27bp);
  % Edge: r_1(z) -> w_2(z)
  \draw [->] (135.3bp,78.099bp) .. controls (142.51bp,75.162bp) and (150.54bp,71.891bp)  .. (167.71bp,64.894bp);
  % Edge: w_1(z) -> w_2(z)
  \draw [->] (135.74bp,35.553bp) .. controls (142.59bp,37.919bp) and (150.14bp,40.531bp)  .. (167.13bp,46.403bp);
  % Edge: w_1(y) -> w_3(y)
  \draw [->] (53.308bp,138.85bp) .. controls (62.587bp,136.89bp) and (73.213bp,134.97bp)  .. (83bp,134bp) .. controls (106.88bp,131.64bp) and (113.12bp,131.64bp)  .. (137bp,134bp) .. controls (143.42bp,134.64bp) and (150.21bp,135.68bp)  .. (166.69bp,138.85bp);
  % Node: w_1(z)
\begin{scope}
  \definecolor{strokecol}{rgb}{0.0,0.0,0.0};
  \pgfsetstrokecolor{strokecol}
  \draw (110bp,27bp) ellipse (27bp and 27bp);
  \draw (110bp,27bp) node {$w_1(z)$};
\end{scope}
  % Node: w_3(y)
\begin{scope}
  \definecolor{strokecol}{rgb}{0.0,0.0,0.0};
  \pgfsetstrokecolor{strokecol}
  \draw (193bp,145bp) ellipse (27bp and 27bp);
  \draw (193bp,145bp) node {$w_3(y)$};
\end{scope}
  % Node: w_2(z)
\begin{scope}
  \definecolor{strokecol}{rgb}{0.0,0.0,0.0};
  \pgfsetstrokecolor{strokecol}
  \draw (193bp,55bp) ellipse (27bp and 27bp);
  \draw (193bp,55bp) node {$w_2(z)$};
\end{scope}
  % Node: r_2(z)
\begin{scope}
  \definecolor{strokecol}{rgb}{0.0,0.0,0.0};
  \pgfsetstrokecolor{strokecol}
  \draw (27bp,27bp) ellipse (27bp and 27bp);
  \draw (27bp,27bp) node {$r_2(z)$};
\end{scope}
  % Node: r_1(z)
\begin{scope}
  \definecolor{strokecol}{rgb}{0.0,0.0,0.0};
  \pgfsetstrokecolor{strokecol}
  \draw (110bp,88bp) ellipse (27bp and 27bp);
  \draw (110bp,88bp) node {$r_1(z)$};
\end{scope}
  % Node: w_2(y)
\begin{scope}
  \definecolor{strokecol}{rgb}{0.0,0.0,0.0};
  \pgfsetstrokecolor{strokecol}
  \draw (110bp,164bp) ellipse (27bp and 27bp);
  \draw (110bp,164bp) node {$w_2(y)$};
\end{scope}
  % Node: w_1(y)
\begin{scope}
  \definecolor{strokecol}{rgb}{0.0,0.0,0.0};
  \pgfsetstrokecolor{strokecol}
  \draw (27bp,145bp) ellipse (27bp and 27bp);
  \draw (27bp,145bp) node {$w_1(y)$};
\end{scope}
%
\end{tikzpicture}


			\item
				
\begin{tikzpicture}[>=latex,line join=bevel,]
  \pgfsetlinewidth{1bp}
%%
\pgfsetcolor{black}
  % Edge: T_1 -> T_2
  \draw [->] (54.075bp,43.882bp) .. controls (60bp,43.769bp) and (66.4bp,43.729bp)  .. (82.974bp,43.883bp);
  % Edge: T_2 -> T_3
  \draw [->] (136.63bp,39.988bp) .. controls (142.89bp,38.519bp) and (149.71bp,36.919bp)  .. (166.54bp,32.973bp);
  % Edge: T_2 -> T_1
  \draw [->] (82.974bp,48.823bp) .. controls (77.051bp,48.974bp) and (70.653bp,49.028bp)  .. (54.075bp,48.824bp);
  % Edge: T_1 -> T_3
  \draw [->] (49.687bp,30.9bp) .. controls (59.407bp,25.019bp) and (71.301bp,18.983bp)  .. (83bp,16bp) .. controls (107.23bp,9.8208bp) and (135.33bp,12.891bp)  .. (166.96bp,19.599bp);
  % Node: T_2
\begin{scope}
  \definecolor{strokecol}{rgb}{0.0,0.0,0.0};
  \pgfsetstrokecolor{strokecol}
  \draw (110bp,46bp) ellipse (27bp and 27bp);
  \draw (110bp,46bp) node {$T_2$};
\end{scope}
  % Node: T_3
\begin{scope}
  \definecolor{strokecol}{rgb}{0.0,0.0,0.0};
  \pgfsetstrokecolor{strokecol}
  \draw (193bp,27bp) ellipse (27bp and 27bp);
  \draw (193bp,27bp) node {$T_3$};
\end{scope}
  % Node: T_1
\begin{scope}
  \definecolor{strokecol}{rgb}{0.0,0.0,0.0};
  \pgfsetstrokecolor{strokecol}
  \draw (27bp,46bp) ellipse (27bp and 27bp);
  \draw (27bp,46bp) node {$T_1$};
\end{scope}
%
\end{tikzpicture}


			\item
				Es wird �berpr�ft ob $S \in FSR$. $S \in FSR \Leftrightarrow \exists S' : S'\ ist\ seriell\ und\ S' \approx_f S$.
				Laut Theorem 3.50 gilt $S \approx_f S' \Leftrightarrow op(S) = op(S') \land LRF(S) = LRF(S')$.
				
				Bei allen zu testenden seriellen Historien $S'$ gilt $op(S) = op(S')$.
				
				\begin{itemize}
					\item Teste $S_1' = T_1 T_2 T_3 = r_1(z) w_1(z) w_1(y) r_2(z) w_2(y) w_2(y) r_3(x) w_3(y)$. $(T_0, z, T_2) \in LRF(S) $ ist in $RF(S_1')$ nicht enthalten. Deshalb ist diese Reihenfolge nicht korrekt. Um $(T_0, z, T_2) \in RF(S_1')$ zu gew�hrleisten muss $T_2$ vor $T_1$ sein, denn $T_1$ schreibt auf $z$. Somit testen wir nur noch folgende drei F�lle:
					\item Teste $S_2' = T_2 T_1 T_3 = r_2(z) w_2(y) w_2(y) r_1(z) w_1(z) w_1(y) r_3(x) w_3(y)$. $(T_2 , z , T_\infty) \in LRF(S)$ aber es gilt nicht $(T_2 , z , T_\infty) \in RF(S_2')$. Somit teste:
					\item Teste $S_3' = T_2 T_3 T_1 = r_2(z) w_2(y) w_2(y) r_3(x) w_3(y) r_1(z) w_1(z) w_1(y)$. $(T_3 , y , T_\infty) \in LRF(S)$ aber es gilt nicht $(T_3 , y , T_\infty) \in RF(S_3')$. Somit testen wir:
					\item Teste $S_4' = T_3 T_2 T_1 = r_3(x) w_3(y) r_2(z) w_2(y) w_2(y) r_1(z) w_1(z) w_1(y)$. $(T_2 , z , T_\infty) \in LRF(S)$ aber es gilt nicht $(T_2 , z , T_\infty) \in RF(S_4')$.
				\end{itemize}
				
				Somit gibt es keine serielle Historie $S'$ mit $op(S) = op(S')$ f�r die $LRF(S) = LRF(S')$ gilt (bei den Tests wird immer gezeigt dass ein Element nicht in $RF(...)$ gilt. Das reicht aus denn falls ein Element nicht in $RF$ liegt, dann ist es auch nicht in $LRF$).
				Daraus folgt dass $S \not\in FSR$ und somit auch nicht in $VSR$ und $CSR$ enthalten ist. 
			
			
Alternative Loesung:

Anhand von LRF(S) lassen sich f�r die seriellen Historien bereits einige Bedingungen heraus lesen:
	\begin{itemize}
		\item aus $(T_0 , z , T_2)$ folgt, dass fuer die serielle Historie gelten muss $T_2 < T_1$
		\item aus $(T_2 , z , T_\infty)$ folgt, dass fuer die serielle Historie gelten muss $T_1 < T_2$	
	\end{itemize}
Beide Bedingungen koennen aber in einer seriellen Hostorie nicht umgesetzt werden, somit gibt es keine serielle Historie mit $S' \approx_f S$. Daraus folgt dass $S \not\in FSR$ und somit auch nicht in $VSR$ und $CSR$ enthalten ist. 
	\end{enumerate}
	
\section{FSR,VSR,CSR,COCSR}
Gegeben sei der Schedule $S := r_1(x)r_3(z)w_1(y)r_2(y)w_1(x)r_3(x)w_3(z)w_2(z)w_3(x)$
	\begin{enumerate}[label={\alph*)}]

		\item 
			$H_{S} (r_{1}(x)) = H_{S} (w_{0}(x)) = f_{0}^{x}()$\\
			$H_{S} (r_{3}(z)) = H_{S} (w_{0}(z)) = f_{0}^{z}()$\\
			$H_{S} (w_{1}(y)) = f_{1}^{y}(H_{S} (r_{1}(x))) = f_{1}^{y}(f_{0}^{x}())$\\
			$H_{S} (r_{2}(y)) = H_{S} (w_{1}(y)) = f_{1}^{y}(f_{0}^{x}())$\\
			$H_{S} (w_{1}(x)) = f_{1}^{x}(H_{S} (r_{1}(x))) = f_{1}^{x}(f_{0}^{x}())$\\
			$H_{S} (r_{3}(x)) = H_{S} (w_{1}(x)) = f_{1}^{x}(f_{0}^{x}())$\\
			$H_{S} (w_{3}(z)) = f_{3}^{z}(H_{S} (r_{3}(z)) , H_{S} (r_{3}(x)) ) = f_{3}^{z}(f_{0}^{z}(),f_{1}^{x}(f_0^{x}()))$\\
			$H_{S} (w_{2}(z)) = f_{2}^{z}(H_{S} (r_{2}(y))) = f_{2}^{y}(f_{1}^{y}(f_0^{x}()))$\\
			$H_{S} (w_{3}(x)) = f_{3}^{x}(H_{S} (r_{3}(x)),H_{S} (r_{3}(z))) = f_{3}^{x}(f_{1}^{x}(f_0^{x}()),f_{0}^{z}())$\\[0.5cm]
			$H_S[S](x) = H_S(w_3 (x)) = f_{3}^{x}(f_{1}^{x}(f_0^{x}()),f_{0}^{z}())$\\
			
		\item
			
\begin{tikzpicture}[>=latex,line join=bevel,]
  \pgfsetlinewidth{1bp}
%%
\begin{scope}
  \pgfsetstrokecolor{black}
  \definecolor{strokecol}{rgb}{1.0,0.0,0.0};
  \pgfsetstrokecolor{strokecol}
  \draw [solid] (8bp,159bp) -- (8bp,229bp) -- (78bp,229bp) -- (78bp,159bp) -- cycle;
\end{scope}
\begin{scope}
  \pgfsetstrokecolor{black}
  \definecolor{strokecol}{rgb}{1.0,0.0,0.0};
  \pgfsetstrokecolor{strokecol}
  \draw [solid] (340bp,62bp) -- (340bp,132bp) -- (410bp,132bp) -- (410bp,62bp) -- cycle;
\end{scope}
  \pgfsetcolor{black}
  % Edge: r_2(y) -> w_2(z)
  \draw [->] (319.07bp,216bp) .. controls (325bp,216bp) and (331.4bp,216bp)  .. (347.97bp,216bp);
  % Edge: w_3(x) -> r_\\infty(x)
  \draw [->] (402.07bp,27bp) .. controls (408bp,27bp) and (414.4bp,27bp)  .. (430.97bp,27bp);
  % Edge: r_1(x) -> w_1(x)
  \draw [->] (151.74bp,114.84bp) .. controls (158.73bp,112.25bp) and (166.46bp,109.39bp)  .. (183.41bp,103.11bp);
  % Edge: r_3(z) -> w_3(x)
  \draw [->] (319.07bp,28.356bp) .. controls (325bp,28.21bp) and (331.4bp,28.052bp)  .. (347.97bp,27.643bp);
  % Edge: w_1(y) -> r_\\infty(y)
  \draw [->] (236.07bp,155bp) .. controls (242bp,155bp) and (248.4bp,155bp)  .. (264.97bp,155bp);
  % Edge: w_2(z) -> r_\\infty(z)
  \draw [->] (402.07bp,216bp) .. controls (408bp,216bp) and (414.4bp,216bp)  .. (430.97bp,216bp);
  % Edge: w_1(x) -> r_3(x)
  \draw [->] (236.07bp,94bp) .. controls (242bp,94bp) and (248.4bp,94bp)  .. (264.97bp,94bp);
  % Edge: w_0(z) -> r_3(z)
  \draw [->] (70.433bp,29bp) .. controls (114.82bp,29bp) and (204.38bp,29bp)  .. (264.76bp,29bp);
  % Edge: r_3(x) -> w_3(x)
  \draw [->] (313.44bp,77.093bp) .. controls (323.22bp,68.999bp) and (335.1bp,59.177bp)  .. (353.57bp,43.902bp);
  % Edge: r_3(x) -> w_3(z)
  \draw [->] (319.07bp,94.966bp) .. controls (325bp,95.185bp) and (331.4bp,95.422bp)  .. (347.97bp,96.036bp);
  % Edge: r_1(x) -> w_1(y)
  \draw [->] (151.74bp,133.47bp) .. controls (158.73bp,136.14bp) and (166.46bp,139.1bp)  .. (183.41bp,145.59bp);
  % Edge: r_3(z) -> w_3(z)
  \draw [->] (313.44bp,46.159bp) .. controls (323.3bp,54.436bp) and (335.28bp,64.495bp)  .. (353.85bp,80.082bp);
  % Edge: w_0(x) -> r_1(x)
  \draw [->] (70.075bp,124bp) .. controls (76bp,124bp) and (82.4bp,124bp)  .. (98.974bp,124bp);
  % Edge: w_1(y) -> r_2(y)
  \draw [->] (231.28bp,171.03bp) .. controls (240.58bp,178.03bp) and (251.65bp,186.36bp)  .. (269.86bp,200.08bp);
  % Node: w_3(x)
\begin{scope}
  \definecolor{strokecol}{rgb}{0.0,0.0,0.0};
  \pgfsetstrokecolor{strokecol}
  \draw (375bp,27bp) ellipse (27bp and 27bp);
  \draw (375bp,27bp) node {$w_3(x)$};
\end{scope}
  % Node: r_1(x)
\begin{scope}
  \definecolor{strokecol}{rgb}{0.0,0.0,0.0};
  \pgfsetstrokecolor{strokecol}
  \draw (126bp,124bp) ellipse (27bp and 27bp);
  \draw (126bp,124bp) node {$r_1(x)$};
\end{scope}
  % Node: w_0(y)
\begin{scope}
  \definecolor{strokecol}{rgb}{0.0,0.0,0.0};
  \pgfsetstrokecolor{strokecol}
  \draw (43bp,194bp) ellipse (27bp and 27bp);
  \draw (43bp,194bp) node {$w_0(y)$};
\end{scope}
  % Node: r_\\infty(y)
\begin{scope}
  \definecolor{strokecol}{rgb}{0.0,0.0,0.0};
  \pgfsetstrokecolor{strokecol}
  \draw (292bp,155bp) ellipse (27bp and 27bp);
  \draw (292bp,155bp) node {$r_\infty(y)$};
\end{scope}
  % Node: w_0(x)
\begin{scope}
  \definecolor{strokecol}{rgb}{0.0,0.0,0.0};
  \pgfsetstrokecolor{strokecol}
  \draw (43bp,124bp) ellipse (27bp and 27bp);
  \draw (43bp,124bp) node {$w_0(x)$};
\end{scope}
  % Node: w_0(z)
\begin{scope}
  \definecolor{strokecol}{rgb}{0.0,0.0,0.0};
  \pgfsetstrokecolor{strokecol}
  \draw (43bp,29bp) ellipse (27bp and 27bp);
  \draw (43bp,29bp) node {$w_0(z)$};
\end{scope}
  % Node: r_\\infty(z)
\begin{scope}
  \definecolor{strokecol}{rgb}{0.0,0.0,0.0};
  \pgfsetstrokecolor{strokecol}
  \draw (458bp,216bp) ellipse (27bp and 27bp);
  \draw (458bp,216bp) node {$r_\infty(z)$};
\end{scope}
  % Node: r_2(y)
\begin{scope}
  \definecolor{strokecol}{rgb}{0.0,0.0,0.0};
  \pgfsetstrokecolor{strokecol}
  \draw (292bp,216bp) ellipse (27bp and 27bp);
  \draw (292bp,216bp) node {$r_2(y)$};
\end{scope}
  % Node: w_1(x)
\begin{scope}
  \definecolor{strokecol}{rgb}{0.0,0.0,0.0};
  \pgfsetstrokecolor{strokecol}
  \draw (209bp,94bp) ellipse (27bp and 27bp);
  \draw (209bp,94bp) node {$w_1(x)$};
\end{scope}
  % Node: r_3(z)
\begin{scope}
  \definecolor{strokecol}{rgb}{0.0,0.0,0.0};
  \pgfsetstrokecolor{strokecol}
  \draw (292bp,29bp) ellipse (27bp and 27bp);
  \draw (292bp,29bp) node {$r_3(z)$};
\end{scope}
  % Node: r_\\infty(x)
\begin{scope}
  \definecolor{strokecol}{rgb}{0.0,0.0,0.0};
  \pgfsetstrokecolor{strokecol}
  \draw (458bp,27bp) ellipse (27bp and 27bp);
  \draw (458bp,27bp) node {$r_\infty(x)$};
\end{scope}
  % Node: r_3(x)
\begin{scope}
  \definecolor{strokecol}{rgb}{0.0,0.0,0.0};
  \pgfsetstrokecolor{strokecol}
  \draw (292bp,94bp) ellipse (27bp and 27bp);
  \draw (292bp,94bp) node {$r_3(x)$};
\end{scope}
  % Node: w_3(z)
\begin{scope}
  \definecolor{strokecol}{rgb}{0.0,0.0,0.0};
  \pgfsetstrokecolor{strokecol}
  \draw (375bp,97bp) ellipse (27bp and 27bp);
  \draw (375bp,97bp) node {$w_3(z)$};
\end{scope}
  % Node: w_1(y)
\begin{scope}
  \definecolor{strokecol}{rgb}{0.0,0.0,0.0};
  \pgfsetstrokecolor{strokecol}
  \draw (209bp,155bp) ellipse (27bp and 27bp);
  \draw (209bp,155bp) node {$w_1(y)$};
\end{scope}
  % Node: w_2(z)
\begin{scope}
  \definecolor{strokecol}{rgb}{0.0,0.0,0.0};
  \pgfsetstrokecolor{strokecol}
  \draw (375bp,216bp) ellipse (27bp and 27bp);
  \draw (375bp,216bp) node {$w_2(z)$};
\end{scope}
%
\end{tikzpicture}


			
				Liest - Von - Relation: 

			$
			RF(S) = \{ (T_0 , x , T_1) , (T_0 , z , T_3) , (T_1 , y , T_2) , (T_1 , x , T_3) , (T_3 , x , T_\infty) , (T_1 , y , T_\infty) , (T_2 , z , T_\infty)\} 
			$

			N�tzlich - Relation: 

			$
			w_0(z) \rightarrow r_3(z) \\
			w_0(x) \rightarrow r_1(x) \\
			w_1(y) \rightarrow r_2(y) \\
			w_1(x) \rightarrow r_3(x) \\
			w_3(x) \rightarrow r_\infty(x) \\
			w_1(y) \rightarrow r_\infty(y) \\
			w_2(z) \rightarrow r_\infty(z) \\
			r_1(x) \rightarrow w_1(y) \\
			r_1(x) \rightarrow w_1(x) \\
			r_2(y) \rightarrow w_2(z) \\
			r_3(z) \rightarrow w_3(z) \\ 
			r_3(x) \rightarrow w_3(z) \\
			r_3(x) \rightarrow w_3(x) \\
			r_3(z) \rightarrow w_3(x) \\
			$

			Lebendige Schritte: 

			$
			w_3(x) , w_1(y) , w_2(z) , w_1(x) , w_0(z) , w_0(x) , r_3(x) , r_2(y) , r_1(x) , r_3(z) , r_\infty(x) , r_\infty(y) , r_\infty(z)
			$

			Lebendige Liest - Von - Relation: 

			$
			LRF(S) = \{ (T_0 , x , T_1) , (T_0 , z , T_3) , (T_1 , y , T_2) , (T_1 , x , T_3) , (T_3 , x , T_\infty) , (T_1 , y , T_\infty) , (T_2 , z , T_\infty)\} \\ 
			$
			
			\item
				
\begin{tikzpicture}[>=latex,line join=bevel,]
  \pgfsetlinewidth{1bp}
%%
\pgfsetcolor{black}
  % Edge: T_1 -> T_2
  \draw [->] (53.308bp,52.149bp) .. controls (62.587bp,54.114bp) and (73.213bp,56.031bp)  .. (83bp,57bp) .. controls (106.88bp,59.365bp) and (113.12bp,59.365bp)  .. (137bp,57bp) .. controls (143.42bp,56.364bp) and (150.21bp,55.32bp)  .. (166.69bp,52.149bp);
  % Edge: T_3 -> T_2
  \draw [->] (136.63bp,33.012bp) .. controls (142.89bp,34.481bp) and (149.71bp,36.081bp)  .. (166.54bp,40.027bp);
  % Edge: T_1 -> T_3
  \draw [->] (53.629bp,39.988bp) .. controls (59.895bp,38.519bp) and (66.712bp,36.919bp)  .. (83.538bp,32.973bp);
  % Node: T_2
\begin{scope}
  \definecolor{strokecol}{rgb}{0.0,0.0,0.0};
  \pgfsetstrokecolor{strokecol}
  \draw (193bp,46bp) ellipse (27bp and 27bp);
  \draw (193bp,46bp) node {$T_2$};
\end{scope}
  % Node: T_3
\begin{scope}
  \definecolor{strokecol}{rgb}{0.0,0.0,0.0};
  \pgfsetstrokecolor{strokecol}
  \draw (110bp,27bp) ellipse (27bp and 27bp);
  \draw (110bp,27bp) node {$T_3$};
\end{scope}
  % Node: T_1
\begin{scope}
  \definecolor{strokecol}{rgb}{0.0,0.0,0.0};
  \pgfsetstrokecolor{strokecol}
  \draw (27bp,46bp) ellipse (27bp and 27bp);
  \draw (27bp,46bp) node {$T_1$};
\end{scope}
%
\end{tikzpicture}


			$
			conf(S) = \{ (r_1(x) , w_3(x)) , (r_3(z) , w_2(z)) , (w_1(y) , r_2(y)) , (w_1(x) , r_3(x)) , \\ (w_1(x) , w_3(x)) , (w_3(z) , w_2(z))\} \\ 
			$
			
			\item
				Der Konfliktgraph ist azyklisch, somit ist $S \in CSR$. Da $CSR \subset VSR \subset FSR$ gilt, ist $S \in VSR$ und $S \in FSR$.
				
				\item
Die Commitoperationen m�ssen anhand der im Konfliktgraphen beschriebenen Reihenfolge ablaufen. Damit $S \in COCSR$ gilt, muss gelten: $c_1 <_S c_3 <_S c_2$
	\end{enumerate}
	
\section{CSR,OCSR,COCSR}
Gegeben sei der Schedule $S := r_1(x)w_1(z)r_2(z)w_1(y)c_1r_3(y)w_2(z)c_2w_3(x)w_3(y)c_3$
	\begin{enumerate}[label={\alph*)}]
		\item 
			$
			conf(S) = \{ (r_1(x) , w_3(x)) , (w_1(z) , r_2(z)) , (w_1(z) , w_2(z)) , (w_1(y) , r_3(y)) , (w_1(y) , w_3(y))\} \\ 
			$
			
		\item
				
\begin{tikzpicture}[>=latex,line join=bevel,]
  \pgfsetlinewidth{1bp}
%%
\pgfsetcolor{black}
  % Edge: T_1 -> T_2
  \draw [->] (52.743bp,66.469bp) .. controls (59.73bp,69.144bp) and (67.459bp,72.102bp)  .. (84.409bp,78.589bp);
  % Edge: T_1 -> T_3
  \draw [->] (52.743bp,47.836bp) .. controls (59.73bp,45.248bp) and (67.459bp,42.386bp)  .. (84.409bp,36.108bp);
  % Node: T_2
\begin{scope}
  \definecolor{strokecol}{rgb}{0.0,0.0,0.0};
  \pgfsetstrokecolor{strokecol}
  \draw (110bp,88bp) ellipse (27bp and 27bp);
  \draw (110bp,88bp) node {$T_2$};
\end{scope}
  % Node: T_3
\begin{scope}
  \definecolor{strokecol}{rgb}{0.0,0.0,0.0};
  \pgfsetstrokecolor{strokecol}
  \draw (110bp,27bp) ellipse (27bp and 27bp);
  \draw (110bp,27bp) node {$T_3$};
\end{scope}
  % Node: T_1
\begin{scope}
  \definecolor{strokecol}{rgb}{0.0,0.0,0.0};
  \pgfsetstrokecolor{strokecol}
  \draw (27bp,57bp) ellipse (27bp and 27bp);
  \draw (27bp,57bp) node {$T_1$};
\end{scope}
%
\end{tikzpicture}


				Der Konfliktgraph ist azyklisch, somit gilt $S \in CSR$
			
			
		\item 
$S \in COCSR$ , da die Ordnung der Commitoperationen der Konfliktordnung entspricht, da gilt: $c_1 <_S c_2$ und $c_1 <_S c_3$. Da zudem $COCSR \subset OCSR$ gilt, ist $S \in OCSR$.
	\end{enumerate}
\end{document}
