\documentclass{article}
\usepackage[T1]{fontenc}
\usepackage[ngerman,english]{babel}
\usepackage[latin1]{inputenc}
\usepackage{libertine,calc,microtype,parskip,lipsum,booktabs,textcomp,csquotes,enumerate,amssymb,vmargin,fancyhdr,fixltx2e,makeidx,listings,ellipsis,remreset,xcolor,lastpage,caption,fancybox,verbatim,amsmath}
\usepackage{graphicx}
\usepackage[pdftex]{hyperref}
\usepackage{amsmath}

% Title
\def\thetitle{Multimedia Datenbanken --- Blatt 02}

% ------------------------------------------
% -------- xcolor - (Tango)
% ------------------------------------------

\definecolor{LightButter}{rgb}{0.98,0.91,0.31}
\definecolor{LightOrange}{rgb}{0.98,0.68,0.24}
\definecolor{LightChocolate}{rgb}{0.91,0.72,0.43}
\definecolor{LightChameleon}{rgb}{0.54,0.88,0.20}
\definecolor{LightSkyBlue}{rgb}{0.45,0.62,0.81}
\definecolor{LightPlum}{rgb}{0.68,0.50,0.66}
\definecolor{LightScarletRed}{rgb}{0.93,0.16,0.16}
\definecolor{Butter}{rgb}{0.93,0.86,0.25}
\definecolor{Orange}{rgb}{0.96,0.47,0.00}
\definecolor{Chocolate}{rgb}{0.75,0.49,0.07}
\definecolor{Chameleon}{rgb}{0.45,0.82,0.09}
\definecolor{SkyBlue}{rgb}{0.20,0.39,0.64}
\definecolor{Plum}{rgb}{0.46,0.31,0.48}
\definecolor{ScarletRed}{rgb}{0.80,0.00,0.00}
\definecolor{DarkButter}{rgb}{0.77,0.62,0.00}
\definecolor{DarkOrange}{rgb}{0.80,0.36,0.00}
\definecolor{DarkChocolate}{rgb}{0.56,0.35,0.01}
\definecolor{DarkChameleon}{rgb}{0.30,0.60,0.02}
\definecolor{DarkSkyBlue}{rgb}{0.12,0.29,0.53}
\definecolor{DarkPlum}{rgb}{0.36,0.21,0.40}
\definecolor{DarkScarletRed}{rgb}{0.64,0.00,0.00}
\definecolor{Aluminium1}{rgb}{0.93,0.93,0.92}
\definecolor{Aluminium2}{rgb}{0.82,0.84,0.81}
\definecolor{Aluminium3}{rgb}{0.73,0.74,0.71}
\definecolor{Aluminium4}{rgb}{0.53,0.54,0.52}
\definecolor{Aluminium5}{rgb}{0.33,0.34,0.32}
\definecolor{Aluminium6}{rgb}{0.18,0.20,0.21}
\definecolor{Brown}{cmyk}{0,0.81,1,0.60}
\definecolor{OliveGreen}{cmyk}{0.64,0,0.95,0.40}
\definecolor{CadetBlue}{cmyk}{0.62,0.57,0.23,0}

% ------------------------------------------
% -------- vmargin
% ------------------------------------------

%\setmarginsrb{hleftmargini}{htopmargini}{hrightmargini}{hbottommargini}%{hheadheighti}{hheadsepi}{hfootheighti}{hfootskipi}
\setpapersize{A4}
\setmarginsrb{3cm}{1cm}{3cm}{1cm}{6mm}{7mm}{5mm}{15mm}

% ------------------------------------------
% -------- fancyhdr
% ------------------------------------------
%\fancyheadoffset[L]{\marginparsep+\marginparwidth}
\fancyhf{}
\fancyhead[L]{\bfseries{\nouppercase{\thetitle}}}
\fancyhead[R]{\bfseries{Seite \thepage\ von \pageref{LastPage}}}
\renewcommand{\headrulewidth}{0.5pt}
\renewcommand{\footrulewidth}{0pt}
\fancypagestyle{plain}{
\fancyhf{}
\fancyfoot[R]{\bfseries{Seite \thepage\ von \pageref{LastPage}}}
\renewcommand{\headrulewidth}{0pt}
\renewcommand{\footrulewidth}{0pt}
}

% ------------------------------------------
% -------- hyperref
% ------------------------------------------

\hypersetup{
	%breaklinks=true,
	pdfborder={0 0 0},
	bookmarks=true,         % show bookmarks bar?
	unicode=false,          % non-Latin characters in Acrobat’s bookmarks
	pdftoolbar=true,        % show Acrobat’s toolbar?
	pdfmenubar=true,        % show Acrobat’s menu?
	pdffitwindow=true,     % window fit to page when opened
	pdfstartview={FitH},    % fits the width of the page to the window
	pdftitle={Multimedia Datenbanken Übung},    % title
	pdfauthor={Huber Bastian},     % author
    pdfsubject={Übungsblatt},   % subject of the document
    pdfcreator={Huber Bastian},   % creator of the document
    pdfproducer={Huber Bastian}, % producer of the document
    pdfkeywords={Komplexitätstheorie, Passau}, % list of keywords
    pdfnewwindow=true,      % links in new window
    colorlinks=true,       % false: boxed links; true: colored links
    linkcolor=black,          % color of internal links
    citecolor=black,        % color of links to bibliography
    filecolor=magenta,      % color of file links
    urlcolor=DarkSkyBlue           % color of external links
}

% ------------------------------------------
% -------- listings
% ------------------------------------------
 
\lstset{
		breakautoindent=true,
		breakindent=2em,
		breaklines=true,
		tabsize=4,
		frame=blrt,
		frameround=tttt,
		captionpos=b,
		basicstyle=\scriptsize\ttfamily,
		keywordstyle={\color{SkyBlue}},
		%commentstyle={\color{OliveGreen}},
		stringstyle={\color{OliveGreen}},
		showspaces=false,
		%numbers=right,
		%numberstyle=\scriptsize,
		%stepnumber=1, 
		%numbersep=5pt,
		%showtabs=false
		prebreak = \raisebox{0ex}[0ex][0ex]{\ensuremath{\hookleftarrow}},
		aboveskip={1.5\baselineskip},
		columns=fixed,
		upquote=true,
		extendedchars=true
}
\fontsize{3mm}{4mm}\selectfont

% ------------------------------------------
% -------- misc
% ------------------------------------------
\newcommand{\bibliographyname}{Bibliography}
\setcounter{secnumdepth}{3}
\setcounter{tocdepth}{3}
\clubpenalty = 10000
\widowpenalty = 10000
\displaywidowpenalty = 10000
\setlength\fboxsep{6pt}
\setlength\fboxrule{1pt}
\renewcommand*\oldstylenums[1]{{\fontfamily{fxlj}\selectfont #1}}
%% Set table margins.
{\renewcommand{\arraystretch}{2}
\renewcommand{\tabcolsep}{0.4cm}}

% ------------------------------------------
% -------- hyphenation rules
% ------------------------------------------
\hyphenation{}

\author{Bastian Huber\\(51432) \and Sebastian Rainer\\(50882) \and Daniel Watzinger\\(51746) \and Benedikt Preis \\(?????)}
\title{\textbf{\huge{\thetitle}}\\\Large\textsc{Team Amazonen}\\\large\textsc{Gruppe ?}}
\date{\today}

\begin{document}

% Specify hyphenation rules.
\hyphenation{}

\maketitle

\pagestyle{fancy}

\section{Aufgabe 1}
\begin{description}
	\item[relational] Alle Informationen werden in Tabellen gespeichert. Dabei werden nur einfache Datentypen verwendet. Die Zeilen der Tabellen bestehen aus Tupeln. Die Struktur liegt in den Tabellen
	\item[objekt - relational] Alle Informationen werden in Tabellen mit komplexen Datentypen gespeichert. Diese Typen werden vom Benutzer erzeugt. Objekte sind Auspr�gungen dieser Typen. Tabellen k�nnen solche Objekte speichern. Die Struktur liegt in den Typen.
\end{description}
\section{Aufgabe 2}
\begin{description}
	\item[Benutzer definierter Datentyp] Datentyp, der vom Benutzer angelegt wird und bestimmte Informationen zusammenfasst und kapselt (Vergleiche Klassen in der Objektorientierung).
	\item[Vererbung] Ein Typ, der eine Spezialisierung eines Obertyps darstellt. Er erbt alle Attribute und Methoden.
	\item[Objekttabelle] Tabelle mit Objekten als Zeilen (im Gegensatz zur Tabellen mit Tupeln als Zeilen in relationalen Datenbanken).
	\item[Polymorphismus] �berschreibung von Methoden in der Typvererbung.
	\item[Collections] Menge von Objekten desselben Typs.
	\item[Dot-Operator] Zugriff auf Attribute in Objekttabellen �ber diesen Operator (z.B. a.thema.id ,  wobei thema ein Typ ist und id ein Attribut von thema).
	\item[OID] Object Identifier, ein systemweit eindeutiger Schl�ssel f�r Objekte.
	\item[REF und DEREF] Referenzen auf Objekte. Damit werden nicht die Objekte mit ihren Attributen als ganzes in den Tabellen gespeichert, sondern nur die Referenz auf das Objekt.
	\item[Beziehungen zwischen benutzer-definierten Datentypen (1:n, 1:1)] Beziehung zwischen Datentypen, die in einem ER oder UML Diagramm angegeben werden k�nnen (z.B. ein Auto hat mehrere Reifen, aber nur ein Farbe).
\end{description}

\section{Logisches objektrelationales DB-Schema}
\textit{
Gegeben sei folgender Ausschnitt eines konzeptuellen Schemas zur Verwaltung von Fussball
meisterschaften:
Bilden Sie diesen Schemaausschnitt auf ein logisches objektrelationales Datenbankschema in
Oracle ab1 . Die Methoden müssen dabei NICHT implementiert werden! Lösen Sie dabei fol
gende Teilaufgaben:}

\subsection{a)}
\textit{
	Definition von Objekttypen. Hierbei sind folgende Vorgaben zu berücksichtigen:
	\begin{itemize}
		\item Die Pfeile bei den Assoziationen geben die Navigationsrichtung an.
		\item Sämtliche Infos zu einer Teamaufstellung werden in einer eingebetteten Tabelle abgelegt. Die entsprechenden Spieler werden dabei referenziert.
		\item Die Teamzuordnungen eines Spielers werden ebenfalls in einer eingebetteten Tabelle abgelegt. Die entsprechenden Teams werden referenziert.
		\item Die Klasse Teilnehmer ist als abstrakte Klasse zu realisieren, da ihre Instanzmethode
	anzahlEinsaetze abstrakt ist.
		\item Das Attribut Position der Assoziationsklasse Aufstellungsinfo kann folgende Werte
	annehmen: T = Torwart, V = Verteidigung, M = Mittelfeld, S = Sturm.
		\item Die restlichen Assoziationen sind unter Verwendung von Referenztypen aufzuählen.
	\end{itemize}
}
\subsection{b)}
\textit{
Definition von Objekttabellen. Für jeden Objekttyp, der nicht eine eingebettete Tabelle,
ein Array variabler Länge oder ein eingebettetes Objekt darstellt, ist eine entsprechende
Objekttabelle zu definieren. Dabei sind Schlüsseleigenschaften, NOT-NULL Werte und
referentielle Integritäten von Attributen zu gewährleisten.}


\section{DML-Operationen auf Objekttabellen}
\textit{
	\begin{itemize}
		\item Fügen Sie in jeden Objektcontainer je nach Bedarf ein bis vier Beispielsobjekte ein.
		\item Aktualisieren Sie eine Aufstellungsinfo einer gegebenen Teamaufstellung.
	\end{itemize}
}
\section{Queries}
\textit{
	\begin{itemize}
		\item Alle Spiele (Nr, Datum + Ergebnis) wo die Heimmannschaft gewonnen hat.
		\item Vor-, Zuname und Trikotnummer von allen Spielern die jemals als Tormann gespielt ha-
		ben.
		\item Alle Objekte der Objekttabelle Teilnehmer, die Spieler sind, und die schon bei mindestens
		drei verschiedenen Teams unter Vertrag waren.
	\end{itemize}
}


\end{document}
