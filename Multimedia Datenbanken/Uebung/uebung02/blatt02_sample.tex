\documentclass{article}
\usepackage[T1]{fontenc}
\usepackage[ngerman,english]{babel}
\usepackage[latin1]{inputenc}
\usepackage{libertine,calc,microtype,parskip,lipsum,booktabs,textcomp,csquotes,enumerate,amssymb,vmargin,fancyhdr,fixltx2e,makeidx,listings,ellipsis,remreset,xcolor,lastpage,caption,fancybox,verbatim,amsmath}
\usepackage{graphicx}
\usepackage[pdftex]{hyperref}
\usepackage{amsmath}

% Title
\def\thetitle{Multimedia Datenbanken --- Blatt 02 --- Musterl�sung }

% ------------------------------------------
% -------- xcolor - (Tango)
% ------------------------------------------

\definecolor{LightButter}{rgb}{0.98,0.91,0.31}
\definecolor{LightOrange}{rgb}{0.98,0.68,0.24}
\definecolor{LightChocolate}{rgb}{0.91,0.72,0.43}
\definecolor{LightChameleon}{rgb}{0.54,0.88,0.20}
\definecolor{LightSkyBlue}{rgb}{0.45,0.62,0.81}
\definecolor{LightPlum}{rgb}{0.68,0.50,0.66}
\definecolor{LightScarletRed}{rgb}{0.93,0.16,0.16}
\definecolor{Butter}{rgb}{0.93,0.86,0.25}
\definecolor{Orange}{rgb}{0.96,0.47,0.00}
\definecolor{Chocolate}{rgb}{0.75,0.49,0.07}
\definecolor{Chameleon}{rgb}{0.45,0.82,0.09}
\definecolor{SkyBlue}{rgb}{0.20,0.39,0.64}
\definecolor{Plum}{rgb}{0.46,0.31,0.48}
\definecolor{ScarletRed}{rgb}{0.80,0.00,0.00}
\definecolor{DarkButter}{rgb}{0.77,0.62,0.00}
\definecolor{DarkOrange}{rgb}{0.80,0.36,0.00}
\definecolor{DarkChocolate}{rgb}{0.56,0.35,0.01}
\definecolor{DarkChameleon}{rgb}{0.30,0.60,0.02}
\definecolor{DarkSkyBlue}{rgb}{0.12,0.29,0.53}
\definecolor{DarkPlum}{rgb}{0.36,0.21,0.40}
\definecolor{DarkScarletRed}{rgb}{0.64,0.00,0.00}
\definecolor{Aluminium1}{rgb}{0.93,0.93,0.92}
\definecolor{Aluminium2}{rgb}{0.82,0.84,0.81}
\definecolor{Aluminium3}{rgb}{0.73,0.74,0.71}
\definecolor{Aluminium4}{rgb}{0.53,0.54,0.52}
\definecolor{Aluminium5}{rgb}{0.33,0.34,0.32}
\definecolor{Aluminium6}{rgb}{0.18,0.20,0.21}
\definecolor{Brown}{cmyk}{0,0.81,1,0.60}
\definecolor{OliveGreen}{cmyk}{0.64,0,0.95,0.40}
\definecolor{CadetBlue}{cmyk}{0.62,0.57,0.23,0}

% ------------------------------------------
% -------- vmargin
% ------------------------------------------

%\setmarginsrb{hleftmargini}{htopmargini}{hrightmargini}{hbottommargini}%{hheadheighti}{hheadsepi}{hfootheighti}{hfootskipi}
\setpapersize{A4}
\setmarginsrb{3cm}{1cm}{3cm}{1cm}{6mm}{7mm}{5mm}{15mm}

% ------------------------------------------
% -------- fancyhdr
% ------------------------------------------
%\fancyheadoffset[L]{\marginparsep+\marginparwidth}
\fancyhf{}
\fancyhead[L]{\bfseries{\nouppercase{\thetitle}}}
\fancyhead[R]{\bfseries{Seite \thepage\ von \pageref{LastPage}}}
\renewcommand{\headrulewidth}{0.5pt}
\renewcommand{\footrulewidth}{0pt}
\fancypagestyle{plain}{
\fancyhf{}
\fancyfoot[R]{\bfseries{Seite \thepage\ von \pageref{LastPage}}}
\renewcommand{\headrulewidth}{0pt}
\renewcommand{\footrulewidth}{0pt}
}

% ------------------------------------------
% -------- hyperref
% ------------------------------------------

\hypersetup{
	%breaklinks=true,
	pdfborder={0 0 0},
	bookmarks=true,         % show bookmarks bar?
	unicode=false,          % non-Latin characters in Acrobat’s bookmarks
	pdftoolbar=true,        % show Acrobat’s toolbar?
	pdfmenubar=true,        % show Acrobat’s menu?
	pdffitwindow=true,     % window fit to page when opened
	pdfstartview={FitH},    % fits the width of the page to the window
	pdftitle={Multimedia Datenbanken Übung},    % title
	pdfauthor={Huber Bastian},     % author
    pdfsubject={Übungsblatt},   % subject of the document
    pdfcreator={Huber Bastian},   % creator of the document
    pdfproducer={Huber Bastian}, % producer of the document
    pdfkeywords={Komplexitätstheorie, Passau}, % list of keywords
    pdfnewwindow=true,      % links in new window
    colorlinks=true,       % false: boxed links; true: colored links
    linkcolor=black,          % color of internal links
    citecolor=black,        % color of links to bibliography
    filecolor=magenta,      % color of file links
    urlcolor=DarkSkyBlue           % color of external links
}

% ------------------------------------------
% -------- listings
% ------------------------------------------
 
\lstset{
		breakautoindent=true,
		breakindent=2em,
		breaklines=true,
		tabsize=4,
		frame=blrt,
		frameround=tttt,
		captionpos=b,
		basicstyle=\scriptsize\ttfamily,
		keywordstyle={\color{SkyBlue}},
		%commentstyle={\color{OliveGreen}},
		stringstyle={\color{OliveGreen}},
		showspaces=false,
		%numbers=right,
		%numberstyle=\scriptsize,
		%stepnumber=1, 
		%numbersep=5pt,
		%showtabs=false
		prebreak = \raisebox{0ex}[0ex][0ex]{\ensuremath{\hookleftarrow}},
		aboveskip={1.5\baselineskip},
		columns=fixed,
		upquote=true,
		extendedchars=true
}
\fontsize{3mm}{4mm}\selectfont

% ------------------------------------------
% -------- misc
% ------------------------------------------
\newcommand{\bibliographyname}{Bibliography}
\setcounter{secnumdepth}{3}
\setcounter{tocdepth}{3}
\clubpenalty = 10000
\widowpenalty = 10000
\displaywidowpenalty = 10000
\setlength\fboxsep{6pt}
\setlength\fboxrule{1pt}
\renewcommand*\oldstylenums[1]{{\fontfamily{fxlj}\selectfont #1}}
%% Set table margins.
{\renewcommand{\arraystretch}{2}
\renewcommand{\tabcolsep}{0.4cm}}

% ------------------------------------------
% -------- hyphenation rules
% ------------------------------------------
\hyphenation{}

\author{Bastian Huber\\(51432) \and Sebastian Rainer\\(50882) \and Daniel Watzinger\\(51746) \and Benedikt Preis \\(?????)}
\title{\textbf{\huge{\thetitle}}\\\large\textsc{Gruppe 2}}
\date{\today}

\begin{document}

% Specify hyphenation rules.
\hyphenation{}

\maketitle

\pagestyle{fancy}

\section{Aufgabe 1}
	\begin{itemize}
		\item{RDB} \begin{itemize}
			\item vordefinierte Datentypen
			\item Schemen/Tabellen mit Attributen
		\end{itemize}
		\item{ORDB}
	\begin{itemize}
			\item Erweiterbarkeit anhand von OO - Mechanismen
			\item eigener, multimedialer Typ m�glich
			\item MM - Objekt als eigenst�ndiges Objekt in der Datenbank
		\end{itemize}
	\end{itemize}

Erweiterungen im ORDB - Modell:
	\begin{itemize}
		\item{Datenmodell} \begin{itemize}
			\item neue Objekte (MM - Objekte, geographische Objekte)
			\item Objekte mit komplexer Struktur
			\item gro�e Objekte
		\end{itemize}
		\item{Anfragesprache}
	\begin{itemize}
			\item spezielle Operationen (Finden �hnlicher Bilder)
			\item spezielle Pr�dikate (�berdeckt, liegt innerhalb)
		\end{itemize}
				\item{Indexstrukturen}
	\begin{itemize}
			\item neue Datentypen als Indexstrukturen
			\item Benutzer definiert Indexstrukturen
		\end{itemize}
						\item{Optimierung}
	\begin{itemize}
			\item Benutzer - definierte Kostenfunktionen f�r die neuen Datentypen
		\end{itemize}
	\end{itemize}
	
Vorteile ORDB:
\begin{itemize}
	\item Erweiterbarkeit (MM - Objekte)
	\item nat�rliche Modellierung
	\item Vererbung
	\item besseres Mapping zwischen Anwendung und Daten
\end{itemize}

Nachteile ORDB:
\begin{itemize}
	\item Performanz
	\item theoretische Fundierung nicht so gut
	\item Komplexit�t (benutzer - definierte Indexstrukturen, OO und relationale Kenntnisse erforderlich)
\end{itemize}


\section{Aufgabe 2}
	\begin{itemize}
		\item{Benutzer definierter Datentyp} \begin{itemize}
			\item Name, Attribute, Methoden
			\item Anweisung: CREATE TYPE
		\end{itemize}
		\item{Vererbung}
	\begin{itemize}
			\item Verweis auf Basistyp
			\item Anweisung: CREATE TYPE name UNDER obertyp
		\end{itemize}
				\item{Objekttabellen}
	\begin{itemize}
			\item Tabellen von Instanzen der neuen Datentypen
			\item Constraints und Verweise auf Typ
			\item Anweisung: CREATE TABLE name OF name\_typ
		\end{itemize}
						\item{Polymorphismus}
	\begin{itemize}
			\item Objekt kann als verschiedener Typ auftreten
		\end{itemize}
					\item{Beziehungs zwischen Datentypen}
	\begin{itemize}
			\item{1:1} Umsetzung mit REF
			\item{1:n} Umsetzung mit nested tables (Tabelle, die als Attribut in einem Typ benutzt wird)
		\end{itemize}
					\item{DOT - Operator}
	\begin{itemize}
			\item Navigieren im Schema f�r Anfragen
		\end{itemize}
					\item{REF}
	\begin{itemize}
			\item Beziehung zwischen zwei Objekten
		\end{itemize}
					\item{OID}
	\begin{itemize}
			\item eindeutiger Identifikator eines Objekts
			\item unabh�ngig vom Zustand des Objekts
		\end{itemize}
					\item{DEREF}
	\begin{itemize}
			\item Dereferenzieren
		\end{itemize}
	\end{itemize}
\section{Aufgabe 3}
Beispiell�sung wird noch hochgeladen von David.
\section{Aufgabe 4}
Beispiell�sung wird noch hochgeladen von David.
\section{Aufgabe 5}
Beispiell�sung wird noch hochgeladen von David.

\end{document}
