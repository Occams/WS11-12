\documentclass{article}
\usepackage[T1]{fontenc}
\usepackage[ngerman,english]{babel}
\usepackage[utf8]{inputenc} 
\usepackage{libertine,calc,microtype,parskip,lipsum,booktabs,textcomp,csquotes,enumerate,amssymb,vmargin,fancyhdr,fixltx2e,makeidx,listings,ellipsis,remreset,xcolor,lastpage,caption,fancybox,verbatim,amsmath}
\usepackage{graphicx}
\usepackage[pdftex]{hyperref}
\usepackage{amsmath,chngcntr}
\usepackage{amsmath}

% Title
\def\thetitle{Multimedia Datenbanken --- Blatt 02}

% ------------------------------------------
% -------- xcolor - (Tango)
% ------------------------------------------

\definecolor{LightButter}{rgb}{0.98,0.91,0.31}
\definecolor{LightOrange}{rgb}{0.98,0.68,0.24}
\definecolor{LightChocolate}{rgb}{0.91,0.72,0.43}
\definecolor{LightChameleon}{rgb}{0.54,0.88,0.20}
\definecolor{LightSkyBlue}{rgb}{0.45,0.62,0.81}
\definecolor{LightPlum}{rgb}{0.68,0.50,0.66}
\definecolor{LightScarletRed}{rgb}{0.93,0.16,0.16}
\definecolor{Butter}{rgb}{0.93,0.86,0.25}
\definecolor{Orange}{rgb}{0.96,0.47,0.00}
\definecolor{Chocolate}{rgb}{0.75,0.49,0.07}
\definecolor{Chameleon}{rgb}{0.45,0.82,0.09}
\definecolor{SkyBlue}{rgb}{0.20,0.39,0.64}
\definecolor{Plum}{rgb}{0.46,0.31,0.48}
\definecolor{ScarletRed}{rgb}{0.80,0.00,0.00}
\definecolor{DarkButter}{rgb}{0.77,0.62,0.00}
\definecolor{DarkOrange}{rgb}{0.80,0.36,0.00}
\definecolor{DarkChocolate}{rgb}{0.56,0.35,0.01}
\definecolor{DarkChameleon}{rgb}{0.30,0.60,0.02}
\definecolor{DarkSkyBlue}{rgb}{0.12,0.29,0.53}
\definecolor{DarkPlum}{rgb}{0.36,0.21,0.40}
\definecolor{DarkScarletRed}{rgb}{0.64,0.00,0.00}
\definecolor{Aluminium1}{rgb}{0.93,0.93,0.92}
\definecolor{Aluminium2}{rgb}{0.82,0.84,0.81}
\definecolor{Aluminium3}{rgb}{0.73,0.74,0.71}
\definecolor{Aluminium4}{rgb}{0.53,0.54,0.52}
\definecolor{Aluminium5}{rgb}{0.33,0.34,0.32}
\definecolor{Aluminium6}{rgb}{0.18,0.20,0.21}
\definecolor{Brown}{cmyk}{0,0.81,1,0.60}
\definecolor{OliveGreen}{cmyk}{0.64,0,0.95,0.40}
\definecolor{CadetBlue}{cmyk}{0.62,0.57,0.23,0}

% ------------------------------------------
% -------- vmargin
% ------------------------------------------

%\setmarginsrb{hleftmargini}{htopmargini}{hrightmargini}{hbottommargini}%{hheadheighti}{hheadsepi}{hfootheighti}{hfootskipi}
\setpapersize{A4}
\setmarginsrb{3cm}{1cm}{3cm}{1cm}{6mm}{7mm}{5mm}{15mm}

% ------------------------------------------
% -------- fancyhdr
% ------------------------------------------
%\fancyheadoffset[L]{\marginparsep+\marginparwidth}
\fancyhf{}
\fancyhead[L]{\bfseries{\nouppercase{\thetitle}}}
\fancyhead[R]{\bfseries{Seite \thepage\ von \pageref{LastPage}}}
\renewcommand{\headrulewidth}{0.5pt}
\renewcommand{\footrulewidth}{0pt}
\fancypagestyle{plain}{
\fancyhf{}
\fancyfoot[R]{\bfseries{Seite \thepage\ von \pageref{LastPage}}}
\renewcommand{\headrulewidth}{0pt}
\renewcommand{\footrulewidth}{0pt}
}

% ------------------------------------------
% -------- hyperref
% ------------------------------------------

\hypersetup{
	%breaklinks=true,
	pdfborder={0 0 0},
	bookmarks=true,         % show bookmarks bar?
	unicode=false,          % non-Latin characters in Acrobat’s bookmarks
	pdftoolbar=true,        % show Acrobat’s toolbar?
	pdfmenubar=true,        % show Acrobat’s menu?
	pdffitwindow=true,     % window fit to page when opened
	pdfstartview={FitH},    % fits the width of the page to the window
	pdftitle={Multimedia Datenbanken Übung},    % title
	pdfauthor={Huber Bastian},     % author
    pdfsubject={Übungsblatt},   % subject of the document
    pdfcreator={Huber Bastian},   % creator of the document
    pdfproducer={Huber Bastian}, % producer of the document
    pdfkeywords={Komplexitätstheorie, Passau}, % list of keywords
    pdfnewwindow=true,      % links in new window
    colorlinks=true,       % false: boxed links; true: colored links
    linkcolor=black,          % color of internal links
    citecolor=black,        % color of links to bibliography
    filecolor=magenta,      % color of file links
    urlcolor=DarkSkyBlue           % color of external links
}

% ------------------------------------------
% -------- listings
% ------------------------------------------
 
\lstset{
		breakautoindent=true,
		breakindent=2em,
		breaklines=true,
		tabsize=4,
		frame=blrt,
		frameround=tttt,
		captionpos=b,
		basicstyle=\scriptsize\ttfamily,
		keywordstyle={\color{SkyBlue}},
		%commentstyle={\color{OliveGreen}},
		stringstyle={\color{OliveGreen}},
		showspaces=false,
		%numbers=right,
		%numberstyle=\scriptsize,
		%stepnumber=1, 
		%numbersep=5pt,
		%showtabs=false
		prebreak = \raisebox{0ex}[0ex][0ex]{\ensuremath{\hookleftarrow}},
		aboveskip={1.5\baselineskip},
		columns=fixed,
		upquote=true,
		extendedchars=true
}
\fontsize{3mm}{4mm}\selectfont

% ------------------------------------------
% -------- misc
% ------------------------------------------
\newcommand{\bibliographyname}{Bibliography}
\setcounter{secnumdepth}{3}
\setcounter{tocdepth}{3}
\clubpenalty = 10000
\widowpenalty = 10000
\displaywidowpenalty = 10000
\setlength\fboxsep{6pt}
\setlength\fboxrule{1pt}
\renewcommand*\oldstylenums[1]{{\fontfamily{fxlj}\selectfont #1}}
%% Set table margins.
{\renewcommand{\arraystretch}{2}
\renewcommand{\tabcolsep}{0.4cm}}



% ------------------------------------------
% -------- hyphenation rules
% ------------------------------------------
\hyphenation{}

\author{Bastian Huber\\(51432) \and Sebastian Rainer\\(50882) \and Daniel Watzinger\\(51746) \and Benedikt Preis \\(?????)}
\title{\textbf{\huge{\thetitle}}\\\Large\textsc{Team Amazonen}\\\large\textsc{Gruppe ?}}
\date{\today}

\begin{document}

% Specify hyphenation rules.
\hyphenation{}

\maketitle

\pagestyle{fancy}

\section{Aufgabe 1: Projekt}
\section{Aufgabe 2: Lineares Filter}
\textit{Abbildung 1 beschreibt die Ausführung der Nachbarschaftsoperation Faltung mit einem quadratischen Kernel. Ein lineares Filter kann definiert werden, indem die Operation allen Pixeln eines Bildes angewendet wird.}

\subsection{}
\textit{Welche Probleme können durch die Anwendung der pixelbasierten Faltung an Randpixel des Bildes auftreten? Schlagen Sie mögliche Ansätze vor, um damit umzugehen!}

Für eine komponentenweise Multiplikation auf die Randpixel an bräuchte 
man die Farbwerte der Pixel die außerhalb des Bildes liegen. Da diese 
nicht existieren muss man sich eine Lösung für dieses Problem überlegen.
Eine davon wäre, für die nicht existenten Pixel, den Durchschnittswert der
vom Kernel umfassten Werte herzunehmen.

\subsection{}
\textit{Welches Problem kann auch bezüglich der Variationsbreite der Faltung Funktion auftreten? Wie kann damit umgegangen werden?}

\section{Aufgabe 3: Smoothing mit linearem Filter}
	\subsection{}
	\textit{Ein Smoothing-Filter kann anhand von einem linearen (Faltung) Filter mit einem
bestimmten Kernel implementiert werden. Berechnen Sie einen 5x5 Kernel zu diesem Zweck
für die ”Gleitender Durchschnitt” Variante des Smoothing-Filters : ein Pixel wird durch
den Durchschnitt der Werte seiner Nachbar ersetzt.}

$
\begin{bmatrix}
\frac{1}{8} & \frac{1}{8} & \frac{1}{8} \\
\frac{1}{8} & 0 & \frac{1}{8} \\
\frac{1}{8} & \frac{1}{8} & \frac{1}{8} \\
\end{bmatrix}
$
	
	\subsection{}
	
	\itshape Gewichtetes Smoothing: in diesem Fall werden höhere Koeffizienten für Pixel, die sich
näher am Zielpixel befinden. Die Koeffizienten können anhand von 2D-Funktionen berechnet werden, wie zum Beispiel:

		\begin{itemize}
			\item die Pyramide-Fläche Funktion:
			$$f (x, y) = -\alpha . max(|x|, |y|) + k $$
			$x$ und $y$ bezeichnen die Distanz zum Zielpixel in den x und y Achsen, $\alpha$ ist ein
			Parameter der Funktion und k eine Konstante, die addiert wird, um positive Werte
			zu bekommen.

			\item die konische Fläche Funktion:
				$$ f (x, y) = -\alpha . \sqrt{x^2 + y^2} + k $$
			
				Berechnen Sie die Koeffizienten eines 5x5 Kernel für diese 2 Fälle mit $\alpha = 2$. Suchen Sie
	dazu eine passende k aus, damit der geringste Koeffizient null ist. Runden Sie die Werte,
	um ganze Zahlen zu bekommen.
		\end{itemize}
	\normalfont
	
	\begin{itemize}
		\item \textbf{Pyramide-Fläche Funktion}

Mit $k = 4$ ergibt sich folgender Kernel:

$
\begin{bmatrix}
0 & 0 & 0 & 0 & 0 \\
0 & 2 & 2 & 2 & 0 \\
0 & 2 & 4 & 2 & 0 \\
0 & 2 & 2 & 2 & 0 \\
0 & 0 & 0 & 0 & 0 \\
\end{bmatrix}
$

		\item \textbf{Konische-Fläche Funktion}

Mit $k = 6$ ergibt sich folgender Kernel:

$
\begin{bmatrix}
0 & 2 & 2 & 2 & 0\\
2 & 3 & 4 & 3 & 2\\
2 & 4 & 6 & 4 & 2\\
2 & 3 & 4 & 3 & 2\\
0 & 2 & 2 & 2 & 0\\
\end{bmatrix}
$

	\end{itemize}
	
\section{Aufgabe 4: Implementierung}
\textit{Implementieren Sie die Smoothing-Funktion des Bildeditors:}

\subsection{Kernel Klasse (quadratischer Kernel von beliebiger Länge)}
\subsection{Hilfsmethoden für die Implementierung der in Aufgabe 1 definierten Ansätze}
\subsection{Pixelbasierte Faltung}
\subsection{Initialisierung der in Aufgabe 2 definierten Kernel}
\subsection{Einbindung des Filters im Model (Methode ”smootheImage”)}




\end{document}
