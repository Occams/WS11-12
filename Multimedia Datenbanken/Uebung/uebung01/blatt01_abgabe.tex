\documentclass{article}
\usepackage[T1]{fontenc}
\usepackage[ngerman,english]{babel}
\usepackage[utf8]{inputenc}
\usepackage{libertine,calc,microtype,parskip,lipsum,booktabs,textcomp,csquotes,enumerate,amssymb,vmargin,fancyhdr,fixltx2e,makeidx,listings,ellipsis,remreset,xcolor,lastpage,caption,fancybox,verbatim,amsmath}
\usepackage{graphicx}
\usepackage[pdftex]{hyperref}
\usepackage{amsmath}

% Title
\def\thetitle{Multimedia Datenbanken --- Blatt 01}

% ------------------------------------------
% -------- xcolor - (Tango)
% ------------------------------------------

\definecolor{LightButter}{rgb}{0.98,0.91,0.31}
\definecolor{LightOrange}{rgb}{0.98,0.68,0.24}
\definecolor{LightChocolate}{rgb}{0.91,0.72,0.43}
\definecolor{LightChameleon}{rgb}{0.54,0.88,0.20}
\definecolor{LightSkyBlue}{rgb}{0.45,0.62,0.81}
\definecolor{LightPlum}{rgb}{0.68,0.50,0.66}
\definecolor{LightScarletRed}{rgb}{0.93,0.16,0.16}
\definecolor{Butter}{rgb}{0.93,0.86,0.25}
\definecolor{Orange}{rgb}{0.96,0.47,0.00}
\definecolor{Chocolate}{rgb}{0.75,0.49,0.07}
\definecolor{Chameleon}{rgb}{0.45,0.82,0.09}
\definecolor{SkyBlue}{rgb}{0.20,0.39,0.64}
\definecolor{Plum}{rgb}{0.46,0.31,0.48}
\definecolor{ScarletRed}{rgb}{0.80,0.00,0.00}
\definecolor{DarkButter}{rgb}{0.77,0.62,0.00}
\definecolor{DarkOrange}{rgb}{0.80,0.36,0.00}
\definecolor{DarkChocolate}{rgb}{0.56,0.35,0.01}
\definecolor{DarkChameleon}{rgb}{0.30,0.60,0.02}
\definecolor{DarkSkyBlue}{rgb}{0.12,0.29,0.53}
\definecolor{DarkPlum}{rgb}{0.36,0.21,0.40}
\definecolor{DarkScarletRed}{rgb}{0.64,0.00,0.00}
\definecolor{Aluminium1}{rgb}{0.93,0.93,0.92}
\definecolor{Aluminium2}{rgb}{0.82,0.84,0.81}
\definecolor{Aluminium3}{rgb}{0.73,0.74,0.71}
\definecolor{Aluminium4}{rgb}{0.53,0.54,0.52}
\definecolor{Aluminium5}{rgb}{0.33,0.34,0.32}
\definecolor{Aluminium6}{rgb}{0.18,0.20,0.21}
\definecolor{Brown}{cmyk}{0,0.81,1,0.60}
\definecolor{OliveGreen}{cmyk}{0.64,0,0.95,0.40}
\definecolor{CadetBlue}{cmyk}{0.62,0.57,0.23,0}

% ------------------------------------------
% -------- vmargin
% ------------------------------------------

%\setmarginsrb{hleftmargini}{htopmargini}{hrightmargini}{hbottommargini}%{hheadheighti}{hheadsepi}{hfootheighti}{hfootskipi}
\setpapersize{A4}
\setmarginsrb{3cm}{1cm}{3cm}{1cm}{6mm}{7mm}{5mm}{15mm}

% ------------------------------------------
% -------- fancyhdr
% ------------------------------------------
%\fancyheadoffset[L]{\marginparsep+\marginparwidth}
\fancyhf{}
\fancyhead[L]{\bfseries{\nouppercase{\thetitle}}}
\fancyhead[R]{\bfseries{Seite \thepage\ von \pageref{LastPage}}}
\renewcommand{\headrulewidth}{0.5pt}
\renewcommand{\footrulewidth}{0pt}
\fancypagestyle{plain}{
\fancyhf{}
\fancyfoot[R]{\bfseries{Seite \thepage\ von \pageref{LastPage}}}
\renewcommand{\headrulewidth}{0pt}
\renewcommand{\footrulewidth}{0pt}
}

% ------------------------------------------
% -------- hyperref
% ------------------------------------------

\hypersetup{
	%breaklinks=true,
	pdfborder={0 0 0},
	bookmarks=true,         % show bookmarks bar?
	unicode=false,          % non-Latin characters in Acrobat’s bookmarks
	pdftoolbar=true,        % show Acrobat’s toolbar?
	pdfmenubar=true,        % show Acrobat’s menu?
	pdffitwindow=true,     % window fit to page when opened
	pdfstartview={FitH},    % fits the width of the page to the window
	pdftitle={Multimedia Datenbanken Übung},    % title
	pdfauthor={Huber Bastian},     % author
    pdfsubject={Übungsblatt},   % subject of the document
    pdfcreator={Huber Bastian},   % creator of the document
    pdfproducer={Huber Bastian}, % producer of the document
    pdfkeywords={Komplexitätstheorie, Passau}, % list of keywords
    pdfnewwindow=true,      % links in new window
    colorlinks=true,       % false: boxed links; true: colored links
    linkcolor=black,          % color of internal links
    citecolor=black,        % color of links to bibliography
    filecolor=magenta,      % color of file links
    urlcolor=DarkSkyBlue           % color of external links
}

% ------------------------------------------
% -------- listings
% ------------------------------------------
 
\lstset{
		breakautoindent=true,
		breakindent=2em,
		breaklines=true,
		tabsize=4,
		frame=blrt,
		frameround=tttt,
		captionpos=b,
		basicstyle=\scriptsize\ttfamily,
		keywordstyle={\color{SkyBlue}},
		%commentstyle={\color{OliveGreen}},
		stringstyle={\color{OliveGreen}},
		showspaces=false,
		%numbers=right,
		%numberstyle=\scriptsize,
		%stepnumber=1, 
		%numbersep=5pt,
		%showtabs=false
		prebreak = \raisebox{0ex}[0ex][0ex]{\ensuremath{\hookleftarrow}},
		aboveskip={1.5\baselineskip},
		columns=fixed,
		upquote=true,
		extendedchars=true
}
\fontsize{3mm}{4mm}\selectfont

% ------------------------------------------
% -------- misc
% ------------------------------------------
\newcommand{\bibliographyname}{Bibliography}
\setcounter{secnumdepth}{3}
\setcounter{tocdepth}{3}
\clubpenalty = 10000
\widowpenalty = 10000
\displaywidowpenalty = 10000
\setlength\fboxsep{6pt}
\setlength\fboxrule{1pt}
\renewcommand*\oldstylenums[1]{{\fontfamily{fxlj}\selectfont #1}}
%% Set table margins.
{\renewcommand{\arraystretch}{2}
\renewcommand{\tabcolsep}{0.4cm}}

% ------------------------------------------
% -------- hyphenation rules
% ------------------------------------------
\hyphenation{}

\author{Bastian Huber\\(51432) \and Sebastian Rainer\\(50882) \and Daniel Watzinger\\(51746) \and Benedikt Preis \\(?????)}
\title{\textbf{\huge{\thetitle}}\\\Large\textsc{Team Amazonen}\\\large\textsc{Gruppe ?}}
\date{\today}

\begin{document}

% Specify hyphenation rules.
\hyphenation{}

\maketitle

\pagestyle{fancy}

\section{Mediendatentypen}
\textit{Nennen Sie die wichtigsten Mediendatentypen und geben Sie deren interne Repräsentationsform an. Welche Eigenschaften können den jeweiligen Mediendatentypen zugeordnet werden? Zeigen Sie die Eigenschaften jeweils an einem kleinen Beispiel!}
\begin{description}
	\item[Text] UTF-8, ASCII usw. Intern also als Folgen von Bits. \textbf{Diskret, zeitunabhängig}
	\item[Bild] Pixelgrafik: RGB Farbwert pro Pixel, braucht viel Speicher, kann mittels Kompressionsverfahren komprimiert werden. Vektorgrafik: Mathematische Beschreibung von Kurven und Geometrischer Formen, beliebig skalierbar, Qualität = Qualität des Ausgabemediums. \textbf{Diskret, zeitunabhängig}
	\item[Video] Abfolge von Bildern, Frames per Second > 25, Komprimierung durch Prediction möglich. \textbf{Kontinuierlich}
	\item[Audio] Audio als Abfolge von Bits durch PCM Codierung möglich, Komprimierung durch Ausfiltern von Frequenzen die der Mensch nur marginal oder gar nicht wahrnimmt. \textbf{Kontinuierlich}
\end{description}

\section{Strukturierte/Unstruktuierte Daten}
\textit{Worin liegt der Unterschied zwischen strukturierten und unstrukturierten Daten? Geben Sie jeweils Beispiele! Welche Auswirkungen haben solche Daten auf eine Datenbank und deren Abfrageeigenschaften?}


	\begin{description}
		\item[strukturierte Daten] klar definierte Daten mit Struktur, die mit klaren, boolschen Anfragen genau durchsucht werden können. Zum Beispiel Tabellen mit Studienleistungen.
		\item[unstrukturierte Daten] ungenau definierte Daten, die durch ungenaue Anfragen durchsucht werden (Ähnlichkeitssuche, Fuzzy Retrieval). Zum Beispiel eine Suche im Web. 
	\end{description}
	
	Auswirkungen auf Datenbank und Abfrageeigenschaften:
	\begin{itemize}
		\item Integration beider Datentypen
		\item Sowohl exakte Suche als auch Retrieval nach unstrukturierten Daten in einer Anfrage. Somit stellt sich die Frage nach der Anfrageauswertung
		\item Speicherung in der Datenbank sollte möglichst effizient sein. Dies ist vorallem bei unstrukturierten Daten nicht trivial.
	\end{itemize}

\section{Bestandteile MMDB}
\textit{Worin unterscheidet sich eine Multimedia DB von einer herkömmlichen Datenbanken? Gehen Sie hier insbesondere auf die einzelnen Bestandteile einer Datenbank ein (z.B.: Indizierung, Datenmodell, Abfragesprachen, etc.) !}

\begin{itemize}
	\item speichert MM- Daten (zum Teil unstrukturiert)
	\item Abfrage über verschiedene Suchmechanismen (Content Based Retrieval)
	\item Indexierung schwieriger, da keine numerischen Werte vorhanden und Länge von unstrukturierten Daten nicht vorhersagbar.
	\item Keine numerischen Werte, nur MM - Objekte im Datenmodell
	\item Integration strukturierter und unstrukturierter Daten
\end{itemize}


\section{MMDBMS-Abfragen}

\textit{Welche Arten von MMDBMS-Abfragen kennen Sie? Geben Sie jeweils ein Beispiel an! Was versteht man unter Content-Based Retrieval? In welchem Zusammenhang stehen hier die englischen Abkürzungen QBE, QBS, QBH?}


\begin{description}
	\item[Browsing] Navigieren über den Datenbestand (Topic Maps notwendig)
	\item[Attribut-Prädikate] Abfrage über gewisse, feste Attribute der Objekte
	\item[Struktur-Prädikate] Abfrage über Prädikate der zeitlichen Abfolge in den MM-Objekten
	\item[Räumliche-Prädikate] Abfrage über Prädikate des räumlichen Aufbaus in den MM-Objekten
	\item[Semantische-Prädikate] Abfrage über Prädikate die die Semantik der MM-Objekte beschreiben
	\item[Content-Based Retrieval] Objekte werden aufgrund ihres Inhalts durchsucht, etwa Farbe, Farbverteilung oder Ähnliches
	\item[QBE] Query-by-Example, Ähnlichkeitssuche anhand eines Beispielobjekts
	\item[QBS] Query-by-Sketching, Ähnlichkeitssuche anhand einer Skizze
	\item[QBH] Query-by-humming, Ähnlichkeitssuche anhand eines Audioobjekts(Summen)
\end{description}


\end{document}
