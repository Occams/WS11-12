\documentclass{article}
\usepackage[T1]{fontenc}
\usepackage[ngerman,english]{babel}
\usepackage[latin1]{inputenc}
\usepackage{libertine,calc,microtype,parskip,lipsum,booktabs,textcomp,csquotes,enumerate,amssymb,vmargin,fancyhdr,fixltx2e,makeidx,listings,ellipsis,remreset,xcolor,lastpage,caption,fancybox,verbatim,amsmath}
\usepackage{graphicx}
\usepackage[pdftex]{hyperref}
\usepackage{amsmath}

% Title
\def\thetitle{Multimedia Datenbanken --- Blatt 07 --- Musterl�sung }

% ------------------------------------------
% -------- xcolor - (Tango)
% ------------------------------------------

\definecolor{LightButter}{rgb}{0.98,0.91,0.31}
\definecolor{LightOrange}{rgb}{0.98,0.68,0.24}
\definecolor{LightChocolate}{rgb}{0.91,0.72,0.43}
\definecolor{LightChameleon}{rgb}{0.54,0.88,0.20}
\definecolor{LightSkyBlue}{rgb}{0.45,0.62,0.81}
\definecolor{LightPlum}{rgb}{0.68,0.50,0.66}
\definecolor{LightScarletRed}{rgb}{0.93,0.16,0.16}
\definecolor{Butter}{rgb}{0.93,0.86,0.25}
\definecolor{Orange}{rgb}{0.96,0.47,0.00}
\definecolor{Chocolate}{rgb}{0.75,0.49,0.07}
\definecolor{Chameleon}{rgb}{0.45,0.82,0.09}
\definecolor{SkyBlue}{rgb}{0.20,0.39,0.64}
\definecolor{Plum}{rgb}{0.46,0.31,0.48}
\definecolor{ScarletRed}{rgb}{0.80,0.00,0.00}
\definecolor{DarkButter}{rgb}{0.77,0.62,0.00}
\definecolor{DarkOrange}{rgb}{0.80,0.36,0.00}
\definecolor{DarkChocolate}{rgb}{0.56,0.35,0.01}
\definecolor{DarkChameleon}{rgb}{0.30,0.60,0.02}
\definecolor{DarkSkyBlue}{rgb}{0.12,0.29,0.53}
\definecolor{DarkPlum}{rgb}{0.36,0.21,0.40}
\definecolor{DarkScarletRed}{rgb}{0.64,0.00,0.00}
\definecolor{Aluminium1}{rgb}{0.93,0.93,0.92}
\definecolor{Aluminium2}{rgb}{0.82,0.84,0.81}
\definecolor{Aluminium3}{rgb}{0.73,0.74,0.71}
\definecolor{Aluminium4}{rgb}{0.53,0.54,0.52}
\definecolor{Aluminium5}{rgb}{0.33,0.34,0.32}
\definecolor{Aluminium6}{rgb}{0.18,0.20,0.21}
\definecolor{Brown}{cmyk}{0,0.81,1,0.60}
\definecolor{OliveGreen}{cmyk}{0.64,0,0.95,0.40}
\definecolor{CadetBlue}{cmyk}{0.62,0.57,0.23,0}

% ------------------------------------------
% -------- vmargin
% ------------------------------------------

%\setmarginsrb{hleftmargini}{htopmargini}{hrightmargini}{hbottommargini}%{hheadheighti}{hheadsepi}{hfootheighti}{hfootskipi}
\setpapersize{A4}
\setmarginsrb{3cm}{1cm}{3cm}{1cm}{6mm}{7mm}{5mm}{15mm}

% ------------------------------------------
% -------- fancyhdr
% ------------------------------------------
%\fancyheadoffset[L]{\marginparsep+\marginparwidth}
\fancyhf{}
\fancyhead[L]{\bfseries{\nouppercase{\thetitle}}}
\fancyhead[R]{\bfseries{Seite \thepage\ von \pageref{LastPage}}}
\renewcommand{\headrulewidth}{0.5pt}
\renewcommand{\footrulewidth}{0pt}
\fancypagestyle{plain}{
\fancyhf{}
\fancyfoot[R]{\bfseries{Seite \thepage\ von \pageref{LastPage}}}
\renewcommand{\headrulewidth}{0pt}
\renewcommand{\footrulewidth}{0pt}
}

% ------------------------------------------
% -------- hyperref
% ------------------------------------------

\hypersetup{
	%breaklinks=true,
	pdfborder={0 0 0},
	bookmarks=true,         % show bookmarks bar?
	unicode=false,          % non-Latin characters in Acrobat’s bookmarks
	pdftoolbar=true,        % show Acrobat’s toolbar?
	pdfmenubar=true,        % show Acrobat’s menu?
	pdffitwindow=true,     % window fit to page when opened
	pdfstartview={FitH},    % fits the width of the page to the window
	pdftitle={Multimedia Datenbanken Übung},    % title
	pdfauthor={Huber Bastian},     % author
    pdfsubject={Übungsblatt},   % subject of the document
    pdfcreator={Huber Bastian},   % creator of the document
    pdfproducer={Huber Bastian}, % producer of the document
    pdfkeywords={Komplexitätstheorie, Passau}, % list of keywords
    pdfnewwindow=true,      % links in new window
    colorlinks=true,       % false: boxed links; true: colored links
    linkcolor=black,          % color of internal links
    citecolor=black,        % color of links to bibliography
    filecolor=magenta,      % color of file links
    urlcolor=DarkSkyBlue           % color of external links
}

% ------------------------------------------
% -------- listings
% ------------------------------------------
 
\lstset{
		breakautoindent=true,
		breakindent=2em,
		breaklines=true,
		tabsize=4,
		frame=blrt,
		frameround=tttt,
		captionpos=b,
		basicstyle=\scriptsize\ttfamily,
		keywordstyle={\color{SkyBlue}},
		%commentstyle={\color{OliveGreen}},
		stringstyle={\color{OliveGreen}},
		showspaces=false,
		%numbers=right,
		%numberstyle=\scriptsize,
		%stepnumber=1, 
		%numbersep=5pt,
		%showtabs=false
		prebreak = \raisebox{0ex}[0ex][0ex]{\ensuremath{\hookleftarrow}},
		aboveskip={1.5\baselineskip},
		columns=fixed,
		upquote=true,
		extendedchars=true
}
\fontsize{3mm}{4mm}\selectfont

% ------------------------------------------
% -------- misc
% ------------------------------------------
\newcommand{\bibliographyname}{Bibliography}
\setcounter{secnumdepth}{3}
\setcounter{tocdepth}{3}
\clubpenalty = 10000
\widowpenalty = 10000
\displaywidowpenalty = 10000
\setlength\fboxsep{6pt}
\setlength\fboxrule{1pt}
\renewcommand*\oldstylenums[1]{{\fontfamily{fxlj}\selectfont #1}}
%% Set table margins.
{\renewcommand{\arraystretch}{2}
\renewcommand{\tabcolsep}{0.4cm}}

% ------------------------------------------
% -------- hyphenation rules
% ------------------------------------------
\hyphenation{}

\author{Bastian Huber\\(51432) \and Sebastian Rainer\\(50882) \and Daniel Watzinger\\(51746) \and Benedikt Preis \\(?????)}
\title{\textbf{\huge{\thetitle}}\\\large\textsc{Gruppe 2}}
\date{\today}

\begin{document}

% Specify hyphenation rules.
\hyphenation{}

\maketitle

\pagestyle{fancy}

\section{Aufgabe 1}
	\begin{itemize}
		\item{Content Based Image Retrieval} \begin{itemize}
			\item inhaltsbasierte Suche
			\item MM-Suche basierend auf Repr�sentation der Inhalte(anhand von Merkmalen, die automatisch berechnet werden)
		\end{itemize}
		\item{Ablauf einer Suche}
	\begin{itemize}
			\item Berechnung der Merkmale
			\item �hnlichkeitsma� definieren (Distanzfunktion)
			\item Anfragen anhand des �hnlichkeitsma�es bewerten
		\end{itemize}
			\item{Kategorien von Merkmalen}
	\begin{itemize}
			\item Farbe
			\begin{itemize}
	\item Farbhistogramm
	\item z.B. Graubild: Farbe $\in$ [0,255]
	\item Aufteilung des Farbraumes in Intervalle
	\item Anzahl von Pixeln pro Intervall
	\item Problem: keine r�umliche Information ($\rightarrow$ Histogramm von Regionen)
\end{itemize}
			\item Textur
			\begin{itemize}
	\item Bewertung von Bildern mit �hnlichem Hintergrund
	\item Unterschied zwischen Regionen mit �hnlichen Farben �ber
	\begin{itemize}
	\item Kontrast
	\item Rauigkeit
	\item Regularit�t
	\item Periodizit�t
\end{itemize}
\end{itemize}
\item Form
\begin{itemize}
	\item Erkennung von Objekten im Bild
	\item Kantenbasiert
	\item Regionsbasiert
\end{itemize}
		\end{itemize}
			\item{CBR Bestandteile} (siehe auch Kapitel 7, Folie 46)
	\begin{itemize}
			\item Merkmalsextraktion: Extrahiert Merkmale aus MM-Daten zur Erzeugung der Datenbank
			\item MM-Anfragesystem: Liefert eine Suchanfrage
			\item �hnlichkeitsmetrik: Berechnet Distanz zwischen Anfrage und m�glichen Treffern in der Datenbank (Anfrageverarbeitung)
			\item Datenbank: H�lt MM-Daten mit Merkmalen
			\item MM-Pr�sentation: Die aufbereiteten Suchergebnisse
			\item keine exakte Suche, sondern Berechnung von �hnlichkeiten (Relevance Feedback m�glich)
		\end{itemize}
							\item{Feature Vektor}
	\begin{itemize}
			\item repr�sentiert MM-Objekt
			\item n-dimensionaler Vektor, jede Dimension entspricht einem Merkmal
			\item Merkmale als numerische Werte (k�nnen aber auch selbst wieder Vektoren sein)
			\item Feature Vektoren spannen den Merkmalsraum auf 
		\end{itemize}		
		\item{Probleme bei Indizierung}
		\begin{itemize}
	\item Problem durch hohe Anzahl von Dimensionen der Feature Vektoren (ab 15 Dimensionen Ineffizient)
	\item $\rightarrow$ feature selection oder feature reduction (Erkl�rung siehe Aufgabe 2)
\end{itemize}
	\end{itemize}


\section{Aufgabe 2}
	\begin{itemize}
		\item {Dominant Color}
			\begin{itemize}
				\item identifiziere h�ufigste Farben pro Bild
				\item Clustering auf Faben des Bildes (mit Generalized Lloyd Algorithmus)
				\item maximal 8 Cluster pro Bild
				\item berechne pro Cluster $i$ den Centroid $c_i$(Punkt, der am n�chsten am Durchschnittspunkt des Clusters liegt)
				\item berechne Prozentsatz des Clusters $i$ am Bild: $p_i = \frac{Pixel im Cluster}{Pixel im Bild}$
				\item berechne Varianz der Pixel im Cluster $i$ $v_i$
				\item Dann gilt f�r die dominante Farbe $D_f = (c_i , p_i , v_i)$ f�r jeden Cluster $i$
				\item Problem: keine r�umliche Information
			\end{itemize}
		\item {Spatial Coherency}
			\begin{itemize}
				\item Einbeziehung der r�umlichen Information
				\item Berechnung der SC f�r jeden Pixel $\in D_f$ mit SC(Pixel) = Anzahl der Nachbarpixel mit derselben $D_f$
				\item SC($D_f$) = AVG(SC(Pixel)) f�r alle Pixel $\in D_f$
				\item SC(Bild = AVG(SC($D_f$))) f�r alle $D_f$ des Bildes
				\item DCD = \{\{($c_i , p_i , v_i$)$^8$\}, SC(Bild)\}
			\end{itemize}
			\item{Distanz Metriken}
			\begin{itemize}
	\item $\delta$ : $\mathbb{R}^n \times \mathbb{R}^n \rightarrow \mathbb{R}$
	\item Berechnung der Distanz zwischen zwei n dimensionalen Feature Vektoren
	\item $\delta_{Euklid} = \sqrt{\sum\limits_{i=1}^{n}{(p_i \cdot q_i)^2}}$
	\item $\delta_{Manhattan} = \sum\limits_{i=1}^{n}{|p_i \cdot q_i|}$
	\item $\delta_{MAX} = MAX(p_i - q_i)$
\end{itemize}
\item{Curse of Dimensionality}
\begin{itemize}
	\item sequentielle Suche ineffizient $\rightarrow$ Indizierung
	\item Ist die Anzahl der Dimensionen der Feature Vektoren gr��er als 15, so sind alle Indizierungsverfahren langsamer als sequentielle Suche $\rightarrow$ Reduktion der Dimensionen
	\item feature selection : selektiere die wichtigsten Merkmale und betrachte nur diese
	\item feature reduction : identifiziere Abh�ngigkeiten zwischen den Merkmalen und kombiniere diese
\end{itemize}
\item{Abfragetypen}
\begin{itemize}
	\item Query-by-Feature: Suche nach genau definierten Featurewerten
	\item Query-by-example: Sucha nach �hnlichen Bildern zu einem Beispielbild (anhand der Distanzmetrik)
	\item k-NN search: finde die k besten Treffer
	\item Range Query: Threshold f�r Distanz und finde alle Treffer, deren Distanz kleiner als der Threshold ist
\end{itemize}
	\end{itemize}
\section{Aufgabe 3}
\begin{itemize}
	\item semantic gap: es gibt kein Feature f�r 'rotes Auto'
	\item QBF: selektiere rote Farben und Formen, die wie ein Auto aussehen
	\item QBE: Bild mit rotem Auto $\rightarrow$ hohe Precision (alle �hnlichen Bilder sollten rote Autos enthalten), aber niedriger Recall (es werden nur Bilder gefunden die genau zu dieser Form von Auto passen)
\end{itemize}

\end{document}
