\documentclass{article}
\usepackage[T1]{fontenc}
\usepackage[ngerman,english]{babel}
\usepackage[latin1]{inputenc}
\usepackage{libertine,calc,microtype,parskip,lipsum,booktabs,textcomp,csquotes,enumerate,amssymb,vmargin,fancyhdr,fixltx2e,makeidx,listings,ellipsis,remreset,xcolor,lastpage,caption,fancybox,verbatim,amsmath}
\usepackage{graphicx}
\usepackage[pdftex]{hyperref}
\usepackage{amsmath,chngcntr}

% Title
\def\thetitle{Multimedia Datenbanken --- Blatt 02}

% ------------------------------------------
% -------- xcolor - (Tango)
% ------------------------------------------

\definecolor{LightButter}{rgb}{0.98,0.91,0.31}
\definecolor{LightOrange}{rgb}{0.98,0.68,0.24}
\definecolor{LightChocolate}{rgb}{0.91,0.72,0.43}
\definecolor{LightChameleon}{rgb}{0.54,0.88,0.20}
\definecolor{LightSkyBlue}{rgb}{0.45,0.62,0.81}
\definecolor{LightPlum}{rgb}{0.68,0.50,0.66}
\definecolor{LightScarletRed}{rgb}{0.93,0.16,0.16}
\definecolor{Butter}{rgb}{0.93,0.86,0.25}
\definecolor{Orange}{rgb}{0.96,0.47,0.00}
\definecolor{Chocolate}{rgb}{0.75,0.49,0.07}
\definecolor{Chameleon}{rgb}{0.45,0.82,0.09}
\definecolor{SkyBlue}{rgb}{0.20,0.39,0.64}
\definecolor{Plum}{rgb}{0.46,0.31,0.48}
\definecolor{ScarletRed}{rgb}{0.80,0.00,0.00}
\definecolor{DarkButter}{rgb}{0.77,0.62,0.00}
\definecolor{DarkOrange}{rgb}{0.80,0.36,0.00}
\definecolor{DarkChocolate}{rgb}{0.56,0.35,0.01}
\definecolor{DarkChameleon}{rgb}{0.30,0.60,0.02}
\definecolor{DarkSkyBlue}{rgb}{0.12,0.29,0.53}
\definecolor{DarkPlum}{rgb}{0.36,0.21,0.40}
\definecolor{DarkScarletRed}{rgb}{0.64,0.00,0.00}
\definecolor{Aluminium1}{rgb}{0.93,0.93,0.92}
\definecolor{Aluminium2}{rgb}{0.82,0.84,0.81}
\definecolor{Aluminium3}{rgb}{0.73,0.74,0.71}
\definecolor{Aluminium4}{rgb}{0.53,0.54,0.52}
\definecolor{Aluminium5}{rgb}{0.33,0.34,0.32}
\definecolor{Aluminium6}{rgb}{0.18,0.20,0.21}
\definecolor{Brown}{cmyk}{0,0.81,1,0.60}
\definecolor{OliveGreen}{cmyk}{0.64,0,0.95,0.40}
\definecolor{CadetBlue}{cmyk}{0.62,0.57,0.23,0}

% ------------------------------------------
% -------- vmargin
% ------------------------------------------

%\setmarginsrb{hleftmargini}{htopmargini}{hrightmargini}{hbottommargini}%{hheadheighti}{hheadsepi}{hfootheighti}{hfootskipi}
\setpapersize{A4}
\setmarginsrb{3cm}{1cm}{3cm}{1cm}{6mm}{7mm}{5mm}{15mm}

% ------------------------------------------
% -------- fancyhdr
% ------------------------------------------
%\fancyheadoffset[L]{\marginparsep+\marginparwidth}
\fancyhf{}
\fancyhead[L]{\bfseries{\nouppercase{\thetitle}}}
\fancyhead[R]{\bfseries{Seite \thepage\ von \pageref{LastPage}}}
\renewcommand{\headrulewidth}{0.5pt}
\renewcommand{\footrulewidth}{0pt}
\fancypagestyle{plain}{
\fancyhf{}
\fancyfoot[R]{\bfseries{Seite \thepage\ von \pageref{LastPage}}}
\renewcommand{\headrulewidth}{0pt}
\renewcommand{\footrulewidth}{0pt}
}

% ------------------------------------------
% -------- hyperref
% ------------------------------------------

\hypersetup{
	%breaklinks=true,
	pdfborder={0 0 0},
	bookmarks=true,         % show bookmarks bar?
	unicode=false,          % non-Latin characters in Acrobat’s bookmarks
	pdftoolbar=true,        % show Acrobat’s toolbar?
	pdfmenubar=true,        % show Acrobat’s menu?
	pdffitwindow=true,     % window fit to page when opened
	pdfstartview={FitH},    % fits the width of the page to the window
	pdftitle={Multimedia Datenbanken Übung},    % title
	pdfauthor={Huber Bastian},     % author
    pdfsubject={Übungsblatt},   % subject of the document
    pdfcreator={Huber Bastian},   % creator of the document
    pdfproducer={Huber Bastian}, % producer of the document
    pdfkeywords={Komplexit�tstheorie, Passau}, % list of keywords
    pdfnewwindow=true,      % links in new window
    colorlinks=true,       % false: boxed links; true: colored links
    linkcolor=black,          % color of internal links
    citecolor=black,        % color of links to bibliography
    filecolor=magenta,      % color of file links
    urlcolor=DarkSkyBlue           % color of external links
}

% ------------------------------------------
% -------- listings
% ------------------------------------------
 
\lstset{
		breakautoindent=true,
		breakindent=2em,
		breaklines=true,
		tabsize=4,
		frame=blrt,
		frameround=tttt,
		captionpos=b,
		basicstyle=\scriptsize\ttfamily,
		keywordstyle={\color{SkyBlue}},
		%commentstyle={\color{OliveGreen}},
		stringstyle={\color{OliveGreen}},
		showspaces=false,
		%numbers=right,
		%numberstyle=\scriptsize,
		%stepnumber=1, 
		%numbersep=5pt,
		%showtabs=false
		prebreak = \raisebox{0ex}[0ex][0ex]{\ensuremath{\hookleftarrow}},
		aboveskip={1.5\baselineskip},
		columns=fixed,
		upquote=true,
		extendedchars=true
}
\fontsize{3mm}{4mm}\selectfont

% ------------------------------------------
% -------- misc
% ------------------------------------------
\newcommand{\bibliographyname}{Bibliography}
\setcounter{secnumdepth}{3}
\setcounter{tocdepth}{3}
\clubpenalty = 10000
\widowpenalty = 10000
\displaywidowpenalty = 10000
\setlength\fboxsep{6pt}
\setlength\fboxrule{1pt}
\renewcommand*\oldstylenums[1]{{\fontfamily{fxlj}\selectfont #1}}
%% Set table margins.
{\renewcommand{\arraystretch}{2}
\renewcommand{\tabcolsep}{0.4cm}}

\renewcommand{\thesubsection}{\alph{subsection})}
\newcommand{\mysection}[1]{\section*{#1} \setcounter{subsection}{0}}


% ------------------------------------------
% -------- hyphenation rules
% ------------------------------------------
\hyphenation{}

\author{Bastian Huber\\(51432) \and Sebastian Rainer\\(50882) \and Daniel Watzinger\\(51746) \and Benedikt Preis \\(?????)}
\title{\textbf{\huge{\thetitle}}\\\Large\textsc{Team Amazonen}\\\large\textsc{Gruppe ?}}
\date{\today}

\begin{document}

% Specify hyphenation rules.
\hyphenation{}

\maketitle

\pagestyle{fancy}

\mysection{Aufgabe 1: Farbmodelle}
\subsection{} \textit{Was ist ein Farbmodell?}

	Ein Farbmodell verbindet die subjektive Wahrnehmung von Farben mit messbaren Gr��en. Da prinzipiell unendlich viele Spektrallichter und
	Farbmischungen existieren stellt ein Farbmodell eine n�tige Abstraktion der realen Welt dar. Ziel ist es einen m�glichst gro�en Bereich
	des Farbempfindens zu erfassen und diesen m�glichst einfach zu beschreiben.
\subsection{} \textit{Beschreiben Sie die wichtigsten Eigenschaften folgender Modelle: RGB, CYMK, HSV.}

	\begin{description}
		\item[RGB] ist ein additives Farbmodell mit den Grundfarben Rot, Gr�n und Blau. Farben enstehen durch mischen der Grundfarben.
		\item[CMYK] ist ein subtraktives Farbmodell mit den Grundfarben Cyan, Magenta, Yellow und Schwarz (Black). Farben enstehen durch mischen der Grundfarben.
		\item[HSV] Die drei Parameter Hue (Farbton), Saturation (S�ttigung) und Brightness (Helligkeit) bestimmen die jeweilige Farbmischung.
			Die Farbt�ne sind auf einem Farbkreis ($0^{\circ}--360^{\circ}$) angeordnet. S�ttigung und Helligkeit werden im Intervall $[0,1]$ festgelegt.
	\end{description}
\subsection{} \textit{Was sind die Vor- und Nachteile dieser Modelle?}

	\begin{description}
		\item[RGB] einfaches Modell, auf Ger�te (Bildschirm, Anzeige) zugeschnitten, keine Trennung von Luminanz und Farbe, das Farbergebnis steht im Vordergrund
		\item[CMYK] einfaches Modell, keine Trennung von Luminanz und Farbe, Grundlage f�r den Vierfarbdruck
		\item[HSV] Orientiert sich an der menschlichen Farbwahrnehmung und erleichtert somit das intuitive spezifizieren von Farbmischungen,
			um einen Farbton in Graustufen umzurechnen m�ssen im Allgemeinen mehr als ein Parameter ver�ndert werden
	\end{description}
	
\mysection{Aufgabe 2: Bildformate}
\subsection{} \textit{Was Sie die Unterschiede zwischen Vektor- und Rastergrafiken?}

	Bei \textbf{Rastergrafiken} definieren Pixelwerte die einzelnen Bildpunkte und somit das Bild.
	In den Pixeln wird entweder ein konkreter Farbwert oder eine Referenz auf eine Farbtabelle (indizierte Farbe) kodiert.
	Eine Rastergrafik besitzt also eine feste Aufl�sung. Die Skalierung des Bilds
	erfordert eine Interpolation oder Downsampling und ist gegebenenfalls mit einem Qualit�tsverlust verbunden. Pixelbasierte Operationen sowie das Darstellen von komplexen
	Bildinhalten sind hingegen einfach zu realisieren. 
	
	\textbf{Vektorgrafiken} werden mit Hilfe von mathematisch/programmatisch definierten Zeichenanweisungen deklariert und besitzen somit im Gegensatz zu Rastergrafiken keine feste Aufl�sung.
	Eine Vektorgrafik muss deshalb jedesmal f�r eine gew�nschte Aufl�sung gerendert werden. Bei einer Skalierung der Vektorgrafik ensteht nat�rlich kein Qualit�tverlust.
	Einzelne Bildelemente/Gruppierungen k�nnen einfach geometrisch verformt und transformiert werden. Komplexe Bildinhalte (Landschaftsaufnahmen, etc.) k�nnen nur sehr schwer
	mathematisch beschrieben werden.
\subsection{} \textit{Skizzieren Sie die wichtigsten Eigenschaften der Bildformate GIF, PNG und JPEG. Ber�cksichtigen Sie dabei folgende Dimensionen: Farbmodelle, Farbtiefe, Kompression, Dateigr��e.}

	\begin{description}
		\item[GIF] mehrere Bilddeklarationen in einer Datei m�glich, Farbtiefe: 256 indizierte Farben (8-Bit Verweise in eine Farbtabelle), nicht-standardisierte Animation von
			Einzelbildern m�glich, verlustfreie Kompression mit \textit{LZW}, Speicherbedarf mittel
		\item[PNG] unterschiedliche Farbtiefen und Transparenz realisierbar, Graustufen oder Farbbilder, verlustfreie Kompression mittels Vorfilter
			(Differenzen zu Nachbarpixel werden anstatt des jeweiligen Pixelwertes gespeichert) und dem \textit{Deflate-Algorithmus}, Speicherbedarf: niedrig-mittel
		\item[JPEG] verschiedene Farbtiefen sowie Komprimierungsmethoden, skalierbare Kompression, sequentieller oder progressiver Bildaufbau, 
			Kompression: Umwandlung von RGB nach YCbCr $\rightarrow$ Subsampling der Chrominanz- und Luminanzkan�le $\rightarrow$ DCT auf 8x8-Bl�cken der Kan�le $\rightarrow$
			Quantisierung der DCT-Werte (verlustbehaftet) $\rightarrow$ Zig-Zag-Entropiekodierung der Bl�cke, Speicherbedarf: niedrig
	\end{description}

\mysection{Aufgabe 3: Bildmanipulation unter Java}
\subsection{} \textit{Entwerfen Sie eine Java-Klasse, die pixelbasierte Operationen auf ein digitales Bild erm�gliochen soll. Achten Sie dabei insbesondere auf folgende Punkte:
\begin{itemize}
	\item Attribute
	\item Grunds�tzliche Methoden
	\item Interne Datenstruktur zur Speicherung der Pixelwerte
\end{itemize}
}
\subsection{} \textit{Vergleichen Sie ihren Design mit der standard Java-Klasse BufferedImage. Sinnvolle Infor-
mationen zu BufferedImage finden Sie in der Java API Dokumentation:
http://download.oracle.com/javase/1.4.2/docs/api/java/awt/image/
BufferedImage.html}



\end{document}
