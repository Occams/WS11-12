\documentclass{article}
\usepackage[T1]{fontenc}
\usepackage[ngerman,english]{babel}
\usepackage[latin1]{inputenc}
\usepackage{libertine,calc,microtype,parskip,lipsum,booktabs,textcomp,csquotes,enumerate,amssymb,vmargin,fancyhdr,fixltx2e,makeidx,listings,ellipsis,remreset,xcolor,lastpage,caption,fancybox,verbatim,amsmath}
\usepackage{graphicx}
\usepackage[pdftex]{hyperref}
\usepackage{amsmath}

% Title
\def\thetitle{Multimedia Datenbanken --- Blatt 03 --- Musterl�sung }

% ------------------------------------------
% -------- xcolor - (Tango)
% ------------------------------------------

\definecolor{LightButter}{rgb}{0.98,0.91,0.31}
\definecolor{LightOrange}{rgb}{0.98,0.68,0.24}
\definecolor{LightChocolate}{rgb}{0.91,0.72,0.43}
\definecolor{LightChameleon}{rgb}{0.54,0.88,0.20}
\definecolor{LightSkyBlue}{rgb}{0.45,0.62,0.81}
\definecolor{LightPlum}{rgb}{0.68,0.50,0.66}
\definecolor{LightScarletRed}{rgb}{0.93,0.16,0.16}
\definecolor{Butter}{rgb}{0.93,0.86,0.25}
\definecolor{Orange}{rgb}{0.96,0.47,0.00}
\definecolor{Chocolate}{rgb}{0.75,0.49,0.07}
\definecolor{Chameleon}{rgb}{0.45,0.82,0.09}
\definecolor{SkyBlue}{rgb}{0.20,0.39,0.64}
\definecolor{Plum}{rgb}{0.46,0.31,0.48}
\definecolor{ScarletRed}{rgb}{0.80,0.00,0.00}
\definecolor{DarkButter}{rgb}{0.77,0.62,0.00}
\definecolor{DarkOrange}{rgb}{0.80,0.36,0.00}
\definecolor{DarkChocolate}{rgb}{0.56,0.35,0.01}
\definecolor{DarkChameleon}{rgb}{0.30,0.60,0.02}
\definecolor{DarkSkyBlue}{rgb}{0.12,0.29,0.53}
\definecolor{DarkPlum}{rgb}{0.36,0.21,0.40}
\definecolor{DarkScarletRed}{rgb}{0.64,0.00,0.00}
\definecolor{Aluminium1}{rgb}{0.93,0.93,0.92}
\definecolor{Aluminium2}{rgb}{0.82,0.84,0.81}
\definecolor{Aluminium3}{rgb}{0.73,0.74,0.71}
\definecolor{Aluminium4}{rgb}{0.53,0.54,0.52}
\definecolor{Aluminium5}{rgb}{0.33,0.34,0.32}
\definecolor{Aluminium6}{rgb}{0.18,0.20,0.21}
\definecolor{Brown}{cmyk}{0,0.81,1,0.60}
\definecolor{OliveGreen}{cmyk}{0.64,0,0.95,0.40}
\definecolor{CadetBlue}{cmyk}{0.62,0.57,0.23,0}

% ------------------------------------------
% -------- vmargin
% ------------------------------------------

%\setmarginsrb{hleftmargini}{htopmargini}{hrightmargini}{hbottommargini}%{hheadheighti}{hheadsepi}{hfootheighti}{hfootskipi}
\setpapersize{A4}
\setmarginsrb{3cm}{1cm}{3cm}{1cm}{6mm}{7mm}{5mm}{15mm}

% ------------------------------------------
% -------- fancyhdr
% ------------------------------------------
%\fancyheadoffset[L]{\marginparsep+\marginparwidth}
\fancyhf{}
\fancyhead[L]{\bfseries{\nouppercase{\thetitle}}}
\fancyhead[R]{\bfseries{Seite \thepage\ von \pageref{LastPage}}}
\renewcommand{\headrulewidth}{0.5pt}
\renewcommand{\footrulewidth}{0pt}
\fancypagestyle{plain}{
\fancyhf{}
\fancyfoot[R]{\bfseries{Seite \thepage\ von \pageref{LastPage}}}
\renewcommand{\headrulewidth}{0pt}
\renewcommand{\footrulewidth}{0pt}
}

% ------------------------------------------
% -------- hyperref
% ------------------------------------------

\hypersetup{
	%breaklinks=true,
	pdfborder={0 0 0},
	bookmarks=true,         % show bookmarks bar?
	unicode=false,          % non-Latin characters in Acrobat’s bookmarks
	pdftoolbar=true,        % show Acrobat’s toolbar?
	pdfmenubar=true,        % show Acrobat’s menu?
	pdffitwindow=true,     % window fit to page when opened
	pdfstartview={FitH},    % fits the width of the page to the window
	pdftitle={Multimedia Datenbanken Übung},    % title
	pdfauthor={Huber Bastian},     % author
    pdfsubject={Übungsblatt},   % subject of the document
    pdfcreator={Huber Bastian},   % creator of the document
    pdfproducer={Huber Bastian}, % producer of the document
    pdfkeywords={Komplexitätstheorie, Passau}, % list of keywords
    pdfnewwindow=true,      % links in new window
    colorlinks=true,       % false: boxed links; true: colored links
    linkcolor=black,          % color of internal links
    citecolor=black,        % color of links to bibliography
    filecolor=magenta,      % color of file links
    urlcolor=DarkSkyBlue           % color of external links
}

% ------------------------------------------
% -------- listings
% ------------------------------------------
 
\lstset{
		breakautoindent=true,
		breakindent=2em,
		breaklines=true,
		tabsize=4,
		frame=blrt,
		frameround=tttt,
		captionpos=b,
		basicstyle=\scriptsize\ttfamily,
		keywordstyle={\color{SkyBlue}},
		%commentstyle={\color{OliveGreen}},
		stringstyle={\color{OliveGreen}},
		showspaces=false,
		%numbers=right,
		%numberstyle=\scriptsize,
		%stepnumber=1, 
		%numbersep=5pt,
		%showtabs=false
		prebreak = \raisebox{0ex}[0ex][0ex]{\ensuremath{\hookleftarrow}},
		aboveskip={1.5\baselineskip},
		columns=fixed,
		upquote=true,
		extendedchars=true
}
\fontsize{3mm}{4mm}\selectfont

% ------------------------------------------
% -------- misc
% ------------------------------------------
\newcommand{\bibliographyname}{Bibliography}
\setcounter{secnumdepth}{3}
\setcounter{tocdepth}{3}
\clubpenalty = 10000
\widowpenalty = 10000
\displaywidowpenalty = 10000
\setlength\fboxsep{6pt}
\setlength\fboxrule{1pt}
\renewcommand*\oldstylenums[1]{{\fontfamily{fxlj}\selectfont #1}}
%% Set table margins.
{\renewcommand{\arraystretch}{2}
\renewcommand{\tabcolsep}{0.4cm}}

% ------------------------------------------
% -------- hyphenation rules
% ------------------------------------------
\hyphenation{}

\author{Bastian Huber\\(51432) \and Sebastian Rainer\\(50882) \and Daniel Watzinger\\(51746) \and Benedikt Preis \\(?????)}
\title{\textbf{\huge{\thetitle}}\\\large\textsc{Gruppe 2}}
\date{\today}

\begin{document}

% Specify hyphenation rules.
\hyphenation{}

\maketitle

\pagestyle{fancy}

\section{Aufgabe 1}
	\begin{itemize}
		\item{Licht} \begin{itemize}
			\item elektromagnetische Strahlung
			\item besteht aus Photonen, die wie Wellen agieren
			\item Amplitude gibt die Helligkeit an
			\item Frequenz gibt den Farbton an
			\item niedrige Frequenz: warme Farbe(rot, gelb)
			\item hohe Frequenz: kalte Farben (blau)
		\end{itemize}
		\item{Auge}
	\begin{itemize}
			\item verschiedene Zapfen, die jeweils auf Licht verschiedener Wellenl�nge reagieren
			\item Staebchen, die die Helligkeit aufnehmen koennen
		\end{itemize}
			\item{Farbmodell}
	\begin{itemize}
			\item abstraktes mathematisches Modell zur Darstellung von Farben
			\item Beruecksichtigt die Eigenschaften des menschlichen Sehsystems
			\item Farbe wird dreidiemnsional dargestellt: (a,b,c), wobei a,b,c $\in$ [0,1]
			\item Gamut = Farbraum
		\end{itemize}
			\item{additives Farbmodell}
	\begin{itemize}
			\item (1,1,1) $\rightarrow$ weiss
			\item (0,0,0) $\rightarrow$ schwarz
			\item basiert auf dem Licht (Mischung der einzelnen Lichtbestandteile)
		\end{itemize}
					\item{subtraktives Farbmodell}
	\begin{itemize}
			\item (1,1,1) $\rightarrow$ schwarz
			\item (0,0,0) $\rightarrow$ weiss
			\item basiert auf Pigmenten
		\end{itemize}
							\item{RGB}
	\begin{itemize}
			\item Rot, Gruen, Blau
			\item additiv
			\item R = G = B $\rightarrow$ Graustufen
			\item Vorteile: 
				\begin{itemize}
					\item aehnlich zum menschlichen Sehsystem
					\item von Computermonitoren benutzt
				\end{itemize}
		\end{itemize}
							\item{CMYK}
	\begin{itemize}
			\item Cyan, Magenta, Yellow, blacK
			\item subtraktiv
			\item CMY $\leftrightarrow$ RGB :
			\begin{itemize}
					\item C = 1 - R
					\item M = 1 - G
					\item Y = 1 - B
				\end{itemize}
			\item K:
			\begin{itemize}
					\item aus praktischen Gruenden
					\item schwarz tritt oft auf
					\item billiger, schwarz direkt zu benutzen anstatt drei Pigmente zu mischen(Tinte)
					\item Mischung gibt glanzloses Schwarz
				\end{itemize}
			\item Vorteile: 
				\begin{itemize}
					\item Drucken
					\item intuitiv
				\end{itemize}
		\end{itemize}
					\item{HSV}
	\begin{itemize}
			\item Hue Saturation Value(Brightness)
			\item Hue = Farbton
			\item Saturation = Farbsaettigung
			\item Value = Helligkeit (Hellwert)
			\item Darstellung als Zylinder
			\begin{itemize}
					\item Hue = Winkel
					\item Saturation = Radius
					\item Value = Hoehe
				\end{itemize}
			\item Value = 0 $\rightarrow$ schwarz
			\item Vorteile: 
				\begin{itemize}
					\item HS und V separat
					\item digitale Bildverarbeitung
					\item Komplementaerfarben leicht zu finden
				\end{itemize}
		\end{itemize}
		
	\end{itemize}


\section{Aufgabe 2}
	\begin{itemize}
		\item Vektorgrafiken
			\begin{itemize}
				\item mathematische und programmatische Zeichenanweisungen in einem Koordinatensystem
				\item bestehend aus graphischen Primitiven (Linien, Kurven, Kreis,...)
				\item exakte geometrische Transformationen (Skalierung, Rotation)
				\item geringer Speicherbedarf
				\item SVG, PDF
			\end{itemize}
		\item Rastergrafiken
			\begin{itemize}
				\item Bildpunkte(Pixel) werden in ein festes Raster eingeteilt
				\item Pixel $\leftrightarrow$ Farbwert (Farbmodell)
				\item Gamut: 24 Bit per Pixel $\leftrightarrow$ 16,7 Mio. Farben
				\item Speicherbedarf $\leftrightarrow$ Kompression
			\end{itemize}
	\end{itemize}
	
Bildformate:

	\begin{itemize}
	\item GIF
		\begin{itemize}
			\item Farbmodell: RGB
			\item Farbtiefe: 8 Bit $\leftrightarrow$ 256 Farben
			\item induziert $\leftrightarrow$ Farbpalette (24 Bit per Pixel)
			\item Kompression: LZW Lempel-Ziv-Welch, verlustfrei
			\item Besonderheiten: animationsfaehig
		\end{itemize}
		\item PNG
		\begin{itemize}
			\item Farbmodell: RGB oder RGBA (RGB mit Transparenz)
			\item Farbtiefe: variabel, Graustufen (1-16 Bit per Pixel)
			\item induzierter Modus (vgl. GIF) aus Farbpalette (24 Bit per Pixel) mit 1-8 Bit per Pixel
			\item True Color (24-48 Bit per Pixel)
			\item Kompression: Deflate (effizienter als GIF)
		\end{itemize}
		\item JPEG
		\begin{itemize}
			\item Farbmodell: RGB (YCbCr)
			\item Farbtiefe: 24 Bit per Pixel (unterschiedliche Modi moeglich) 
			\item Kompression: DCT basiert, verlustbehaftet
		\end{itemize}
\end{itemize}
\section{Aufgabe 3}
\begin{itemize}
	\item Attribute
	\begin{itemize}
	\item interne Repraesentation des Rasters
	\item Hoehe, Breite, Farbkanaele
\end{itemize}
	\item Methoden
	\begin{itemize}
	\item Konstuktor
	\item get/setPixel(x,y)
	\item get/setSample(x,y,b)
\end{itemize}
\item Raster naiv mit dreidimensionalem Array
\item Speicherverbauch
\begin{itemize}
	\item theoretisch: Hoehe*Breite*Farbkanaele
	\item praktisch: viel hoeher, wegen dreidimensionalem Array (Array von Arrays in Java)
\end{itemize}
\item Optimierung des Rasters
\begin{itemize}
	\item eindimensionales Array
	\item Pixel als Integerwert
	\item Linearisierung des Rasters
\end{itemize}
\item BufferedImage
\begin{itemize}
	\item aehnlich, nur mit mehr Methoden
	\item eigene Klasse fuer Raster
\end{itemize}
\end{itemize}

\end{document}
